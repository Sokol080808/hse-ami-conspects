\section{Последовательности}

\begin{defn}{Последовательность}{seq}
    Последовательность --- это \( f : \NN \rightarrow \RR \).

    \( f(n) \) обозначается \( a_n \)

    Сама последовательность обозначается \( \{ a_n \} \)
\end{defn}

Примеры:
\begin{itemize}
    \item
        \( a_n = \frac{1}{n} \)
    \item
        \( F_1 = F_2 = 1, \ F_n = F_{n - 2} + F_{n - 1} \)
    \item 
        \( a_1 = 1, \ a_n = \frac{1}{a_{n - 1} + 1} \)
    \item 
        \( a_1 = 1, a_2 = 1.4, a_3 = 1.41, \ldots \) --- десятичное приближение числа
\end{itemize}

Из последнего примера видно, что \( \QQ \) всюду плотно.

\subsection{Предел}

\begin{defn}{Предел 0}{limit1}
    Последовательность \( \{ a_n \} \) стремится к \( A \), если

    \[
        \forall U(A) \ \exists N : \forall n > N \ a_n \in U(A)
    \]
\end{defn}

\begin{defn}{Предел 1}{limit2}
    Последовательность \( \{ a_n \} \) стремится к \( A \), если 

    \[
        \forall \varepsilon > 0 \ \exists N : \forall n > N \ a_n \in U_\varepsilon(A)
    \]
\end{defn}

\begin{defn}{Отсутствие предела}{}
    \( A \) не является пределом последовательности \( \{ a_n \} \), если

    \[
        \exists \varepsilon > 0 \ \forall N : \exists n > N \ a_n \notin U_\varepsilon(A)
    \]

    Чтобы получить отсутствие предела, нужно добавить в начало \( \forall A \)
\end{defn}

\begin{example}{}{}
    \[ 
        \lim_{n \to \infty} \frac{1}{n} = 0
    \]
\end{example}

Доказательство:

\[
    \frac{1}{n} < \varepsilon \Leftrightarrow n > \frac{1}{\varepsilon} \Rightarrow N = \left[ \frac{1}{\varepsilon} \right]
\]

\begin{example}{}{}
    \( b_n = (-1)^n + \frac{1}{n} \) не имеет предела
\end{example}

Доказательство:

Пусть 
\( 
    A_n = \lim\limits_{n \to \infty} b_n 
    \Rightarrow
    \forall \varepsilon > 0 \ \exists N : \forall n > N \ |b_n - A| < \varepsilon
\)

Возьмём \( \varepsilon = \frac{1}{2} \):

\begin{gather*}
    |b_n - b_{n + 1}| \leq |b_n - A| + |A - b_{n + 1}| < 2 \varepsilon = 1
    \\
    \left| (-1)^{n} + \frac{1}{n} - (-1)^{n + 1} - \frac{1}{n + 1} \right| 
        = \left| \pm 2 + \frac{1}{n} - \frac{1}{n + 1} \right| > 1
\end{gather*}

Основные свойства пределов:
\begin{enumerate}
    \item
        Пусть \( A = \lim\limits_{n \to \infty} a_n \)
        Тогда \( A \) единственно.

        Доказательство:
        
        Пусть \( \exists B \neq A : B = \lim\limits_{n \to \infty} \)

        \( \forall \varepsilon > 0: \)

        \begin{gather*}
            \exists N_1 : \forall n > N_1 \ |a_n - A| < \varepsilon
            \\
            \exists N_2 : \forall n > N_1 \ |a_n - B| < \varepsilon
        \end{gather*}

        Получим противоречие: 
        \begin{gather*}
            \varepsilon = \frac{|B - A|}{2}, \ N = \max\{ N_1, N_2 \} \Rightarrow
            \\
            \forall n > N 2 \varepsilon = |B - A| = |B - a_n| + |a_n - A| < 2 \varepsilon
        \end{gather*}
    \item
        \begin{lemma}{Лемме об отделимости}{}
            Пусть
            \(
                A = \lim\limits_{n \to \infty} a_n, \ A \neq 0 \Rightarrow \exists N : \forall n > N \ |a_n| > \frac{|A|}{2}
            \)
        \end{lemma}
        Доказательство:

        Возьмём \( \varepsilon = \frac{|A|}{2} \Rightarrow \exists N : \forall n > N \ |a_n - A| < \frac{|A|}{2} \)
\end{enumerate}

\begin{defn}{Ограниченная последовательность}{}
    Последовательность \( \{ a_n \} \) ограничена, если
    \[
        \exists c, C \in \RR : \forall n \in \NN \ c < a_n < C
    \]
    или
    \[
        \exists M \in \RR : \forall n \in \NN \ |a_n| \leq M
    \]
\end{defn}

\begin{lemma}{}{}
    \( \{ a_n \} \) имеет предел \( \Rightarrow \{ a_n \} \) является ограниченной
\end{lemma}

Доказательство: тривиально

\begin{thrm}{Арифметика пределов}{}
    Пусть \( \lim\limits_{n \to \infty} a_n = A, \lim\limits_{n \to \infty} b_n = B \)

    \begin{enumerate}
        \item
            \( \forall \alpha, \beta \in \RR \ \lim\limits_{n \to \infty} (\alpha a_n + \beta b_n) = \alpha A + \beta B \)
        \item 
            \( \lim\limits_{n \to \infty} a_n b_n = AB \)
        \item
            Если \( \forall n \ b_n \neq 0 \), \( \lim\limits_{n \to \infty} \frac{a}{b} = \frac{A}{B} \)
    \end{enumerate}
\end{thrm}

Доказательство:

\(
    \forall \varepsilon_1 > 0 \ \exists N_1 \in \NN \land \exists N_2 \in \NN : 
    \forall l > N_1, \forall k > N_2 \ |a_l - A| < \varepsilon_1, |b_k - B| < \varepsilon_1
\)

\begin{enumerate}
    \item
        \begin{gather*}
            \forall \varepsilon > 0 \ \exists N \in \NN : 
                \forall n > N | \alpha a_n + \beta b_n - \alpha A - \beta B | < \varepsilon
            \\
            \text{Пусть} \ N = \max \{ N_1, N_2 \}
            \\
            | \alpha a_n + \beta b_n - \alpha A - \beta B | \leq
                | \alpha a_n - \alpha A | + |\beta b_n - \beta B | <
                    ( |\alpha| + |\beta| ) \varepsilon_1
        \end{gather*}

        Если \( |\alpha| + |\beta| = 0 \), то очевидно, иначе положим 
        \(
            \varepsilon_1 = \frac{\varepsilon}{|\alpha| + |\beta|}
        \)
    \item
        \begin{gather*}
            \forall \varepsilon > 0 \ \exists N \in \NN : 
                \forall n > N | a_n b_n - A B | < \varepsilon
            \\
            \text{Пусть} \ N = \max \{ N_1, N_2 \}
            \\
            | a_n b_n - A B | = |a_n b_n - A b_n + A b_n - A B| \leq 
                | b_n | |a_n - A| + | A | | b_n - B | <
            \\
            < ( | b_n | + | A | ) \varepsilon_1
        \end{gather*}

        Так как \( \exists \lim\limits_{n \to \infty} b_n \), то
        \( \exists C > 0 : \forall n \in \NN \ | b_n | \leq C \).
        
        Тогда \( | a_n b_n - A B | < ( C + | A |) \varepsilon_1 \).
        Возьмем \( \varepsilon_1 = \frac{\varepsilon}{ C + | A |} \).
    \item 
        Покажем, что \( \lim\limits_{n \to \infty} \frac{1}{b_n} = \frac{1}{B} \)

        Так как \( \lim\limits_{n \to \infty} b_n = B \neq 0 \), то
        \( \exists N' \ \forall n > N' \ | b_n | > \frac{|B|}{2} \).

        Пусть \( N = \max \{ N_2, N' \} \)

        \begin{gather*} 
            \forall \varepsilon > 0 \ \exists N \in \NN : 
                \forall n > N \left| \frac{1}{b_n} - \frac{1}{B} \right| < \varepsilon
            \\
            \left| \frac{1}{b_n} - \frac{1}{B} \right|
            =
            \frac{ |B - b_n| }{ |B| |b_n| }
            <
            \frac{\varepsilon_1}{|B| |b_n|}
            <
            \frac{2 \varepsilon_1}{|B|^2}
        \end{gather*}

        Возьмем \( \varepsilon_1 = \frac{B^2}{2 \varepsilon} \)

        Тогда
        \[
            \lim\limits_{n \to \infty} \frac{a_n}{b_n}
            =
            \lim\limits_{n \to \infty} a_n
            \cdot
            \lim\limits_{n \to \infty} \frac{1}{b_n}
            =
            \frac{A}{B}
        \]
\end{enumerate}

\begin{thrm}{Предельный переход в неравенствах}{} 
    Пусть \( \lim\limits_{n \to \infty} a_n = A, \lim\limits_{n \to \infty} b_n = B \)

    \( \exists n_0 \in \NN : \forall n > n_0 \ a_n \leq b_n \Rightarrow A \leq B \)
\end{thrm}

Доказательство:

Пусть \( A > B \Rightarrow A - B = \varepsilon > 0 \).

Возьмем \( U_\frac{\varepsilon}{3} (A), U_\frac{\varepsilon}{3} (B) \).
\begin{gather*}
    \exists N_1 : \forall n > N_1 \ a_n \in U_\frac{\varepsilon}{3} (A)
    \\
    \exists N_2 : \forall n > N_2 \ b_n \in U_\frac{\varepsilon}{3} (B)
\end{gather*}

Тогда \( \forall n > \max \{ N_1, N_2 \} \ a_n > b_n \) --- противоречие.

\begin{thrm}{Теорема о двух милиционерах}{}
    Пусть \( \lim\limits_{n \to \infty} a_n = \lim\limits_{n \to \infty} b_n = A \)
    
    Пусть
    \(
        \exists n_0 : \forall n > n_0 \ a_n \leq c_n \leq b_n \Rightarrow \lim\limits_{n \to \infty} c_n = A
    \)
\end{thrm}

Доказательство:

\begin{align*}
    & \forall \varepsilon > 0
    \\
    & \exists N_1 \in \NN : \forall n > N_1 \ |a_n - A| < \varepsilon
    \\
    & \exists N_2 \in \NN : \forall n > N_2 \ |b_n - A| < \varepsilon
    \\
    & N = \max \{ N_1, N_2 \} \Rightarrow \forall n > N \ A - \varepsilon < a_n \leq c_n \leq b_n < A + \varepsilon
\end{align*}

\subsection{Бесконечные малые последовательности}

\begin{defn}{Бесконечно малая последовательность}{}
    \[
        \{ a_n \} - \text{БМП, если} \ \lim\limits_{n \to \infty} a_n = 0
    \]
\end{defn}

\begin{defn}{Предел 2}{}
    \[
        \lim\limits_{n \to \infty} b_n = B \Leftrightarrow \exists \ \text{БМП} \ \{ a_n \} : b_n = B + a_n
    \]
\end{defn}

\begin{lemma}{}{}
    Все определения пределов равносильны
\end{lemma}

Доказательство:

В эту сторону очевидно, в эту сторону тривиально.

\begin{defn}{Верхние и нижние грани}{}
    Пусть \( A \subset \RR, A \neq \varnothing \).

    Если существует \( \exists C \in \RR \ \forall a \in A \ a \leq C \), 
    то \( C \) называется верхней гранью множества \( A \).


    Если существует \( \exists c \in \RR \ \forall a \in A \ a \geq c \), 
    то \( c \) называется нижней гранью множества \( A \).

    Наименьшая из верхних граней множества \( A \) называется точной верхней гранью
    и обозначается \( \sup \).

    Наибольшая из нижних граней множества \( A \) называется точной нижней гранью
    и обозначается \( \inf \).
\end{defn}

\begin{defn}{Супремум}{}
    \( C = \sup A \), если:
    \begin{itemize}
        \item
            \(
                \forall a \in A \ a \leq C
            \)
        \item
            \(
                \forall \varepsilon > 0 \ \exists a \in A : a > C - \varepsilon
            \)
    \end{itemize}
\end{defn}

\begin{defn}{Инфимум}{}
    \( c = \inf A \), если:
    \begin{itemize}
        \item
            \(
                \forall a \in A \ a \geq c
            \)
        \item
            \(
                \forall \varepsilon > 0 \ \exists a \in A : a < c + \varepsilon
            \)
    \end{itemize}
\end{defn}

\begin{thrm}{}{}
    \[
        A \in \RR, A \neq \varnothing, \exists \ \text{верхняя грань} \ C \Rightarrow \exists \sup A
    \]
\end{thrm}
Доказательство:

Пусть \( B \) --- множество всех верхних граней \( A \).
Так как \( C \in B \), то \( B \neq \varnothing \).
Поскольку \( A \leq B \), по принципу полноты существует разделяющий элемент \( c \)
--- нетрудно убедиться, что он будет единственным и что он будет точной верхней гранью.

\begin{defn}{Монотонная последовательность}{}
    Если \( \forall n \in N \ a_n \leq a_{n + 1} \), то
    \( \{ a_n \} \) называется неубывающей, если \( a_n < a_{n + 1} \) --- возрастающей.

    Если \( \forall n \in N \ a_n \geq a_{n + 1} \), то
    \( \{ a_n \} \) называется невозрастающей, если \( a_n > a_{n + 1} \) --- убывающей.
\end{defn}

Будем обозначать убывающую \( \downarrow \downarrow \), невозрастающую \( \downarrow \).
Аналогично возрастающую \( \uparrow \uparrow \) и неубывающую \( \uparrow \).

\begin{thrm}{Теорема Вейерштрасса}{}
    Если \( \{ a_n \} \) монотонна и ограничена, то у нее есть предел.
\end{thrm}

Доказательство:

Не умаляя общности, \( \{ a_n \} \uparrow \). 
\[
    \exists \sup \{ a_n \} \Rightarrow \forall \varepsilon > 0 \ \exists n_0 \in \NN :
    \sup \{ a_n \} - \varepsilon < a_{n_0} \leq \sup \{ a_n \} < \sup \{ a_n \} + \varepsilon
\]

Но последовательность не убывает, поэтому
\[
    \forall n \geq n_0 \ |a_n - \sup \{ a_n \}| < \varepsilon \Rightarrow \exists \lim\limits_{n \to \infty} a_n
\]

Если \( \{ a_n \} \downarrow \), рассмотрим \( \{ -a_n \} \)

\begin{example}{}{}
    \( c > 0, a_1 = 1, a_n = \sqrt{c + a_{n - 1}} \)
\end{example}

Пусть \( \lim\limits_{n \to \infty} a_n = A \)
\begin{gather*} 
    \lim\limits_{n \to \infty} a_n = A = \lim\limits_{n \to \infty} \sqrt{c + a_{n - 1}} = \sqrt{c + A}
    \\
    A = \sqrt{c + A}
    \\
    \begin{cases}
        A^2 - A - C = 0
        \\
        A \geq 0
    \end{cases}
    \\
    A = \frac{1 + \sqrt{1 + 4c}}{2}
\end{gather*}

Докажем, что \( \exists A \).
\begin{enumerate}
    \item
        \( a_1 = 1 < \sqrt{c + 1} = a_2 \)
        
        \( a_{n - 1} < a_n \)

        \( a_n = \sqrt{c + a_{n - 1}} < \sqrt{c + a_n} = a_{n - 1} \)
    \item
        \( a_1 \leq 10 + 2c \)

        \( a_{n + 1} = \sqrt{c + a_n} \leq \sqrt{c + 10 + 2c} < 10 + 2c \)
\end{enumerate}

\newpage

\begin{example}{}{}
    \[
        a_1 = a > 1, a_{n + 1} = \frac{1}{2} \left( a_n + \frac{a}{a_n} \right)
    \]
\end{example}

\begin{enumerate}
    \item
        \[
            a_{n + 1} \geq \sqrt{a_n \cdot \frac{a}{a_n}} = \sqrt{a}
        \]
    \item
        Теперь сделаем предельный переход:
        \[
            A = \frac{1}{2} \left( A + \frac{a}{A} \right) \Leftrightarrow A^2 = a \Rightarrow A = \sqrt{a}
        \]
    \item
        \[
            a_{n + 1} = \frac{1}{2} \left( a_n + \frac{a}{a_n} \right) \leq 
            \frac{1}{2} \left( a_n + \frac{a_n^2}{a_n} \right) = a_n
        \]
\end{enumerate}

Оценим скорость сходимости:
\begin{gather*}
    \{ a_n \} \downarrow
    \\
    | a_{n + 1} \pm \sqrt{a} | = \frac{| a_n^2 + a \pm 2 \sqrt{a} a_n |}{2 a_n} = \frac{| a_n \pm \sqrt{a} |^2}{2 a_n}
    \\
    \frac{a_1 - \sqrt{a}}{a_1 + \sqrt{a}} = q, 0 < q < 1
    \\
    \frac{a_n - \sqrt{a}}{a_n + \sqrt{a}} = q^{2^{n - 1}} \Leftrightarrow 
        a_n = \frac{\sqrt{a} ( q^{2^{n - 1}} + 1 )}{1 - q^{2^{n - 1}}}
    \\
    | a_n - \sqrt{a} | = \sqrt{a} \frac{2 q^{2^{n - 1}}}{1 - q^{2^{n - 1}}}
\end{gather*}

Это называется ``Итерационная формула Герона''.

\subsection{Число e}

\begin{thrm}{Число \( e \)}{}
    Существует предел этой последовательности:

    \[
        a_n = \left( 1 + \frac{1}{n} \right)^n
    \]
\end{thrm}

\begin{itemize}
    \item
        \( a_n < 3 \) --- доказано в ДЗ
    \item
        Покажем возрастание

        \begin{gather*}
            a_n = \left( 1 + \frac{1}{n} \right)^n = 
                2 
                + \frac{1}{2!} \left( 1 - \frac{1}{n} \right)
                + \ldots
                + \frac{1}{n!} \left( 1 - \frac{1}{n} \right) \cdots \left( 1 - \frac{n - 1}{n} \right) <
            \\
            < 
                2
                + \frac{1}{2!} \left( 1 - \frac{1}{n + 1} \right)
                + \ldots
                + \frac{1}{n!} \left( 1 - \frac{1}{n + 1} \right) \cdots \left( 1 - \frac{n - 1}{n + 1} \right) <
            \\
            < a_{n + 1}
        \end{gather*}
\end{itemize}

\begin{defn}{Число e}{number-e}
    \[
        \lim\limits_{n \to \infty} \left( 1 + \frac{1}{n} \right)^n = e
    \]
\end{defn}

\begin{defn}{}{}
    \[
        \lim\limits_{n \to \infty} p_n = +\infty 
        \Leftrightarrow
        \forall M \ \exists N : \forall n > N \ p_n > M
    \]
\end{defn}

\begin{lemma}{}{}
    \begin{gather*}   
        \lim\limits_{n \to \infty} p_n = +\infty
        \\
        \lim\limits_{n \to \infty} q_n = -\infty
    \end{gather*}

    Тогда
    \[
        \lim\limits_{n \to \infty} \left( 1 + \frac{1}{p_n} \right)^{p_n} = 
        \lim\limits_{n \to \infty} \left( 1 + \frac{1}{q_n} \right)^{q_n} =
        e
    \]
\end{lemma}

\begin{enumerate}
    \item    
        Пусть \( n_k \in \NN, \lim\limits_{n \to \infty} n_k = +\infty \)
        \begin{gather*}
            \forall \varepsilon > 0 \ \exists N : \forall n > N \ \left| \left( 1 + \frac{1}{n} \right)^n - e \right| < \varepsilon
            \\
            \exists K : \forall k > K \ n_k > N \Rightarrow \left| \left( 1 + \frac{1}{n_k} \right)^n_k - e \right| < \varepsilon
        \end{gather*}

        Так как \( \lim\limits_{n \to \infty} p_n = + \infty \), 
        существует последовательность
        \[
            \{ n_k \} : n_k \leq p_k < n_k + 1
        \]
        .

        Тогда
        \begin{gather*}
            \forall k > K
            \\
            \frac{ \left( 1 + \frac{1}{n_k + 1} \right)^{n_k + 1} }{1 + \frac{1}{n_k + 1}}
                \leq
                \left( 1 + \frac{1}{p_k} \right)^{p_k}
                \leq
                \left( 1 + \frac{1}{n_k} \right)^{n_k} \cdot \left( 1 + \frac{1}{n_k} \right)         
        \end{gather*}

        По теорема о зажатой последовательности \( \lim\limits_{n \to \infty} \left( 1 + \frac{1}{p_n} \right)^{p_n} = e \)
    \item
        \( \exists N_1 : \forall n > N_1 \ q_n < -1 \)

        Пусть \( q_n = -\beta_n, \beta_n > 1 \ \forall n > N_1 \)

        \begin{gather*}
            \left( 1 + \frac{1}{q_n} \right)^{q_n} = \left(1 - \frac{1}{\beta_n} \right)^{-\beta_n}
            =
            \left( \frac{\beta_n}{\beta_n - 1} \right)^{\beta_n} 
            =
            \\
            =
            \left( 1 + \frac{1}{\beta_n - 1} \right)^{\beta_n - 1} \left( 1 + \frac{1}{\beta_n - 1} \right)
            \to
            e
        \end{gather*}
\end{enumerate}

\subsection{Подпоследовательности и частичные пределы}

\begin{defn}{Подпоследовательность}{subseq}
    Пусть \( n_1 < n_2 < \ldots < n_k < \ldots \) --- последовательность натуральных чисел
    и \( \lim\limits_{n \to \infty} n_k = +\infty \)

    Пусть \( \{ a_n \} \) --- последовательность, тогда последовательность \( b_k = a_{n_k} \) 
    называют подпоследовательностью \( \{ a_n \} \)
\end{defn}

\begin{defn}{Частичный предел}{}
    Число \( B \) называют частичным пределом, если \( \exists b_k = a_{n_k} : \lim\limits_{n \to \infty} b_k = b \)
\end{defn}

\begin{lemma}{}{}
    \[
        \lim\limits_{n \to \infty} a_n = A \Rightarrow \forall \{ b_k \} \lim\limits_{n \to \infty} b_k = A
    \]
\end{lemma}

Доказательство:
\begin{gather*}
    \forall \varepsilon > 0 \ \exists N : \forall n > N |a_n - A| < \varepsilon
    \\
    \{ n_k \} \uparrow \uparrow \ \Rightarrow \exists K : \forall k > K \ n_k > N 
        \Rightarrow
        \forall k > K \ | b_k - A | < \varepsilon
\end{gather*}

\begin{thrm}{Теорема Больцано-Вейерштрасса}{}
    Если \( \{ a_n \} \) ограничена, то у нее есть частичный предел.
\end{thrm}

Доказательство:
\begin{gather*}
    \exists c > 0 : \forall n \in N \ | a_n | \leq C
    \\
    I_1 = [ -C , C ]
\end{gather*}

Разобьем отрезок пополам, в какой-то половине будет бесконечное число элементов последовательности.
Не умаляя общности это правая половина.
\begin{gather*} 
    I_2 = [ 0, C ]
    \\
    I_3 = \left[ 0, \frac{C}{2} \right] \lor I_3 = \left[ \frac{C}{2}, C \right]
    \\
    \vdots
\end{gather*}

Продолжим так деление, получим последовательность стягивающихся отрезков, так как \( | I_n | = \frac{| I_1 |}{2^{n - 1}} \)

Значит \( \exists ! b : \forall n \in N \ b \in I_n \).

Выберем элементы из \( \{ a_n \} \) из полученных отрезков, предел такой подпоследовательности будет \( b \).

\begin{defn}{Верхний и нижний пределы}{}
    Пусть \( \{ a_n \} \) ограничена. \( M_n = \sup \{ a_{n + 1}, a_{n + 2}, \ldots \} \).

    Очевидно, что \( \{ M_n \} \downarrow \) и \( M_n \geq \inf \{ a_n \} \).
    Значит у \( \{ M_n \} \) есть предел.

    \( \lim\limits_{n \to \infty} M_n = M \) называется верхним пределом последовательности \( \{ a_n \} \)

    Аналогично \( m_n = \inf \{ a_{n + 1}, a_{n + 2}, \ldots \} \). 

    \( \lim\limits_{n \to \infty} m_n = m \) называется нижним пределом последовательности \( \{ a_n \} \)
\end{defn}

\begin{thrm}{}{}
    Пусть \( \{ a_n \} \) ограничена.

    Тогда \( m, M \) --- частичные пределы \( \{ a_n \} \).

    Если \( b \) --- частичный предел, то \( b \in [m, M] \)
\end{thrm}

Доказательство:

Пусть
\begin{gather*}
    \exists n_1 : M_1 - 1 < a_{n_1} \leq M_1
    \\
    \exists n_2 > n_1 : M_{n_1} + \frac{1}{2} < a_{n_2} \leq M_{n_1}
\end{gather*}

Продолжим:
\[
    \exists n_{k + 1} > n_k: M_{n_k} -\frac{1}{k + 1} < a_{n_{k + 1}} \leq M_{n_k}
\]

Получаем \( n_1 < n_2 < \ldots < n_k < \ldots \)
\begin{gather*}
    b_k = a_{n_k} \Rightarrow M_{n_k} - \frac{1}{k + 1} < b_{k + 1} \leq M_{n_k}
    \\
    \lim\limits_{n \to \infty} M_n = M = \lim\limits_{n \to \infty} M_{n_k}
    \\
    \Downarrow
    \\
    \lim\limits_{n \to \infty} \left( M_{n_k} - \frac{1}{k + 1} \right) 
        = \lim\limits_{k \to \infty} b_{k + 1}
        = \lim\limits_{k \to \infty} M_{n_k}
    \\
    \lim\limits_{k \to \infty} b_k = M
\end{gather*}

Для нижнего предела аналогично.

Пусть \( \lim\limits_{l \to \infty} a_{n_l} = a \).
\begin{gather*}
    m_{n_l - 1} \leq a_{n_l} \leq M_{n_l - 1}
    \\
    \Downarrow
    \\
    m \leq a \leq M
\end{gather*}

\begin{thrm}{}{}
    Ограниченная последовательность \( \{ a_n \} \) имеет предел \( \Leftrightarrow m = M \)
    \begin{itemize}
        \item
            \( \Rightarrow \)

            Уже доказано
        \item
            \( \Leftarrow \)
            \begin{gather*}
                m_n \leq a_{n + 1} \leq M_n
                \\
                \lim\limits_{n \to \infty} m_n = m = M = \lim\limits_{n \to \infty} M_n
                \\
                \lim\limits_{n \to \infty} a_{n + 1} = M
            \end{gather*}
    \end{itemize}
\end{thrm}

\begin{example}{}{}
    \begin{enumerate}
        \item
            \( a_n = n \)
            
            \( m = M = +\infty \)
        \item
            \( a_n = (-1)^n n \)

            \( m = -\infty \)

            \( M = +\infty \)
    \end{enumerate}
\end{example}

\begin{defn}{Фундаментальная последовательность}{}
    \( \{ a_n \} \) называется фундаментальной, если
    \[
        \forall \varepsilon > 0 \ \exists N \in \NN : \forall n, m > N \ | a_n - a_m | < \varepsilon
    \]
\end{defn}

\begin{thrm}{Критерий Коши}{}
    \( \{ a_n \} \) имеет предел \( \Leftrightarrow \) \( \{ a_n \} \) --- фундаментальная
\end{thrm}

Доказательство:
\begin{itemize}
    \item
        \( \Rightarrow \)
        \begin{gather*}
            \forall \varepsilon > 0 \ \exists N \in \NN : \forall n > N \ | a_n - A | < \frac{\varepsilon}{2}
            \\
            \Downarrow
            \\
            \forall m, n > N \ | a_n - a_m| \leq | a_n - A | + | a_m - A | < \varepsilon
        \end{gather*}
    \item
        \( \Leftarrow \) 
        \begin{gather*}
            \varepsilon = 1 \Rightarrow \exists N : \forall n > N | a_{N + 1} - a_n | < 1
            \\
            \Downarrow
            \\
            a_{N + 1} - 1 < a_n < A_{N + 1} + 1
        \end{gather*}

        Значит \( \{ a_n \} \) --- ограниченная.

        Значит по теореме Больцано-Вейерштрасса \( \exists \{ a_{n_k} \} : \lim\limits_{k \to \infty} a_{n_k} = A \)

        Проверим, что это предел всей последовательности.
        \begin{gather*}
            \forall \varepsilon > 0 \ \exists K : \forall k > K \ | a_{n_k} - A | < \varepsilon
            \\
            \exists N_1 : \forall n, m > N_1 \ | a_m - a_n | < \varepsilon
            \\
            M = \max \{ N_1, K \}
            \\
            \Downarrow
            \\
            \forall n, k > M \ | a_n - A | \leq | a_n - a_{n_k} | + | a_{n_k} - A | 
                < \varepsilon + \varepsilon = 2 \varepsilon
        \end{gather*}
\end{itemize}

\newpage

\begin{example}{}{}
    \begin{enumerate}
        \item
            \(
                a_1 = 1, a_n = 1 + \frac{1}{1 + a_{n - 1}}
            \)
        \item
            \(
                a_n = \sum\limits_{k = 1}^n \frac{1}{k}
            \)
    \end{enumerate}
\end{example}

\begin{enumerate}
    \item
        \begin{gather*}
            \forall n \in \NN \ a_n \geq 1
            \\
            | a_{n + 1} - a_n | = \left| \frac{1}{1 + a_n} - \frac{1}{1 + a_{n - 1}} \right|
                = \frac{ | a_n - a_{n - 1} | }{ (1 + a_n)(1 + a_{n - 1}) } 
                \leq \frac{ | a_n - a_{n - 1} | }{4}
            \\
            | a_{n + 1} - a_n | \leq \frac{ | a_n - a_{n - 1} | }{4} \leq \frac{ | a_{n - 1} - a_{n - 2} | }{4^2}
                \leq \frac{ | a_1 - a_2 | }{ 4^{n - 1} } = \frac{1}{2} \cdot \frac{1}{4^{n - 1}}
            \\
            m > n 
            \\
            \Downarrow
            \\
            | a_m - a_n | 
                \leq | a_m - a_{m - 1} | + | a_{m - 1} - a_{m - 2} + \ldots + | a_{n + 1} - a_n |
                \leq
            \\
            \leq \frac{1}{2} \left( \frac{1}{ 4^{m - 2} } + \ldots + \frac{1}{ 4^{n - 1} } \right)
                < \frac{1}{2} \cdot \frac{1}{4^n} \cdot \frac{3}{4} = \frac{3}{8} \cdot \frac{1}{4^n}
            \\
            \frac{1}{4^n} \leq \frac{1}{1 + 3n} \leq \frac{1}{n} < \varepsilon 
                \Rightarrow N = \left[ \frac{1}{\varepsilon} \right]
        \end{gather*}
    \item \ \\
        Покажем, что критерий Коши не выполняется.
        \begin{gather*}
            \exists \varepsilon > 0 : \forall N \ \exists n, m > N : | a_n - a_m | \geq \varepsilon
            \\
            \varepsilon = \frac{1}{2}, n = N + 1, m = 2N + 2 = 2n
            \\
            | a_m - a_n | = \left| \frac{1}{n + 1} + \frac{1}{n + 2} + \ldots + \frac{1}{2n} \right|
                > \frac{n}{2n} = \frac{1}{2}
        \end{gather*}
\end{enumerate}
