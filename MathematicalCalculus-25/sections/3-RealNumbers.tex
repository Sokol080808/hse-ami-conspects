\section{Вещественные числа}

\subsection{Определение}

Пусть числовое множество \( R \) удовлетворяет следующим свойствам:

\begin{enumerate}[start=0]
    \item {
        Замкнутость относительно основных арифметических операций.
    }
    \item {
        \( \forall a, b, c \in R \ a + (b + c) = (a + b) + c \)    
    }
    \item {
        \( \forall a, b \in R \ a + b = b + a \)
    }
    \item {
        \( \exists 0 \in R : a + 0 = a \)
    }
    \item {
        \( \forall a \in R \ \exists b \in R : a + b = 0 \)
    }
    \item {
        \( \forall a, b, c \in R \ a \cdot (b \cdot c) = (a \cdot b) \cdot c \)
    }
    \item {
        \( \forall a, b \in R \ a \cdot b = b \cdot a \)
    }
    \item {
        \( \exists 1 \in R, 1 \neq 0 : 1 \cdot a = a \)
    }
    \item {
        \( \forall a \in R, a \neq 0 \ \exists b \in R, b \neq 0: a \cdot b = 1 \)
    }
    \item {
        \( \forall a, b, c \in R \ a \cdot (b + c) = a \cdot b + a \cdot c \)
    }
    \item {
        \( \forall a, b \in R \ a \leq b \lor b \leq a \)


        \( a \leq b, \forall c \in R \ a + c \leq b + c \)


        \( a \leq b, \forall c \in R, c \geq 0 \ a \cdot c \leq b \cdot c \)
    }
    \item {
        На \( R \) выполнено свойство полноты.
    }
\end{enumerate}

(Подробнее можно почитать Колмогоров, Фомин "Элементы теории функций и функционального анализа")

Чтобы закончить введение вещественных чисел, нужно привести пример.
Мы будем использовать бесконечные десятичные дроби.
Но в них возможна неоднозначность при построении периода.
Можно приближать число либо слева, либо справа, поэтому получится неоднозначность вида \( 0.(9) = 1.(0) \).
Поэтому просто запретим периоды из \( 9 \).
Любая бесконечная десятичная дробь нами будет записывать так \( a = \pm a_0,a_1 a_2 a_3 \dots \),
где \( a_0 \in \NN \cup {0} = \NN_0 \), а \( a_1, a_2, \ldots \in \{0, 1, \dots, 9\} \).

Тогда \( 0 = \pm 0,(0) \)

\( a \leq b \Leftrightarrow a = b \lor \exists k \in \NN : a_0 = b_0, a_1 = b_1, \dots, a_{k - 1} = b_{k - 1}, a_k < b_k \)

Два числа можно сложить, пользуясь принципом полноты (по сути сечения Додекинда).

\newpage

\begin{thrm}{}{}
    На множестве всех бесконечных десятичных дробей с введённым отношением порядке выполняется свойство полноты.
\end{thrm}

Пусть \( A, B \) --- два непустых подмножества и \( A \leq B \).

Если \( A \) состоит из неположительных чисел, а \( B \) --- из неотрицательных, то \( 0 \) --- разделитель.

Если в \( B \) все числа неотрицательные, то возьмём все дроби, где \( b_0 \) минимально (можем это сделать, так как в любом непустом подмножестве \( \NN \) есть минимум).
Среди этих дробей возьмём все дроби с минимальным \( b_1 \).
Среди дробей с началом \( b_0, b_1, \) возьмем все дроби с минимальным \( b_2 \). 
И так далее.

Получим число \( c = c_0, c_1 c_2 c_3 \ldots \)

Покажем, что это разделитель.
По построению \( \forall b \in B \ c \leq b \).

Если \( \exists a \in A : c < a \), то \( \exists k \in \NN : a_k > c_k = b_k \Rightarrow \exists a \in A, b \in B : a > b \).

Если \( \exists b \in B : b < 0 \), то нужно повторить рассуждения, но искать супремум \( A \).

\begin{coroll}{Аксиома Архимеда}{}
    \( \forall a \in \RR , a > 0 \ \exists n \in \NN : na > 1 \).
\end{coroll}

Т.к. \( a > 0 \), то \( \exists k \in \NN : a_k > 0 \), где \( a = a_0, a_1 a_2 \ldots \).
Тогда \( n = 10^{k + 1} \) подойдет.

\begin{coroll}{}{}
    Между любыми двумя вещественными числами есть рациональное и иррациональное число.

    \( \forall a, b \in \RR : a < b \)
    \begin{enumerate}
        \item {
            \( \exists c \in \QQ : a < c < b \)

            \( b - a > 0 \Rightarrow \exists k \in \NN : k(b - a) > 1 \Rightarrow 2k(b - a) > 2 \Rightarrow \exists m \in \ZZ : a < \frac{m}{2k} < b \)
        }
        \item {
            \( \exists c \in \II : a < c < b \)

            По первому пункту: \( 
                \exists \frac{m}{n} \in \QQ : \frac{a}{\sqrt{2}} < \frac{m}{n} < \frac{b}{\sqrt{2}}
                \Rightarrow
                \exists c = \frac{m}{\sqrt{2} n} \in \II : a < c < b
            \)
        }
    \end{enumerate}
\end{coroll}

\begin{defn}{Отрезок}{segment}
    Пусть \( a, b \in \RR : a \leq b \ [a, b] = \{ x \ \vline \ a \leq x \leq b \} \) --- называется отрезком \( \RR \).
\end{defn}

\begin{lemma}{Лемма о стягивающихся отрезках}{}
    \begin{defn}{Система вложенных отрезков}{}
        Множество отрезков \( M \) называется системой вложенных отрезков, 
        если для любых отрезков \( I_1, I_2 \) верно \( I_1 \subset I_2 \) или \( I_2 \subset I_1 \).
    \end{defn}

    \begin{defn}{Последовательность вложенных отрезков}{}
        Система вложенных отрезков \( M \) называется последовательностью вложенных отрезков,
        если отрезки пронумерованы и для любых двух отрезков 
        отрезок с меньшим номером включает в себя отрезок с большим номером.
    \end{defn}

    \begin{defn}{Последовательность стягивающихся отрезков}{}
        Последовательность вложенных отрезков \( M \) называется последовательностью стягивающихся отрезков, 
        если \( \forall \varepsilon > 0 \ \exists n \in \NN : |I_n| < \varepsilon \)
    \end{defn}

    \begin{enumerate}
        \item
            Если \( M \) --- система вложенных отрезков, то они имеют общую точку.
        \item 
            Если \( M \) --- последовательность стягивающихся отрезков, то общая точка единственна.
    \end{enumerate}
\end{lemma}

Доказательство:
\begin{enumerate}
    \item
        Пусть \( I_n = [l_n, r_n] \).
        Возьмем \( L = \{ l_1, l_2, \ldots, l_n, \ldots \}, \ R = \{ r_1, r_2, \ldots, r_n, \ldots \}\).
        Тогда \( L \leq R \Rightarrow \) по свойству полноты существует разделитель \( c \) --- искомая общая точка.
    \item
        Пусть есть общие точки \( c \neq c' \). Тогда \( \forall n \in \NN \ c', c \in I_n \rightarrow |I_n| \geq |c' - c| \).
        Но тогда неверно утверждение, что существует отрезок сколь угодно малой длины.
\end{enumerate}

\subsection{Ассоциация вещественных чисел с прямой}

Сначала определим положительные числа, Давайте сделаем последовательность стягивающихся отрезков:
\[ [a_0, a_0 + 1], [a_0 + a_1, a_0 + a_1 + 0.1], \ldots \] 
Их общая точка и будет числом \( a \).

Отрицательные числа --- положительные, отражённые относительно \( 0 \).

\begin{defn}{Окрестность}{}
    Окрестностью точки \( c \) называется любой интервал \( (a, b) \) такой, что \( c \in (a, b) \).

    \( U_c \) --- окрестность точки \( c \)

    \( \mathring{U} = U' = (a, c) \cup (c, b) \) --- проколотая окрестность точки \( c \)

    \( U_\varepsilon = (c - \varepsilon, c + \varepsilon) \) --- \( \varepsilon \)-окрестность точки \( c \)

    \( \mathring{U}_\varepsilon = U_\varepsilon' = (c - \varepsilon, c) \cup (c, c + \varepsilon) \) 
    --- проколотая \( \varepsilon \)-окрестность точки \( c \)
\end{defn}
