\section{Числовые ряды}

\subsection{Что это такое?}

\begin{defn}{}{}
    Пусть дана \( \{ a_n \} \).

    Формальная сумма \( a_1 + a_2 + \ldots + a_n + \ldots = \sum\limits_{n = 1}^{\infty} \)
    называется числовым рядом.

    Последовательность \( S_N = \sum\limits_{n = 1}^N a_n \)
    называется последовательностью частичных сумм ряда.

    Элементы последовательности \( \{ a_n \} \) называются элементами ряда.

    Если \( \exists S = \lim\limits_{N \to \infty} S_N \),
    то \( S \) называется суммой ряда и если \( S \) существует, то говорят,
    что ряд сходится.
\end{defn}

\begin{example}{}{}
    \[
        \sum\limits_{n = 0}^{\infty} q^n
    \]
    \begin{enumerate}
        \item
            \( q = 0 \)

            \( S = 0 \) (тут считаем \( \sum\limits_{n = 1}^{\infty} q^n \)
        \item
            \( q = 1 \)

            \( S_N = N + 1 \to +\infty \)
        \item 
            \( S_N = \frac{1 - q^{N + 1}}{1 - q} \)

            \begin{itemize}
                \item
                    \( |q| < 1 \Rightarrow \lim\limits_{N \to \infty} S_N = \frac{1}{1 - q} \)
                \item
                    \( |q| > 1 \Rightarrow \) расходится
                \item
                    \( q = -1 \Rightarrow \) предела тоже нет
            \end{itemize}
    \end{enumerate}
\end{example}

\begin{example}{}{}
    \[
        \sum\limits_{n = 2}^{\infty} = \ln \left( 1 - \frac{1}{n^2} \right)
    \]
\end{example}

Суммирование по Чезаро:
\[
    S = \lim\limits_{n \to \infty} \frac{S_1 + S_2 + \ldots + S_n}{n}
\]

По такому суммированию
\[
    \lim\limits_{n \to \infty} \sum\limits_{n = 1}^{\infty} = (-1)^n = -\frac{1}{2}
\]

Суммирование по Пуассону-Абелю:
\[
    S = \lim\limits_{x \to 1-} \sum\limits_{n = 1}^{\infty} a_n x^n
\]

По такому суммированию
\begin{gather*}
    \sum\limits_{n = 0}^{\infty} (-1)^n x^n = 1 - x + x^2 - x^3 + \ldots = \frac{1}{1 + x}
    \\
    \lim\limits_{x \to 1-} \frac{1}{1 + x} = \frac{1}{2}
\end{gather*}

\( S_n \) имеет предел \( \Leftrightarrow \forall \varepsilon > 0 \exists N \in \NN : \forall n, m > N \ | S_n - S_m | < \varepsilon \)
\begin{gather*}
    m = n + p, p \in \NN
    \\
    \Downarrow
    \\
    | S_{n + p} - S_n | = \left| \sum\limits_{k = n + 1}^{n + p} a_k \right| < \varepsilon
\end{gather*}

\begin{thrm}{Критерий Коши}{}
    Ряд \( \sum\limits_{n = 1}^{\infty} a_n \) сходится
    тогда и только тогда, когда: 
    \[ 
        \forall \varepsilon > 0 \ \exists N \in \NN : \forall n > N, p \in \NN \ 
        \left| \sum\limits_{k = n + 1}^{n + p} a_k \right| < \varepsilon
    \]
\end{thrm}

\newpage

\begin{example}{}{}
    \[
        \sum\limits_{n = 1}^{\infty} \frac{\arctan n}{n^3}
    \]
\end{example}

\begin{gather*}
    0 < \sum\limits_{k = n + 1}^{n + p} \frac{\arctan n}{n^3} 
        < \frac{\pi}{2} \sum\limits_{k = n + 1}^{n + p} \frac{1}{n^3}
        < \frac{\pi}{2} \sum\limits_{k = n + 1}^{n + p} \frac{1}{n^2}
        <
    \\
    < \frac{\pi}{2} \sum\limits_{k = n + 1}^{n + p} \frac{1}{(k - 1)k}
        = \frac{\pi}{2} \left( \frac{1}{n} - \frac{1}{n + p} \right)
        < \frac{\pi}{2} \cdot \frac{1}{n}
\end{gather*}

\( N_{\varepsilon} = \left[ \frac{\pi}{2 \varepsilon} \right] \)

\begin{example}{}{}
    \[
        \sum\limits_{n = 1}^{\infty} \frac{1}{\sqrt{n}}
    \]
\end{example}

\[
    \exists \varepsilon > 0 : \forall N \in \NN : \exists n > N, p \in \NN 
    \left| \sum\limits_{k = n + 1}^{n + p} a_k \right| \geq \varepsilon
\]

Пусть \( n = N + 1, p = 2n \)

\[
    \sum\limits_{k = n + 1}^{2n} \frac{1}{\sqrt{k}} < n \cdot \frac{1}{\sqrt{2n}} = \frac{\sqrt{n}}{\sqrt{2}}
\]


\begin{prop}{}{}    
    \( \sum\limits_{n = 1}^{\infty} a_n \) сходится \( \Rightarrow \lim\limits_{n \to \infty} a_n = 0 \)
\end{prop}

Пусть \( S_0 = 0 \)

Тогда \( \lim\limits_{n \to \infty} a_n = \lim\limits_{n \to \infty} ( S_n - S_{n - 1} ) = S - S = 0 \)

\newpage

\begin{example}{}{}
    \[
        \sum\limits_{n = 1}^{\infty} \left( \frac{n + 5}{n + 10} \right)^n
    \]
\end{example}

\begin{gather*}
    \lim\limits_{n \to \infty} \left( \frac{n + 10}{n + 10} - \frac{5}{n + 10} \right)^n
    =
    \lim\limits_{n \to \infty} \left( 1 + \frac{1}{-\frac{n + 10}{5}} \right)^{ \left( \frac{n + 10}{5} \right) (-5) - 10}
    =
    \\
    =
    \lim\limits_{n \to \infty} \left( 1 + \frac{1}{-\frac{n + 10}{5}} \right)^{ \left( \frac{n + 10}{5} \right) (-5) }
    \cdot
    \left( 1 + \frac{1}{-\frac{n + 10}{5}} \right)^{-10}
    =
    \frac{1}{e^5}
    \\
    \Downarrow
    \\
    \lim\limits_{n \to \infty} a_n = \frac{1}{e^5}
\end{gather*}

Значит ряд расходится.

\begin{thrm}{Критерий сходимости ряда \\ с неотрицательными слагаемыми}{}
    Ряд сходится \( \Leftrightarrow S_n = \sum\limits_{k = 1}^n a_k \) --- огр
\end{thrm}

Доказательство тривиально (по теореме Вейерштрасса).

\begin{prop}{Признак сравнения}{}
    Пусть \( \exists n_0 \in \NN : \forall n > n_0 \ 0 \leq a_n \leq b_n \), 
    где \( a_n \) --- элементы ряда \( \sum\limits_{n = 1}^{\infty} a_n \),
    а \( b_n \) --- элементы ряда \( \sum\limits_{n = 1}^{\infty} b_n \).

    Тогда если \( \sum\limits_{n = 1}^{\infty} b_n \) сходится, то и \( \sum\limits_{n = 1}^{\infty} a_n \) сходится.

    если же \( \sum\limits_{n = 1}^{\infty} a_n \) расходится, то и \( \sum\limits_{n = 1}^{\infty} b_n \) расходится.
\end{prop}

Доказательство:

\( \sum\limits_{n = 1}^{\infty} b_n\) сходится \( \Rightarrow S_n^{(b)} \) ограничено 
\( \Rightarrow S_n^{(a)} \) ограничено \( \Rightarrow \sum\limits_{n = 1}^{\infty} a_n \) сходится.

В другую сторону аналогично.

\begin{example}{}{}
    \[
        \sum\limits_{n = 1}^{\infty} \frac{1}{n^{\varepsilon}}, \varepsilon < \frac{1}{2} 
    \]
\end{example}

\begin{example}{}{}
    \[
        \sum\limits_{n = 1}^{\infty} \frac{1}{n^2}
    \]
\end{example}

При \( n > 1 \):
\[
    \frac{1}{n^2} < \frac{1}{(n - 1)n}
\]

Ряд
\( 
    \sum\limits_{n = 1}^{\infty} \frac{1}{(n - 1)n} = \sum\limits_{n = 1}^{\infty} \left( \frac{1}{n - 1} - \frac{1}{n} \right) 
\)
сходится, значит сходится и наш.

\begin{prop}{Признак разрежения Коши}{}
    Пусть дан ряд \( \sum\limits_{n = 1}^{\infty} a_n \) и \( \{ a_n \} \) не возрастает.

    Тогда 
    \( 
        \sum\limits_{n = 1}^{\infty} a_n \) сходится тогда и только тогда, когда сходится \( \sum\limits_{k = 1}^{\infty} 2^k a_{2^k} 
    \)
\end{prop}

Доказательство:
\begin{gather*}
    2 a_4 + 4 a_8 + 8 a_{16} + \ldots \leq a_1 + a_2 + a_3 + \ldots + a_{2^n}
    \\
    a_1 + a_2 + a_3 + \ldots + a_{2^n} \leq a_1 + 2 a_2 + 4 a_4 + 8 a_8 + \ldots + 2^{n - 1} a_{2^{n - 1}}
\end{gather*}

Отсюда нетрудно доказать, что \( \sum\limits_{k = 1}^{\infty} 2^k a_{2^k} \) сходится
\( \Rightarrow \sum\limits_{n = 1}^{\infty} a_n \) сходится.
Аналогично \( \sum\limits_{k = 1}^{\infty} 2^k a_k \) расходится
\( \Rightarrow \sum\limits_{n = 1}^{\infty} a_n \) расходится.

\begin{example}{}{}
    \[
        \sum\limits_{n = 2}^{\infty} \frac{1}{n \ln n}
    \]
\end{example}

\begin{thrm}{Признак сравнения в предельной форме}{}
    Пусть
    \begin{gather*}
        \exists n_0 : \forall n > n_0 \ a_n \geq 0, b_n > 0
        \\
        \lim\limits_{n \to \infty} \frac{a_n}{b_n} = A \in (0, +\infty)
    \end{gather*}

    Тогда либо \( \sum\limits_{n = 1}^{\infty} a_n, \ \sum\limits_{n = 1}^{\infty} b_n \) оба сходятся,
    либо оба расходятся.
\end{thrm}

Доказательство:

Не умаляя общности условия выполняются с \( n = 1 \).
\begin{gather*}
    \varepsilon = \frac{A}{2} \Rightarrow \exists N \in \NN : \forall n > N
    \\
    \left| \frac{a_n}{b_n} - A \right| < \frac{A}{2} \Leftrightarrow \frac{A}{2} b_n < a_n < \frac{3A}{2} b_n
    \\
    \sum\limits_{n = 1}^{\infty} b_n \ \text{сх.} \Rightarrow \sum\limits_{n = 1}^{\infty} \frac{3A}{2} b_n \ \text{сх.}
        \Rightarrow \sum\limits_{n = 1}^{\infty} a_n \text{сх. по признаку сравнения}
    \\ 
    \sum\limits_{n = 1}^{\infty} b_n \ \text{расх.} \Rightarrow \sum\limits_{n = 1}^{\infty} \frac{A}{2} b_n \ \text{расх.}
        \Rightarrow \sum\limits_{n = 1}^{\infty} a_n \text{расх. по признаку сравнения}
\end{gather*}

\subsection{Абсолютная и условная сходимость ряда}

\begin{defn}{Абсолютная сходимость}{}
    \( \sum\limits_{n = 1}^{\infty} a_n \) называется абсолютной сходящейся, 
    если сходится \( \sum\limits_{n = 1}^{\infty} |a_n| \)
\end{defn}

\begin{defn}{Условная сходимость}{}
    \( \sum\limits_{n = 1}^{\infty} a_n \) называется условно сходящимся,
    если он сходится, но расходится \( \sum\limits_{n = 1}^{\infty} |a_n| \)
\end{defn}

\begin{thrm}{Мажорантный признак Вейерштрасса}{}
    Пусть \( \forall n > N \ |a_n| \leq b_n \)

    \( \sum\limits_{n = 1}^{\infty} b_n \) сходится \( \Rightarrow \sum\limits_{n = 1}^{\infty} a_n \) сходится.
\end{thrm}

\begin{example}{}{}
    \begin{enumerate}
        \item
            \( \displaystyle \sum\limits_{n = 1}^{\infty} \frac{\sin nx}{n \sqrt{n}} \)
        \item
            \( \displaystyle \sum\limits_{n = 1}^{\infty} \frac{x^n}{n^2} \)
    \end{enumerate}
\end{example}

\begin{enumerate}
    \item
        Оценим модуль сверху \( \frac{1}{n \sqrt{n}} \)
    \item
        \begin{itemize}
            \item 
                \( \displaystyle |x| \leq 1 \Rightarrow \frac{x^n}{n^2} \leq \frac{1}{n^2} \)
            \item
                \( \displaystyle |x| > 1 \Rightarrow \lim\limits_{n \to \infty} \frac{x^n}{n^2} = \infty \)
        \end{itemize}
\end{enumerate}

\begin{thrm}{Признак Д'аламбера}{}
    Пусть \( \lim\limits_{n \to \infty} \left| \frac{a_{n + 1}}{a_n} \right| = q \).

    Тогда
    \begin{enumerate}
        \item
            \( q < 1 \Rightarrow \sum\limits_{n = 1}^{\infty} a_n \) сходится
        \item
            \( q > 1 \Rightarrow \sum\limits_{n = 1}^{\infty} a_n \) расходится
        \item
            \( q = 1 \Rightarrow \) может быть что угодно
    \end{enumerate}
\end{thrm}

\begin{enumerate}
    \item
        \( q < 1 \)

        \begin{gather*}
            \exists \alpha < 1 : q < \alpha
            \\
            \exists N : \forall n > N \ \left| \frac{a_{n + 1}}{a_n} \right| < \alpha
            \\
            |a_n| 
                = |a_1| \cdot \left| \frac{a_2}{a_1} \right| \cdot \ldots \cdot \left| \frac{a_n}{a_{n - 1}} \right| 
                < |a_1| \alpha^{n - 1}
            \\
            0 < \alpha < 1 \Rightarrow \sum\limits_{n = 1}^{\infty} \alpha^{n - 1} \ \text{сх.}
                \Rightarrow \sum\limits_{n = 1}^{\infty} a_n \ \text{сх.}
        \end{gather*}
    \item
        \( q > 1 \)

        \( \exists N : \forall n > N \ |a_{n + 1}| > |a_n| \Rightarrow \lim\limits_{n \to \infty} a_n \neq 0 \)
    \item
        Пристально смотрим на
        \begin{itemize}
            \item
                \( \sum\limits_{n = 1}^{\infty} \frac{1}{n} \)
            \item
                \( \sum\limits_{n = 1}^{\infty} \frac{1}{n^2} \)
        \end{itemize}
\end{enumerate}


\begin{example}{}{}
    \[
        \sum\limits_{n = 0}^{\infty} \frac{x^n}{n!} \ x \neq 0
    \]
    
    (на самом деле это экспонента)
\end{example}

\begin{thrm}{Радикальный признак Коши}{}
    Пусть дан \( \sum\limits_{n = 1}^{\infty} a_n \), пусть \( \overline{\lim\limits_{n \to \infty}} \sqrt[n]{|a_n|} = q \)

    Тогда
    \begin{enumerate}
        \item
            \( q < 1 \Rightarrow \) сходится
        \item
            \( q > 1 \Rightarrow \) расходится
        \item
            \( q = 1 \Rightarrow \) что угодно
    \end{enumerate}
\end{thrm}

\begin{enumerate}
    \item
        \( q < 1 \)

        \begin{gather*}
            \exists \alpha : q < \alpha < 1
            \\
            \exists N : \forall n > N \ |a_n| < \alpha^n
        \end{gather*}
    \item
        \( q > 1 \)

        \( \overline{\lim\limits_{n \to \infty}} \sqrt[n]{|a_n|} > 1 \Rightarrow \exists \)
        бесконечно много \( a_n : |a_n| > 1 \)
    \item
        Вновь пристально смотрим на
        \begin{itemize}
            \item
                \( \sum\limits_{n = 1}^{\infty} \frac{1}{n} \)
            \item
                \( \sum\limits_{n = 1}^{\infty} \frac{1}{n^2} \)
        \end{itemize}
\end{enumerate}

\begin{example}{}{}
    \[
        a_n > 0, \ \lim\limits_{n \to \infty} a_n = a \Rightarrow \lim\limits_{n \to \infty} \sqrt[n]{a_1 a_2 \ldots a_n} = a
    \]

    \begin{gather*}
        \lim\limits_{n \to \infty} \frac{a_{n + 1}}{a_n} = a
        \Downarrow
        \lim\limits_{n \to \infty} \sqrt[n]{a_1 \cdot \frac{a_2}{a_1} \cdot \ldots \cdot \frac{a_n}{a_{n - 1}}}
            = \lim\limits_{n \to \infty} \sqrt[n]{a_n} = a
    \end{gather*}
\end{example}

Значит из применимости Д'аламбера следует применимость Коши

\begin{example}{}{}
    Пусть \( \tau(n) \) --- количество делителей

    Исследовать на сходимость \( \sum\limits_{n = 1}^{\infty} \tau(n) x^n \ x > 0 \)
\end{example}

Понятно, что по Д'аламберу проверить нереально. 

Проверим по Коши:
\[
    x \leq \sqrt[n]{\tau(n)} x \leq \sqrt[n]{n} x
\]

Штука посередине сходится к \( x \).

Значит если \( x < 1 \), ряд сходится, если \( x > 1 \), то расходится, а в случае \( x = 1 \) очевидно не сходится.

\begin{thrm}{Признак Гаусса}{}
    Пусть дан ряд \( \sum\limits_{n = 1}^{\infty} a_n \) и \( a_n \geq 0 \ \forall n \in \NN \)

    Пусть
    \[
        \exists \delta > 0 : \frac{a_n}{a_{n + 1}} = \alpha + \frac{\beta}{n} + \gamma_n \cdot \frac{1}{n^{1 + \delta}}
    \]

    причем \( \gamma_n \) --- ограниченная последовательность.

    \begin{enumerate}
        \item
            \( \alpha < 1 \Rightarrow \) расходится
        \item
            \( \alpha > 1 \Rightarrow \) сходится
        \item
            \( \alpha = 1 \)

            \begin{enumerate}
                \item
                    \( \beta > 1 \Rightarrow \) сходится
                \item
                    \( \beta \leq 1 \Rightarrow \) расходится
            \end{enumerate}
    \end{enumerate}
\end{thrm}

\begin{example}{}{}
    \begin{itemize}
        \item
            \begin{gather*}
                \sum\limits_{n = 1}^{\infty} \frac{1}{n}
                \\
                \frac{\frac{1}{n}}{\frac{1}{n + 1}} = 1 + \frac{1}{n} + 0
                \\
                \alpha = 1, \ \beta = 1 \Rightarrow \ \text{расходится}
            \end{gather*}
        \item
            \begin{gather*}
                \sum\limits_{n = 1}^{\infty} \frac{1}{n}
                \\
                \frac{\frac{1}{n}}{\frac{1}{n + 1}} = 1 + \frac{2}{n} + \frac{1}{n^2}
                \\
                \alpha = 1, \ \beta = 2 \Rightarrow \ \text{сходится}
            \end{gather*}
    \end{itemize}
\end{example}

\begin{exercise}{}{}
    \( \sigma : \NN \to \NN \) --- перестановка

    \begin{enumerate}{}{}
        \item
            Теорема Дирихле

            Пусть \( \sum\limits_{n = 1}^{\infty} a_n \) сходится абсолютно.
            Тогда для любой \( \sigma \) \( \sum\limits_{n = 1}^{\infty} a_{\sigma(n)} \) 
            сходится абсолютно к тому же пределу
        \item
            Теорема Римана

            Пусть \( \sum\limits_{n = 1}^{\infty} a_n \) сходится условно.
            Тогда \( \forall A \in \RR \ \exists sigma : \sum\limits_{n = 1}^{\infty} a_{\sigma(n)} = A \).
            \( A \) может быть равно \( \pm \infty, \infty \).
            Так же можно переставить так, чтобы конечно суммы не было.
    \end{enumerate}
\end{exercise}
