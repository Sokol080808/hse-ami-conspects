\section{Множества}

\subsection{Примеры}

Множество - совокупность каких-то элементов.

Описание свойств множества для его задания иногда приводит к парадоксам.
Например множество можно построить с помощью аксиом Цермело-Френкеля
(можно прочитать в Куратовский, Мостовский ``Основы теории множеств'')

Обозначаем так: \( A = \{ x, y, z \} \). Принадлежность элемента --- \( x \in A \).

Мощность множества --- аналог количества для бесконечных множеств.

Основные числовые множества:
\begin{itemize}
    \item
        \( \NN \) --- множество всех натуральных чисел. Начинаем с \( 1 \). (подробно почитать можно в книжке Э. Ландау)
    \item
        \( \ZZ \) --- множество всех целых чисел.
    \item
        \( \QQ \) --- множество всех рациональных чисел.

        \( \QQ = \{ \frac{m}{n} \ \vline \ m \in \ZZ, n \in \ZZ \setminus \{0\} \} \)

        Но есть проблема с \( \frac{1}{2} = \frac{2}{4} = \frac{3}{6} = \cdots \)

        Поэтому \(a \in \QQ \Leftrightarrow a = \frac{m}{n}, m \in \ZZ, n \in \NN, (m, n) = 1 \)
    \item
        \( \II \) --- множество всех иррациональных чисел.
    \item
        \( \RR \) --- множество всех вещественных чисел.
    \item
        \( \CC \) --- множество всех комплексных чисел.
\end{itemize}

\( \QQ \cup \II = \RR \)

\( \NN \subset \ZZ \subset \QQ \subset \RR \subset \CC \)

\subsection{Функции}

\begin{defn}{Декартово произведение}{cartesian-product}
Декартово произведение множеств \( X \) и \( Y \) обозначается \( X \times Y \)
и является множеством всех упорядоченных пар \( (x, y) \), где \( x \in X, y \in Y \).
\end{defn}

\begin{defn}{Функция}{function}
Подмножество \( f \) декартова произведения \( X \times Y \)
называется функцией из \( X \) в \( Y \) (\(f : X \rightarrow Y \)),
если \( \forall x \in X \: \exists ! \ (x, y) \in f \)
\end{defn}

\newpage

\( y = f(x) \).

\( x \) --- прообраз \( y \), \( y \) --- образ \( x \).

\( X \) --- область определения, \( Y \) --- область значений.

\begin{defn}{Сюръекция}{}
    \( f \) --- сюръекция (накрытие) множества \( X \) на множество \( Y \), если
    \[
        \forall y \in Y \: \exists x \in X : f(x) = y
    \]
\end{defn}
\begin{defn}{Инъекция}{}
    \( f \) --- инъекция (вложение) множества \( X \) на множество \( Y \), если
    \[
        f(x_1) = f(x_2) \Rightarrow x_1 = x_2
    \]
\end{defn}
\begin{defn}{Биекция}{}
    \( f \) --- биекция множества \( X \) на множество \( Y \), если она является и сюръекцией, и инъекцией.
\end{defn}

Примеры:
\begin{itemize}
    \item
        \( y = x \), \( \RR \rightarrow \RR \) --- биекция
    \item
        \( y = x^2 \), \( \RR \rightarrow \RR \)
    \item
        \( y = x^2 \), \( \RR \rightarrow [0; +\infty) \) --- сюръекция
    \item
        \( y = x^2 \), \( [0; +\infty) \rightarrow [0; +\infty) \) --- биекция
\end{itemize}

\begin{defn}{Равномощность}{equicardinity}
    Множества равномощны, если между их элементами можно сделать биекцию.
\end{defn}

\begin{thrm}{}{}
    \( \NN \sim \QQ \)
\end{thrm}

\subsection{Принцип полноты}

\begin{defn}{}{}
    Говорят, что числовое множество \( A \) лежит левее числового множества \( B \), если
    \( \forall a \in A, \ b \in B \: a \leq b \)

    Обозначается: \( A \leq B \)
\end{defn}

\begin{defn}{Полнота}{completeness}
    Числовое множество \( C \) обладает свойстом полноты, если
    \begin{gather*}
        \forall A, B \subset C : A, B - \text{не пустые}, \ A \leq B
        \\
        \exists c \in C : \forall a \in A, \ b \in B \ a \leq c \leq b
    \end{gather*}
\end{defn}

\( \NN \) обладает свойством полноты, а \( \QQ \) --- нет.

Пусть \( A = \{ a \in \QQ : a > 0, a^2 < 2 \} \), а \( B = \{ b \in \QQ : b > 0, b^2 > 2 \} \).

\( 0 < b^2 - a^2 = (b - a)(b + a), \ b + a > 0 \Rightarrow a \leq b \).

Пусть наш разделитель --- \( c \)
\begin{enumerate}
    \item
        \( c^2 < 2 \)

        Возьмем \( c' = c + \frac{2 - c^2}{5} \).
        \[
            (c')^2 = c^2 + 2 c \cdot \frac{2 - c^2}{5} + \left( \frac{2 - c^2}{5} \right)^2
            < c^2 + 4 \cdot \frac{2 - c^2}{5} + \frac{2 - c^2}{5} = 2
        \]
    \item
        \( c^2 > 2 \)

        Возьмем \( c' = c - \frac{c^2 - 2}{4} \).
        \[
            (c')^2 = c^2 - 2 c \cdot \frac{c^2 - 2}{4} + \left( \frac{c^2 - 2}{4} \right)^2
            > c^2 - 4 \cdot \frac{c^2 - 2}{4} = 2
        \]
\end{enumerate}

Вышеописанные неравенства верны для \( c < 2 \), но мы можем считать, что это верно, 
так как можно рассматривать эти неравенства только для \( c \), 
довольно близких к \( \sqrt{2} \).

\( \sqrt{2} \) --- разделяющий элемент множеств \( A, B \), но \( \sqrt{2} \notin \QQ \).
