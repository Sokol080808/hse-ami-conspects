\section{Топология прямой}

\subsection{Покрытие и предельные точки}

\begin{defn}{}{}
    Система множеств \( S = \{ U_\alpha \} \) называется покрытием множества \( A \),
    если \( A \subseteq \bigcup\limits_{U_\alpha \in S} U_\alpha \)
\end{defn}

\begin{thrm}{Принцип Бореля-Лебега}{}
    Из любого покрытия отрезка \( [a, b] \) системы интервалов \( S = \{ J_\alpha \} \)
    можно выбрать подсистему в \( S \), состоющую ищ конечного числа интервалов и покрывающую \( [a, b] \).
\end{thrm}

Доказательство:

Предположим обратное.

Делим \( [a, b] \) пополам и выбираем половину, которую нельзя покрыть конечным числом интервалов из \( S \).
Обозначим ее за \( I_1 \). Продолжаем так деление.
Получаем последовательность вложенных отрезков, имеющих ровно одну общую точку \( c \).
\begin{gather*}
    c \in \bigcap\limits_{n = 1}^{\infty} J_n
    \\
    c \in [a, b] \Rightarrow \exists J_\alpha : c \in J_\alpha
\end{gather*}

Пусть \( |J_\alpha| = \varepsilon \).

Пусть \( J_\alpha = ( \beta, \gamma ) \), сделаем \( \varepsilon = \min \{ \frac{c - \beta}{2}, \frac{\gamma - c}{2} \} \)
\( \exists k : |I_k| < \varepsilon \Rightarrow I_k \subset J_\alpha \) --- противоречие

\begin{defn}{Предельная точка 1}{}
    Пусть дано множество \( A \).

    Тогда \( a \) называется предельной точной множества \( A \), 
    если \( \forall \mathring{U}(a) \ \exists b \in A : b \in \mathring{U}(a) \)
\end{defn}

\begin{defn}{Предельная точка 2}{}
    Пусть дано множество \( A \).
    Тогда \( a \) называется предельной точной множества \( A \), 
    если \( \forall U(a) \) существует бесконечно много элементов из \( A \), лежащих в \( U(a) \)
\end{defn}

\begin{exercise}{}{}
    Два определения выше равносильны
\end{exercise}

Частичный предел не всегда является предельной точкой множества значений последовательности.

\begin{example}{}{}
    Пусть \( A = (0, 1) \)

    Тогда множество предельных точек \( A' = [0, 1] \)
\end{example}

\begin{thrm}{Принцип Больцано-Вейерштрасса}{}
    Пусть \( A \) --- бесконечное множество (бесконечное число элементов),
    причем ограниченное (то есть \( \exists C > 0 : \forall a \in A \ |a| \leq C \)

    Тогда существует хотя бы одна предельная точка множества \( A \)
\end{thrm}

Доказательство:

Пусть нет ни одной предельной точки.

Для любой точки \( a \) отрезка \( [-c, c] \) \( \exists U(a) \), 
в которой лежит конечное число элементов \( A \).

Значит \( \displaystyle [-c, c] \subseteq \bigcup\limits_{a \in [-c, c]} U(a) \)

Выделим конечное подпокрытие.
В каждом интервале оттуда конечное число элементов из \( A \), а значит и в \( [-c, c] \) тоже --- проиворечие.

\begin{prop}{}{}
    Пусть \( A \) --- бесконечное несчетное множество.

    Тогда у \( A \) есть предельная точка.
\end{prop}

Доказательство:

Разобием прямую на отрезки \( [n, n + 1] \).
В каком-то отрезке будет хотя бы счетное число элементов.
Применим к нему принцип Больцано-Вейерштрасса --- доказали.

\subsection{Открытые и замкнутые множества}

\begin{defn}{}{}
    Если \( a \in A \) и \( \exists U(a) : U(a) \subset A \), то \( a \) --- внутренняя точка множества \( A \).

    Если \( \forall U(b) \ \exists \) точки из \( A \) и точки из \( \RR \setminus A \),
    лежащие в \( U(b) \), то \( b \) называется граничной точкой множества \( A \).

    Если \( c \in A \) и \( \exists U(c) : U(c) \cap A = c \), то \( c \) называется изолированной.
\end{defn}

\begin{example}{}{}
    \[
        A = [0, 1] \cup (2, 3) \cup {4, 5}
    \]

    Тогда \( A' = [0, 1] \cup [2, 3] \)
    Граничные точки \( \delta A = \{ 0, 1, 2, 3, 4, 5 \} \)
    Изолированные точки: \( \{ 4, 5 \} \)
\end{example}

\begin{defn}{}{}
    Открытое множество --- это такое множество \( A \), что любая его точка внутренняя.
\end{defn}

\begin{example}{}{}
    \[ 
        (0, 1)
    \]
\end{example}

\begin{example}{}{}
    Множество \( B \) замкнутое, если \( \RR \setminus B \) --- открыто.
\end{example}

\begin{exercise}{}{}
    \begin{enumerate}
        \item
            Пусть \( t \in T \) и \( \forall t \in T \ U_t \) --- открытое множество 
            \( \displaystyle \Rightarrow \bigcup_{t \in T} U_t \) открыто.
        \item
            \( \displaystyle \bigcap\limits_{n = 1}^m U_m \) открыто, если \( U_1, \ldots, U_m \) --- открытые
        \item
            Если \( V_t \) --- замкнутые множество, то \( \displaystyle \bigcap\limits_{t \in T} V_t \) --- замкнутое общество
        \item
            \( \displaystyle \bigcup\limits_{n = 1}^m V_m \) замкнуто, если \( V_1, \ldots, V_m \) --- замкнутые
        \item
            Любое открытое множество на прямой можно представить в виде не больше чем счетного объединения интервалов
            (возможно бесконечных)
        \item
            Не существует открытых множеств \( U_1, U_2 \) таких, 
            что
            \begin{gather*}
                U_1, U_2 \neq \RR, U_1, U_2 \neq \varnothing, U_1 \cap U_2 = \varnothing
                \\
                U_1 \cup U_2 = \RR
            \end{gather*}

            (Лобода говорит, что это односвязность, Тимофей, что связность)
        \item
            То же верно и для замкнутых \( V_1, V_2 \)
        \item
            Единственные открытые и замкнутые множества на прямой --- \( \varnothing \) и \( \RR \)
    \end{enumerate}
\end{exercise}

\begin{thrm}{}{}
    Множество \( B \) замкнуто тогда и только тогда, когда \( B' \subseteq B \)
\end{thrm}

Доказательство:
\begin{itemize}
    \item
        \( \Rightarrow \)
        
        \( \RR \setminus B \) --- открыто, поэтому все элементы \( \RR \ B \) внутренние 
        \( \Rightarrow \) не являются предельной точкой \( B \)
        \( \Rightarrow \) все предельные лежат в \( B \)
    \item
        \( \Leftarrow \)

        Все предельные точки лежат в \( B \), 
        поэтому каждая точка \( \RR \setminus B \) содержит окрестность, в которой нет точек \( B \).
        Значит \( \RR \setminus B \) открыто, а значит \( B \) замкнуто.
\end{itemize}

\begin{thrm}{}{}
    Множества частичных пределов ограниченной последовательности \( \{ a_n \} \) замкнуто.
\end{thrm}

Обозначим множество частичных пределов через \( B \).
Пусть \( a \) --- предельная точка \( B \) 
\( \Rightarrow \forall U(a) \) существует бесконечно много точек из \( B \).
Значит \( \forall V(a) \) существует бесконечно много точек \( \{ a_n \} \).
Если \( V(b) \) достаточно мала, то \( V(b) \subseteq U(a) \), так как \( U(a) \) --- открытое.
Это означает, что внутри \( U(a) \) бесконечно много точек из \( \{ a_n \} \), значит \( a \) --- частичный предел.

\subsection{Компакты}

\begin{defn}{}{}
    Множество \( K \subset \RR \) называется компактом, 
    если из любого его покрытия открытыми множествами можно выбрать конечное подпокрытие.
\end{defn}

\begin{thrm}{}{}
    \( K \) --- компакт на \( \RR \) тогда и только тогда, когда выполнено любое из из условий:
    \begin{enumerate}
        \item
            \( K \) ограничено и замкнуто
        \item
            Любое бесконечное подможество \( K \) имеет предельную точку, которая принадлежит \( K \)
    \end{enumerate}
\end{thrm}
