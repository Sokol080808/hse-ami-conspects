\section{Комплексные числа}

\subsection{Определение}

%TODO
[TODO] 05.11 Lection

\subsection{Алгебраическая замкнутость}

\begin{thrm}{}{}
    \( \CC \) алгебраически замкнуто.

    (\( \forall f \in \CC[x] \setminus \CC \Rightarrow \exists \alpha \in \CC : f(\alpha) = 0 \))
\end{thrm}

Доказательство:

Шаг 1:

\begin{gather*}
    \varphi : \CC \to \RR
    \\
    z \to |f(z)|
\end{gather*}

Хотим \( \exists z_0 \in \CC : \forall z \in \CC \varphi(z_0) \leq \varphi(z) \).

\begin{lemma}{}{}
    \( f \in \CC[x] \setminus \CC \)

    \( \forall C > 0 \ \exists R > 0 : ( |z| > R \Rightarrow |f(z)| > C ) \)
\end{lemma}

Зафиксируем \( C > 0 \) и \( f \).
\begin{gather*}
    f
    =
    a_0 + a_1 x + \ldots + a_n x^n
    =
    a_n x^n \left( 1 + \frac{a_{n - 1}}{a_n} \cdot \frac{1}{x} + \ldots + \frac{a_0}{a_n} \cdot \frac{1}{x^n} \right)
    =
    a_n x^n (1 + \omega(x) )
    \\
    |f(z)| = |a_n| |z^n| |1 + \omega(z)| \geq |a_n| R^n (1 - |\omega(z)|)
    \\
    |\omega (z)| =  \left| \frac{a_{n - 1}}{a_n} \cdot \frac{1}{z} + \ldots + \frac{a_0}{a_n} \cdot \frac{1}{z^n} \right|
        \leq
        \left| \frac{a_{n - 1}}{a_n} \right| \cdot \frac{1}{|z|}
            + \ldots
            + \left| \frac{a_0}{a_n} \right| \cdot \frac{1}{|z|^n}
\end{gather*}

Будем, считать, что \( |z| > 1 \). Тогда
\[
    |\omega(z)|
        \leq \left( \left| \frac{a_{n - 1}}{a_n} \right| + \ldots + \left| \frac{a_0}{a_n} \right| \right) \cdot \frac{1}{|z|}
        \leq \left( \left| \frac{a_{n - 1}}{a_n} \right| + \ldots + \left| \frac{a_0}{a_n} \right| \right) \cdot \frac{1}{R}
\]

Последнее можно оценить как \( \frac{1}{2} \) при достаточно большом \( R \).

Вернемся к оценке \( |f(z)| \):
\[
    |f(z)| \geq |a_n| R^n ( 1 - |\omega(z)|) \geq |a_n| \frac{R^n}{2}
\]

Получили несколько ограничений снизу на \( R \), выберем максимальное.

Вернемся к шагу \( 1 \).

Покажем, что если мы нашли минимум внутри диска для \( C = |f(0)| \),
то он будет минимумом на всей плоскости.

И правда, пусть минимум в \( z_0 \).
Тогда \( f(z_0) \leq f(0) = C \leq f(z) \) для любого числа \( z \) вне диска.

Остается найти минимум внутри диска (обозначим диск \(D(R)\)).

Заметим, что \( |f(z)| \geq 0 \),
поэтому у \( |f(z)| \) есть инфимум внутри диска.
\( a = \inf\limits_{z \in D(R)} |f(z)| \)
Он является предельной точкой, поэтому  \( \exists \{ z_n \} : \lim\limits_{n \to \infty} |f(z_n)| = a \).

Если последовательность комплексных чисел ограничена диском,
то у нее можно выбрать сходящуюся подпоследовательность:
\(z_n = a_n + b_n \cdot i \) --- давайте проредим так, чтобы \( a_n \) сходились,
а потом полученную последовательность проредим еще раз, чтобы теперь и \( b_n \) сходились.
Мы можем так сделать, потому что \( a_n, b_n \in [-R, R] \).

Давайте проредим нашу \( \{ z_n \} \) так, чтобы она сходилась к какому-то \( z_0 \).

Получим \( \{ z'_n \} \)

\( a = \lim\limits_{n \to \infty} |f(z'_n)| = |f(\lim\limits_{n \to \infty} z'_n)| = |f(z_0)| \)

Шаг 2:

Есть \( f \in \CC[x] \setminus \CC \)
Уже знаем, что \( \exists z_0 \in \CC : \forall z \in \CC \ |f(z_0)| \leq |f(z)| \).

Если \( |f(z_0)| = 0 \), то мы нашли корень --- это \( z_0 \).

Иначе \( |f(z_0)| > 0 \).
Покажем, что такое невозможно.

Сделаем замену \( f(z) \to f(z + z_0) \).
Теперь минимальный модуль достигается в точке \( 0 \),
поэтому можно считать, что \( z_0 = 0 \).

\( f(z_0) = f(0) = a_0 \neq 0 \).

Поделим многочлен на \( a_0 : f \to \frac{f}{a_0} \).

Тогда \( f = 1 + a_k z^k + \ldots + a_n z^n \), где \( a_k \) --- первая ненулевой коэффициент после \( 1 \).
\( |f(z)| \geq |f(z_0)| = |f(0)| = 1 \).

Сделаем замену \( z \to \alpha z \), где \( \alpha \) --- какое-то комплексное число.

Теперь \( f = 1 + a_k \alpha^k z^k + \ldots + a_n \alpha^n z^n \).

Хотим, чтобы \( a_k \alpha^k = -1 \).

Но мы точно можем такое сделать, потому что \( \alpha^k = -\frac{1}{a_k} \),
очевидно, решается (корень можно найти через тригонометрическую форму).

Теперь \( f(z) = 1 - z^k + a_{k + 1} z^{k + 1} + a_n z^n \) (\( \{ a_n \} \) --- это уже новые коэффициенты).

Заметим, что последняя замена не сдвигает \( 0 \) и является биективной,
поэтому все еще \( |f(z)| \geq |f(0)| = 1 \).
\begin{gather*}
    f(z) = 1 - z^k + z^{k + 1} ( a_{k + 1} + \ldots + a_n z^{n - k - 1} ) = 1 - z^k + z^{k + 1} \omega(z)
    \\
    f(z) = 1 - z^k ( 1 - 2 \omega (z) )
\end{gather*}

Если бы не последнее слагаемое, то мы бы смогли получить \( |f(z)| < 1 \)
с помощью любого \( z \in \RR : 0 < |z| < 1 \).

Покажем, что мы можем сделать последнюю скобку \( < \varepsilon \) с помощью таких \( z \).
\[
    |z \omega(z)|
    =
    |z ( a_{k + 1} + \ldots + a_n z^{n - k + 1} ) |
    \leq
    |z| ( |a_{k + 1}| + \ldots + |a_n z^{n - k + 1}| )
    \leq
    |z| ( |a_{k + 1}| + \ldots + |a_n| )
\]

Последнее может быть сколь угодно малым при \( 0 < |z| < 1 \).

Но тогда \( f(z) = 1 - z^k (1 - z \omega(z)) \) может быть меньше \( 1 \) --- противоречие.
