\section{Векторные пространства}

\subsection{Определение}

\( F \) --- поле.

\begin{itemize}
    \item
        Структура:

        \begin{enumerate}
            \item
                \( V \) --- множество
            \item
                \( + : V \times V \to V \)
            \item
                \( \cdot : F \times V \to V \)
        \end{enumerate}
    \item
        Аксиомы:

        \begin{enumerate}[start = 4]
            \item
                \( (v + u) + w = v + (u + w) \)
            \item
                \( \exists ! 0 \in V : v + 0 = 0 + v = v \)
            \item
                \( \forall v \in V \ \exists ! -v : v + (-v) = (-v) + v = 0 \)
            \item
                \( v + u = u + v \)
            \item
                \( \lambda \cdot (v + u) = \lambda \cdot v + \lambda \cdot u \)
            \item
                \( (\lambda + \mu) \cdot v = \lambda \cdot v + \mu \cdot v \)
            \item
                \( (\lambda \cdot \mu) \cdot v = \lambda \cdot (\mu \cdot v) \)
            \item
                \( 1 \cdot v = v \)
        \end{enumerate}
\end{itemize}

\begin{example}{}{}
    \begin{itemize}
        \item
            \( \RR^n \) (на самом деле \( F^n \))
        \item
            \( M_{mn} (F) \)
        \item
            \( F[x] \)
        \item
            \( \{ f : \RR \to \RR \} \)
    \end{itemize}
\end{example}

\begin{remark}{}{}
    \begin{enumerate}
        \item
            \( 0 \in F, v \in V \Rightarrow 0 \cdot v = 0 \in V \)
        \item
            \( \alpha \in F, 0 \in V \Rightarrow \alpha \cdot 0 = 0 \in V \)
        \item
            \( -1 \in F, v \in V \Rightarrow (-1) \cdot v = -v \)
    \end{enumerate}
\end{remark}

\subsection{Подпространство}

\( V \) --- векторное пространство над \( F \)

Если выполнено:
\begin{itemize}
    \item
        \( U \subseteq V \)
    \item
        \( \forall u_1, u_2 \in U \ u_1 + u_2 \in U \)
    \item
        \(
            \begin{cases}
                \lambda \in F
                \\
                u \in U
            \end{cases}
            \Rightarrow
            \lambda \cdot u \in U
        \)
\end{itemize}

\begin{example}{}{}
    \begin{enumerate}
        \item
            \( V = F^n \) и \( A \in M_mn(F) \)

            \( U = \{ y \in F^n \ \vline \ Ay = 0 \} \)
        \item
            \( U = 0 \)

            \( U = V \)

            (тривиальные подпространства)
        \item
            \( V = \{ f : \RR \to \RR \} \)

            \( U = R[x] \)
    \end{enumerate}
\end{example}

\subsection{Линейные комбинации}

\( \alpha_1 v_1 + \ldots + \alpha_k v_k \in V \) --- линейная комбинация.

Если \( \alpha_1 = \ldots = \alpha_k \), комбинация называется тривиальной.

\begin{defn}{}{}
    \( v_1, v_2, \ldots, v_k \in V \)
    \begin{enumerate}
        \item
            Если
            \(
                \exists (\alpha_1, \ldots, \alpha_k) \neq 0 : \alpha_1 v_1 + \ldots + \alpha_k v_k = 0
            \),
            то вектора называют линейно зависимыми.
        \item
            Вектора называют линейно независимыми (ЛНЗ), если
            \(
                (\alpha_1 v_1 + \ldots + \alpha_k v_k = 0 \Rightarrow \alpha_1 = \ldots = \alpha_k = 0
            \)
    \end{enumerate}
\end{defn}

\newpage

\begin{defn}{}{}
    \( E \subseteq V \)
    \begin{enumerate}
        \item
            \( E \) --- линейно зависимо, если существует конечный набор линейно зависимых векторов из \( E \).
        \item
            \( E \) --- ЛНЗ, если любой конечный набор векторов из \( E \) --- ЛНЗ.
    \end{enumerate}
\end{defn}

\begin{example}{}{}
    \( V = \RR[x] \)

    \( E = \{ 1, x, x^2, \ldots, x^n, \ldots \} \) --- ЛНЗ
\end{example}

\begin{defn}{}{}
    \begin{itemize}
        \item
            \( v_1, \ldots, v_k \in V \) порождают \( V \), если
            \( \forall v \in V \ \exists \alpha_1, \ldots, \alpha_k : v = \alpha_1 v_1 + \ldots + \alpha_k v_k \)
        \item
            \( E \subseteq V \) --- порождающее, если
            \(
                \forall v \in V \ \exists v_1, \ldots, v_k \in E :
                \exists \alpha_1, \ldots, \alpha_k : v = \alpha_1 v_1 + \ldots + \alpha_k v_k
            \)
    \end{itemize}
\end{defn}

\begin{defn}{}{}
    \begin{itemize}
        \item
            \( v_1, v_2, \ldots, v_k \in V \)

            Линейная оболочка \( < v_1, \ldots, v_k > \)
        \item
            \( E \subseteq V \)

            Линейная оболочка
            \(
                <E> = \{ \alpha_1 v_1 + \ldots + \alpha_k v_k
                \ \vline \
                \forall v_1, \ldots, v_k \in E, \ \alpha_1, \ldots, \alpha_k \in F
            \)
    \end{itemize}
\end{defn}

\begin{remark}{}{}
    Линейная оболочка является векторным подпространством
\end{remark}

\begin{thrm}{}{}
    \( V \) --- векторное пространство над \( F \)

    \( E \subseteq V \)

    Тогда эквивалентны следующие утверждения:
    \begin{enumerate}
        \item
            \( E \) --- максимальное ЛНЗ (нельзя добавить еще вектор с сохранением ЛНЗ).
        \item
            \( E \) --- минимальное порождающее (нельзя убрать вектора).
        \item
            \( E \) --- ЛНЗ и порождающее.
    \end{enumerate}
\end{thrm}

\begin{defn}{Базис}{}
    Если \( E \) удовлетворяет свойствам выше, его называют базисом \( V \).
\end{defn}

\begin{itemize}
    \item
        \( (1) \Rightarrow (3) \)

        Рассмотрим \( v \in V \)
        \begin{enumerate}
            \item
                \( v \in E \Rightarrow \) можем получить \( v \) линейной комбинацией из \( E \)
            \item
                \( v \notin E \Rightarrow E \cup \{ v \} \) --- линейно зависимо.
                Тогда существует конечная линейная комбинация векторов из \( E \),
                равная \( 0 \), где участвует \( v \):
                \[
                    \alpha_1 v_1 + \ldots + \alpha_k v_k + \alpha v = 0
                \]

                Если \( \alpha = 0 \), то \( E \) было линейно зависимым --- противоречие.

                Значит \( \displaystyle v = \frac{\alpha_1 v_1 + \ldots + \alpha_k v_k}{-\alpha} \).
        \end{enumerate}

        Значит \( E \) --- порождающее, что и требовалось доказать.
    \item
        \( (3) \Rightarrow (1) \)

        \( V \) --- порождающее \( \Rightarrow \) существует конечная линейная комбинация,
        равная \( v \in V \).
        Поэтому \( E \cup \{ v \} \) --- линейно зависимое.
    \item
        \( (2) \Rightarrow (3) \)

        Пусть \( E \) --- линейно зависимое.
        Тогда \( \exists v_1, \ldots, v_k \in E, \ (\alpha_1, \ldots, \alpha_k) \neq 0 \).

        Не умаляя общности, \( \alpha_1 \neq 0 \).
        Тогда \( \displaystyle v_1 = \frac{\alpha_2 v_2 + \ldots + \alpha_k v_k}{-\alpha_1} \)

        Давайте заменим \( v_1 \) во всех комбинациях на линейную комбинацию выше.
        Тогда \( E \setminus \{ v_1 \} \) --- тоже порождающее --- противоречие.

        Значит \( E \) --- ЛНЗ, что и требовалось доказать.
    \item
        \( (3) \Rightarrow (2) \)

        Пусть \( E \) --- не минимальное порождающее.

        Значит \( \exists v \in V : E \setminus \{ v \} \) --- тоже порождающее.

        Тогда \( v \) можно представить линейной комбинацией из \( E \setminus \{ v \} \).

        Но тогда \( E = (E \setminus \{ v \}) \cup \{ v \} \) --- линейно зависимо --- противоречие.
\end{itemize}

\begin{remark}{}{}
    Как быстро записывать линейные комбинации:
    \begin{gather*}
        v = \begin{pmatrix}
            v_1, \ldots, v_k
        \end{pmatrix}
        \quad
        \alpha = \begin{pmatrix}
            \alpha_1
            \\
            \vdots
            \\
            \alpha_k
        \end{pmatrix}
        \\
        \alpha_1 v_1 + \ldots \alpha_k v_k = v \cdot \alpha
    \end{gather*}
\end{remark}

\subsection{Базис}

\( V \) --- векторное пространство.

Вопросы:
\begin{enumerate}
    \item
        Существует ли базис?
    \item
        Существует ли конечный базис?
    \item
        \( e \) и \( f \) --- базисы.

        Правда ли, что \( |e| = |f| \)?
\end{enumerate}

Ответы:
\begin{enumerate}
    \item
        Если вы верите в аксиому выбора, то можно показать существование базиса в любом векторном пространстве.
        Если же нет, то в некоторых пространствах нельзя ни доказать, ни опровергнуть существование базиса:
        \[
            V = \{ f : \RR \to \RR \}
        \]
    \item
        Если \( V \) имеет конечный базис, то и \( U \subseteq V \) тоже имеет конечный базис.
    \item
        Это правда, но докажем только для конечных базисов.
\end{enumerate}

Доказательство последнего:
\begin{enumerate}
    \item
        Пусть \( v_1, \ldots, v_k \) --- порождающее \( V \).
        Тогда \( u_1, \ldots, u_{k + 1} \in V \) --- линейно зависимые.
    \item
        Пусть \( e_1, \ldots, e_n \) --- базис.

        Тогда \( E \subseteq V \) базис \( \Rightarrow |E| < \infty, \ |E| = n \)
\end{enumerate}

\begin{gather*}
    u_i
    =
    \begin{pmatrix}
        v_1, \ldots, v_k
    \end{pmatrix}
    \cdot
    \begin{pmatrix}
        \alpha_1,
        \\
        \vdots
        \\
        \alpha_k
    \end{pmatrix}
    \\
    \\
    \begin{pmatrix}
        u_1, \ldots, u_{k + 1}
    \end{pmatrix}
    =
    \begin{pmatrix}
        v_1, \ldots, v_k
    \end{pmatrix}
    \begin{pmatrix}
        \alpha_{1 \: 1} & \ldots & \alpha_{1 \: {k + 1}}
        \\
        \vdots & & \vdots
        \\
        \alpha_{k \: 1} & \ldots& \alpha_{k \: {k + 1}}
    \end{pmatrix}
\end{gather*}

Назовем матрицу \( A \). Давайте найдем ненулевое решение \( Ax = 0\)
(такое есть, потому \( A \) --- широкая).

Умножим последнее равенство на \( x \) справа.
Тогда справа получится \( 0 \),
а слева \( ux \).

Но это значит, что \( u \) --- линейно зависимое.

Теперь пусть у нас есть базисы \( e \) и \( f \), причем \( e \) конечный.
По предыдущему утверждению \( |f| \leq |e| \), то есть \( f \) тоже конечный.
Но тогда аналогично \( |e| \leq |f| \). Значит \( |e| = |f| \).

\begin{defn}{}{}
    \( V \) --- векторное пространство над \( F \).

    \( \dim V = |E| \), где \( E \) --- базис.
\end{defn}


\begin{lemma}{}{}
    \( V \) --- векторное пространство над \( F \), причем \( \dim V < \infty \).

    \( U \subseteq V \) --- подпространство.

    Тогда
    \begin{enumerate}
        \item
            \( \dim U < \dim V \)
        \item
            \( U = V \Leftrightarrow \dim U = \dim V \)
    \end{enumerate}
\end{lemma}

Рассмотрим \( E \) --- базис \( U \).
\( E \) --- ЛНЗ в \( V \) \( \Rightarrow \) \( |E| \leq \dim V \).
Но \( \dim U = |E| \).

Докажем вторую часть утверждения.

Слева направо очевидно.

Пусть \( \dim U = \dim V = n \).
Теперь рассмотрим \( e_1, \ldots, e_n \) --- базис \( U \).
Тогда \( U = < e_1, \ldots, e_n > \).
Но с другой стороны \( E \) --- максимальное ЛНЗ в \( V \), поэтому \( E \) является и базисом \( V \).
Но тогда \( V = < e_1, \ldots, e_n > = U \).

\newpage

\subsection{Координаты}

\( V \) --- векторное пространство над \( F \).

\( e_1, \ldots, e_n \) --- базис (фиксированный)

Пусть \( v \in V \)

\( v = x_1 e_1 + \ldots + x_n e_n \), тогда \( x_1, \ldots, x_n \) определяются однозначно.

Иначе
\(
    y_1 e_1 + \ldots + y_n e_n = v = x_1 e_1 + \ldots + x_n v_n \Rightarrow (x_1 - y_1) v_1 + \ldots (x_n - y_n) v_n = 0
\)

откуда по линейной независимости базиса получаем, что \( x_i = y_i \).

Тогда \( x_1, \ldots, x_n \) --- координаты \( v \) в базисе \( e_1, \ldots, e_n \).

Теперь можно заменить \( V \) на \( F^n \)

Пусть есть базисы \( e_1, \ldots, e_n \) и \( f_1, \ldots, f_n \)

\( (e_1, \ldots, e_n) x = v = (f_1, \ldots, f_n) y \), \( x, y \in F^n \).

Для того, чтобы понять, как меняются координаты, нужно понять как связаны два базиса.
В этом поможет матрица перехода:
\[
    f_1
    =
    \begin{pmatrix}
        e_1, \ldots, e_n
    \end{pmatrix}
    \begin{pmatrix}
        c_{1 \: 1}
        \\
        \vdots
        \\
        c_{n \: 1}
    \end{pmatrix}
    \quad \ldots \quad
    f_n
    =
    \begin{pmatrix}
        e_1, \ldots, e_n
    \end{pmatrix}
    \begin{pmatrix}
        c_{1 \: n}
        \\
        \vdots
        \\
        c_{n \: n}
    \end{pmatrix}
\]

Теперь объединим это все в одну матрицу:
\begin{gather*}
    \begin{pmatrix}
        f_1, \ldots, f_n
    \end{pmatrix}
    =
    \begin{pmatrix}
        e_1, \ldots, e_n
    \end{pmatrix}
    \begin{pmatrix}
        c_{1 \: 1} & \ldots & c_{1 \: n}
        \\
        \vdots & & \vdots
        \\
        c_{n \: 1} & \ldots & c_{n \: n}
    \end{pmatrix}
    \\
    \\
    \begin{pmatrix}
        f_1, \ldots, f_n
    \end{pmatrix}
    =
    \begin{pmatrix}
        e_1, \ldots, e_n
    \end{pmatrix}
    C
\end{gather*}

Покажем, что \( C \) --- невырожденная.

Аналогично можем получить
\(
    \begin{pmatrix}
        e_1, \ldots, e_n
    \end{pmatrix}
    =
    \begin{pmatrix}
        f_1, \ldots, f_n
    \end{pmatrix}
    D
\)

Значит
\(
    \begin{pmatrix}
        e_1, \ldots, e_n
    \end{pmatrix}
    =
    \begin{pmatrix}
        e_1, \ldots, e_n
    \end{pmatrix}
    CD
    \Rightarrow
    \begin{pmatrix}
        e_1, \ldots, e_n
    \end{pmatrix}
    (E - CD)
    = 0
\)

В силу линейной независимости \( e_1, \ldots, e_n \),
каждый столбец \( E - CD \) является нулевым.
Значит \( C \) имеет обратную справа, а значит является обратимой.

Отсюда можно получить, что при умножении базиса на обратимую матрицу получается другой базис.

Пусть
\(
    \begin{pmatrix}
        e_1, \ldots, e_n
    \end{pmatrix}
    C
    =
    \begin{pmatrix}
        f_1, \ldots, f_n
    \end{pmatrix}
\)

Покажем, что \( f_1, \ldots, f_n \) --- ЛНЗ.

Пусть
\(
    \begin{pmatrix}
        f_1, \ldots, f_n
    \end{pmatrix}
    x
    =
    0
    \Rightarrow
    \begin{pmatrix}
        e_1, \ldots, e_n
    \end{pmatrix}
    Cx
    =
    0
\)

По аналогичным рассуждениям (как с \( E - CD \)), \( Cx = 0 \).
Но \( C \) --- обратимая, значит \( x = 0 \).

Теперь наконец-то научимся менять координаты:
\begin{gather*}
    \begin{pmatrix}
        e_1, \ldots, e_n
    \end{pmatrix}
    C
    =
    \begin{pmatrix}
        f_1, \ldots, f_n
    \end{pmatrix}
    \\
    \\
    \begin{pmatrix}
        e_1, \ldots, e_n
    \end{pmatrix}
    x
    =
    v
    =
    \begin{pmatrix}
        f_1, \ldots, f_n
    \end{pmatrix}
    y
    =
    \begin{pmatrix}
        e_1, \ldots, e_n
    \end{pmatrix}
    Cy
\end{gather*}

Поскольку координаты определяются однозначно, \( x = Cy \).

\begin{defn}{}{}
    Фyндаментальная система решений --- максимальный набор ЛНЗ решений системы \( Ay = 0 \)

    Иначе говоря, базис \( u = \{ y \in F^n \ \vline \ Ay = 0 \} \)
\end{defn}

Как искать ФСР?

\begin{gather*}
    \begin{pmatrix}
        1 & 0 & 2 & 4 & 0 & -1 & -1
        \\
        0 & 1 & 3 & 5 & 0 & 1 & 1
        \\
        0 & 0 & 0 & 0 & 1 & 0 & 1
    \end{pmatrix}
    x = 0
    \\
    \begin{array}{c c c c c c c c c}
        x_1 & = & -2 & -3 & 1 & 0 & 0 & 0 & 0
        \\
        x_2 & = & -4 & -5 & 0 & 1 & 0 & 0 & 0
        \\
        x_3 & = & 1 & -1 & 0 & 0 & 0 & 1 & 0
        \\
        x_4 & = & 1 & -1 & 0 & 0 & -1 & 0 & 1
    \end{array}
\end{gather*}

Нетрудно видеть, что такие решения линейно-независимы
(потому что для каждой свободной переменной только одно решение имеет ненулевое коэффициент).

Почему оно порождающее?

Очевидно, что в силу построения такого решения мы можем получить решение с любым набором свободных коэффициентов.
А если у решений совпадают все свободные, то и все главные тоже равны.

\newpage

\subsection{Ранг матрицы}

\begin{defn}{Столбцовый ранг}{}
    \begin{gather*}
        \begin{pmatrix}
            A_1 & \ldots & A_n
        \end{pmatrix}
        \\
        rk_\text{столб} \: A = \dim < A_1, \ldots, A_n >
    \end{gather*}
\end{defn}

\begin{defn}{Строчной ранг}{}
    \begin{gather*}
        \begin{pmatrix}
            A_1
            \\
            \vdots
            \\
            A_n
        \end{pmatrix}
        \\
        rk_\text{строч} \: A = \dim < A_1, \ldots, A_n >
    \end{gather*}
\end{defn}

\begin{defn}{Факториальный ранг}{}
    Пусть \( A \) --- матрица \( m \times n \).

    Рассмотрим все возможные разбиения \( A \) на множители размерами \( m \times k \), \( k \times n \).
    Среди всех таких выберем наименьшее \( k \) --- это \( rk_\text{Ф} \).

    Для нулевой матрицы: \( rk_\text{Ф} \: 0 = 0 \)
\end{defn}

\begin{defn}{Тензорный ранг}{}
    Пусть \( A \) --- матрица \( m \times n \).

    \( rk_T A \: = \{ k \ \vline \ A = x_1 y_1 + \ldots + x_n y_n, \: x_i \in F^m, y_i^T \in F^n \} \)

    Для нулевой матрицы: \( rk_T \: 0 = 0 \)
\end{defn}

\begin{defn}{Минорный ранг}{}
    Квадратная подматрица --- элементы на пересечении \( k \) столбцов и \( k \) строк.

    \( rk_M \: A = \) размер максимальной (именно максимальной, а не наибольшей) невырожденной квадратной подматрицы.
\end{defn}

\begin{remark}{}{}
    В силу блочного умножения \( rk_\text{Ф} = rk_T \).
\end{remark}

\newpage

\begin{lemma}{}{}
    \( A \in M_{mn} (F) \)

    \( C \in M_m(F) \) --- невырожденная

    \( D \in M_n(F) \) --- невырожденная

    Покажем, что \( rk_* \: A = rk_* \: (CAD) \) для \( * = \) столб, строк, Ф, Т
\end{lemma}

Поскольку \( rk_\text{столб} \: A = rk_\text{строч} \: A^T \), можем доказывать только для столбцового.
\begin{itemize}
    \item
        \( A \mapsto CA = B \)

        \(
            A = \begin{pmatrix}
                A_1, \ldots, A_n
            \end{pmatrix}
            \quad
            B = \begin{pmatrix}
                B_1, \ldots, B_n
            \end{pmatrix}
        \)

        Столбцы --- координаты в базисе, умножение на \( C \) --- смена координат.

        Поэтому
        \( \dim < A_1, \ldots, A_n > = \dim < B_1, \ldots, B_n > \)
    \item
        \( A \mapsto AD = B \)

        \(
            A = \begin{pmatrix}
                A_1, \ldots, A_n
            \end{pmatrix}
            \quad
            B = \begin{pmatrix}
                B_1, \ldots, B_n
            \end{pmatrix}
        \)

        Заметим, что \( B_i \) --- линейная комбинация \( A_1, \ldots, A_n \).

        Поэтому \( < B_1, \ldots, B_n > \subseteq < A_1, \ldots, A_n > \).

        С другой стороны \( A = D^{-1} B \Rightarrow \) включение верно в другую сторону,
        а значит линейные оболочки равны, откуда и следует нужное.
\end{itemize}

Теперь докажем для факториального.

Пусть \( A = B_1 \cdot B_2 \) и на этом достигается ранг.

Но тогда \( B = CAD = (C B_1) \cdot (B_2 D) \), а значит \( rk_\text{Ф} \: A \geq rk_\text{Ф} \: B \).

Но \( A = C^{-1} B D^{-1} \), а значит равенство верно и в другую сторону.
Значит \( rk_\text{Ф} \: A = rk_\text{Ф} \: CAD \)

\begin{lemma}{}{}
    \( A \in M_{mn} (F) \)

    Тогда \( rk_\text{столб} \: A = rk_\text{строк} \: A = rk_\text{Ф} \: A = rk_T \: A \)
\end{lemma}

Давайте элементарными преобразованиями приведем \( A \) к виду
\(
    \begin{pmatrix}
        E & 0
        \\
        0 & 0
    \end{pmatrix}
\)

В силу предыдущей леммы можно доказывать равенство рангов на этой матрице.
Пусть размеры \( E \) --- \( r \times r \).

Очевидно, что \( rk_\text{столб} \: A = rk_\text{строч} \: A = r \)

\(
    \begin{pmatrix}
        E & 0
        \\
        0 & 0
    \end{pmatrix}
    =
    \begin{pmatrix}
        E
        \\
        0
    \end{pmatrix}
    \begin{pmatrix}
        E & 0
    \end{pmatrix}
\)

Причем размеры множителей \( n \times r \) и \( r \times m \)

Значит \( rk_\text{Ф} \leq rk_\text{столб} \)

Теперь покажем, что \( rk_\text{Ф} \geq rk_\text{столб} \).


Рассмотрим разбиения на множители, где достигается факториальный ранг
\( A = BC \), и размер \( B \) --- \( m \times k \), а \( C \) --- \( k \times n \)

\(
    \begin{pmatrix}
        A_1 & \ldots & A_n
    \end{pmatrix}
    =
    \begin{pmatrix}
        B_1
        \\
        \vdots
        \\
        B_n
    \end{pmatrix}
    C
\)

\( A_i \) --- линейная комбинация \( B_1, \ldots, B_k \)
\( \Rightarrow \) \( < A_1, \ldots, A_n > \subseteq < B_1, \ldots, B_k > \).

Но тогда \( rk_\text{столб} \dim < A_1, \ldots, A_n > \leq k = rk_\text{Ф} \).

Выше показали, что неравенство верно и в другую сторону, поэтому \( rk_\text{Ф} = rk_\text{столб} \)

\begin{lemma}{}{}
    Пусть \( B \) максимальная по включению невырожденная квадратная подматрица размера \( r \).

    Тогда \( r = rk_\text{столб} \: A = rk_\text{строч} \: A = rk_\text{Ф} \: A = rk_T \: A \)
\end{lemma}

Для начала приведем приведем матрицу \( A \) к виду
\(
    \begin{pmatrix}
        B & *
        \\
        * & *
    \end{pmatrix}
\)

Заметим, что элементарные преобразования над первыми \( r \) строками
оставляют \( B \) максимальной невырожденной подматрицей:
\begin{itemize}
    \item
        Невырожденность \( B \) не меняется (так как элементарные преобразования сохраняют ее)
    \item
        Если при расширении матрицы мы нашли большую невырожденную,
        то обратными элементарными преобразованиями получим, что \( B \) была не максимальной.
\end{itemize}

Теперь приведем матрицу \( B \) к виду
\(
    \begin{pmatrix}
        E & *
        \\
        * & *
    \end{pmatrix}
\)

По аналогичным причинам мы можем прибавлять первые \( r \) строк к любым, сохраняя условие леммы.
Значит мы можем занулить в первом столбце все кроме первых \( r \) строк.
Точно так же мы можем занулить и все в первых \( r \) строках, кроме первых \( r \) столбцов.

Привели \( B \) к виду
\(
    \begin{pmatrix}
        E & 0
        \\
        0 & *
    \end{pmatrix}
\)

Теперь будем доказывать для такой матрицы.
Предположим, что в \( * \) есть ненулевой элемент.
Но тогда мы можем расширить \( E \) до другой невырожденной квадратной подматрицы --- противоречие.

Но тогда на самом деле у \( B \) такой вид:
\(
    \begin{pmatrix}
        E & 0
        \\
        0 & 0
    \end{pmatrix}
\)

Но это значит, что \( r \) совпал со всеми ранними рангами.

\subsection{Линейные отображения}

\subsubsection{Определение}

\begin{defn}{Линейное отображение}{}
    \( V, U \) --- векторные пространства над \( F \)

    \( \varphi : V \to U \)

    \begin{enumerate}
        \item
            \( \varphi (v_1 + v_2) = \varphi (v_1) + \varphi (v_2) \)
        \item
            \( \varphi ( \lambda v ) = \lambda \varphi (v) \)
    \end{enumerate}

    В случае выполнения этих свойств \( \varphi \) --- линейное отображение.

    Если верно еще и
    \begin{enumerate}[start=3]
        \item
            \( \varphi \) --- биективное
    \end{enumerate}
    тогда \( \varphi \) --- изоморфизм.
\end{defn}

\begin{example}{}{}
    \begin{itemize}
        \item
            \( \varphi : F^n \to F^m \)

            \( \varphi(x) = Ax \)
        \item
            \( \varphi = 0 \)
        \item
            \( id(x) = x \)
        \item
            \( D[0, 1] = \{ f[0, 1] \to \RR \ \vline \ \exists f' \} \)

            \( F[0, 1] = \{ f[0, 1] \to \RR \} \)

            \( \varphi : D[0, 1] \to F[0, 1] \)

            \( \varphi(f) = f' \)
        \item
            \( \varphi : C[0, 1] \to D[0, 1] \)

            \( \varphi(f) = \hat{f} (x) = \int\limits_0^x f(t) dt \)
        \item
            \( \varphi : M_n (F) \to M_n (F) \)

            \( \varphi(X) = X^T \)
    \end{itemize}
\end{example}

\( \hom_F (v, u) = \{ \varphi \ \vline \ \varphi - \text{\( F \)-линейно} \} \)

\begin{remark}{}{}
    \( (\hom_F (v, u), +, \cdot) \) --- векторное пространство.
\end{remark}

Очевидно, что для задания линейного отображения нужно задать его на базисных векторах,
причем если у двух отображения совпадают значения на базисе,
то они совпадают везде.

\subsubsection{Матрица линейного отображения}

Сделаем аналог координат.

Идея:

\(
    \Phi_\text{коорд} F^n \to F^m
\)

\(
    \Phi(x) = Ax
\)

\( \Phi : V \to U \)

\( e_1, \ldots, e_n \) --- базис \( V \)

\( f_1, \ldots, f_m \) --- базис \( U \)

\bigskip

Сделаем

\(
    \begin{cases}
        \Phi(e_1) = u_1 \in U
        \\
        \vdots
        \\
        \Phi(e_n) = u_n \in U
    \end{cases}
\)

\bigskip

\(
    \begin{cases}
        u_1 = a_{1 \: 1} f_1 + a_{2 \: 1} f_2 + \ldots + a_{m \: 1} f_m
        \\
        \vdots
        \\
        u_n = a_{1 \: n} f_1 + a_{2 \: n} f_2 + \ldots + a_{m \: n} f_m
    \end{cases}
\)

\bigskip

\(
    A
    =
    \begin{pmatrix}
        a_{1 \: 1} & \vline & \ & \vline & a_{1 \: n}
        \\
        a_{2 \: 1} & \vline & \ & \vline & a_{2 \: n}
        \\
        \vdots & \vline & \ & \vline & \vdots
        \\
        a_{m \: 1} & \vline & \ & \vline & a_{m \: n}
    \end{pmatrix}
\)

\bigskip

Получаем \( \Phi (e_1, \ldots, e_n) = (f_1, \ldots, f_m) A \)
(чтобы нормально определить такое равенство, считаем, что
\( \Phi \) --- матрица \( 1 \times 1 \) с элементом равным линейному отображению)

Остается понять, как с помощью матрицы задать линейное отображение для любого вектора:
\begin{gather*}
    v = (e_1, \ldots, e_n) x \quad x \in F^n
    \\
    \Phi(v) = \Phi( (e_1, \ldots, e_n) x )
    =
    \Phi ( x_1 e_1 + \ldots + x_n e_n ) = x_1 \Phi (e_1) + \ldots + x_n \Phi (e_n)
    =
    \\
    =
    ( \Phi (e_1, \ldots, e_n) ) x = ( f_1, \ldots, f_n ) Ax
\end{gather*}

\subsubsection{Смена координат}

Что такое смена координат?
\begin{gather*}
    \Phi(e_1, \ldots, e_n) = (f_1, \ldots, f_m) A
    \\
    \Phi(e'_1, \ldots, e'_n) = (f'_1, \ldots, f'_m) A'
    \\
    (e'_1, \ldots, e'_n) = (e_1, \ldots, e_n) C
    \\
    (f'_1, \ldots, f'_m) = (f_1, \ldots, f_m) D
    \\
    \Phi(e_1, \ldots, e_n) C = (f_1, \ldots, f_m) D A'
    \\
    \Phi(e_1, \ldots, e_n) = (f_1, \ldots, f_m) D A' C^{-1}
    \\
    A = D A' C^{-1}
    \\
    A' = D^{-1} A C
\end{gather*}

Но сделаем \( D^{-1} \) и \( C \) элементарными преобразованиями над строками и столбцами.
Тогда \( A \) можно привести к виду
\(
    \begin{pmatrix}
        E & 0
        \\
        0 & 0
    \end{pmatrix}
\)

\subsection{Ядро и образ}

\begin{defn}{Ядро}{}
    \( \ker \Phi = \{ v \in V \ \vline \ \Phi(v) = 0 \} \)
\end{defn}

\begin{defn}{Образ}{}
    \( Im \: \Phi = \{ \Phi(v) \in U \ \vline \ v \in V \} = \Phi(V) \)
\end{defn}

\begin{remark}{}{}
    \( \ker \Phi \subseteq V \) --- подпространство

    \( Im \: \Phi \subseteq U \) --- подпространство
\end{remark}

\begin{example}{}{}
    \( \Phi : F^n \to F^m \)

    \( \Phi(x) = Ax \)

    \( A \in M_{mn} (F) \)

    \( \ker \Phi = \{ x \in F^n \ \vline \ Ax = 0 \} \)

    \(
        Im \: \Phi = \{ Ax \in F^m \ \vline \ x_i \in F \} = < A_1, \ldots, A_n >
    \)

    (\( A_i \) --- столбцы \( A \))
\end{example}

\begin{remark}{}{}
    \( \dim Im \: \Phi = rk \: A \)

    \( \dim \ker \Phi = n - rk \: A \)
\end{remark}

\begin{lemma}{}{}
    \( \Phi : V \to U \) --- линейное отображение

    Тогда
    \begin{enumerate}
        \item
            \( \Phi \) инъективно \( \Leftrightarrow \) \( \ker \Phi = 0 \)
        \item
            \( \Phi \) сюръективно \( \Leftrightarrow \) \( Im \: \Phi = U \)
        \item
            \( \dim \ker \Phi + \dim Im \: \Phi = \dim V \ \)
    \end{enumerate}
\end{lemma}

Второе очевидно, третье следует из замечания выше.

Докажем первое:
\begin{itemize}
    \item
        \( \Rightarrow \)

        Очевидно
    \item
        \( \Leftarrow \)

        \(
            \Phi(v_1) = \Phi(v_2)
            \Rightarrow
            \Phi(v_1) - \Phi(v_2) = 0
            \Rightarrow
            \Phi(v_1 - v_2) = 0
            \Rightarrow
            v_1 - v_2 = 0
            \Rightarrow
            v_1 = v_2
        \)
\end{itemize}

\subsection{Неравенства на ранги}

\subsubsection{Сумма}

\begin{lemma}{}{}
    \( A, B \in M_{mn} (F) \)

    Тогда \( | rk \: A - rk \: B | \leq rk \: (A + B) \leq rk \: A + rk \: B \)
\end{lemma}

Пусть \( rk \: A = r, rk \: B = d \).

Заметим, что из тензорного ранга сразу следует, что \( A + B \) можно
представить в нужной форме из \( r + d \) слагаемых,
поэтому \( rk \: (A + B) \leq r + d = rk \: A + rk \: B \)

Для другого неравенства воспользуемся трюком:
\begin{gather*}
    rk \: ((A + B) - (-B)) \leq rk \: (A + B) + rk \: (-B)
    \\
    rk \: A \leq rk \: (A + B) + rk \: B
    \\
    rk \: A - rk \: B \leq rk \: (A + B)
\end{gather*}

Сделав такое же неравенство на \( rk \: B - rk \: A \) получим желаемое.

\subsubsection{Произведение}

\begin{lemma}{}{}
    \( A \in M_{mk} (F) \), \( B \in M_{kn} (F) \)

    Тогда \( rk \: A + rk \: B - k \leq rk \: (AB) \leq \min (rk \: A, rk \: B) \)
\end{lemma}

\( rk \: A = r \)

Тогда \( A = A_1 \cdot A_2 \) и \( A_1 \in M_{mr} (F) \)

Значит \( AB = A_1 \cdot (A_2 B) \Rightarrow rk \: AB \leq r = rk \: A \).

Аналогично можно получить неравенство с \( rk \: B \)

Для оценки снизу рассмотрим линейные отображения:
\begin{enumerate}
    \item
        \( B: F^n \to F^k \)

        \( x \mapsto Bx \)
    \item
        \( A: F^k \to F^m \)

        \( x \mapsto Ax \)
    \item
        \( AB: F^n \to F^m \)

        \( x \mapsto ABx \)
\end{enumerate}

\( Im \: A \) = <столбцы \( A \)>

\( \ker A = \{ x \in F^k \ \vline \ x \in F^k \} \)

\( Im \: A = \{ Ax \in F^m \ \vline \ x \in F^k \} \)

Поэтому \( rk \: A = \dim Im \: A \)

Хотим доказать:
\[
    \dim Im \: A + \dim Im \: B - k \leq \dim Im \: AB
\]

Рассмотрим \( A \: \vline_{\: Im \: B} \).
Оно переводит вектор из \( Im \: B \) в \( F^m \),
причем:
\[
    Im \: A \: \vline_{\: Im \: B} = Im \: AB
\]

Из суммы размерностей ядра и образа:
\[
    \dim Im \: A \: \vline_{\: Im \: B} + \dim \ker A \: \vline_{\: Im \: B} = \dim Im \: B
\]

Из соображений здравого смысла:
\begin{gather*}
    \ker A \: \vline_{\: Im \: B} = Im \: B \cap \ker A \subseteq \ker A
    \\
    \Downarrow
    \\
    \dim \ker A \: \vline_{\: Im \: B} \leq \dim \ker A
\end{gather*}

Тогда:
\begin{gather*}
    \dim Im \: A \: \vline_{\: Im \: B} + \dim \ker A \: \vline_{\: Im \: B} = \dim Im \: B
    \\
    \dim Im \: AB + \dim \ker A \: \vline_{\: Im \: B} = \dim Im \: B
    \\
    \dim Im \: AB
        = \dim Im \: B - \dim \ker A \: \vline_{\: Im \: B}
        \leq \dim Im \: B - \dim \ker A
        = \\
        = \dim Im \: B - (k - \dim Im \: A)
        = \dim Im \: A + \dim Im \: B - k
\end{gather*}

Что и требовалось доказать.

\subsection{Суммы и пересечения}

\( V \) --- векторное пространство над \( F \)

\( U, W \subseteq V \) --- подпространства.

Очевидно, что \( U \cap W \) --- тоже подпространство.

К сожалению про \( U \cup W \) такого мы сказать не можем,
поэтому определим сумму:
\( U + W \) --- наименьшее векторное подпространство \( V \),
содержащие и \( U \), и \( W \). Иными словами:
\[
    U + W = \{ u + w \ \vline \ u \in U, w \in W \}
\]

Отсюда следует, что такие штуки тоже векторные подпространства:
\begin{gather*}
    \bigcap\limits_{ \alpha \in A } U_\alpha
    \\
    \sum\limits_{ \alpha \in A } U_\alpha = \{ u_{\alpha_1} + \ldots + u_{\alpha_k} \ \vline \ \alpha_1, \ldots, \alpha_k \in A \}
\end{gather*}

\begin{lemma}{}{}
    \[
        \dim ( U + W ) + \dim ( U \cap W ) = \dim U + \dim W
    \]
\end{lemma}

Рассмотрим \( e_1, \ldots, e_k \in U \cap W \) --- базис.
Его можно дополнить до базиса \( U \) векторами \( f_1, \ldots, f_s \)

Аналогично можно дополнить до базиса \( W \) векторами \( g_1, \ldots, g_t \)

Если мы докажем, что \( e_1, \ldots, e_k, f_1, \ldots, f_s, g_1, \ldots, g_t \) --- базис \( U + W \),
то докажем и равенство.
В силу определения \( U + W \),
такой набор векторов является порождающим.
Значит нужно показать линейную независимость:
\begin{gather*}
    \sum \alpha_i e_i + \sum \beta_i f_i + \sum \gamma_i g_i = 0
    \\
    \sum \alpha_i e_i + \sum \beta_i f_i = - \sum \gamma_i g_i = z \in U \cap W
\end{gather*}

Значит \( z = \sum \delta_i e_i \)
\begin{gather*}
     - \sum \gamma_i g_i = \sum \delta_i e_i
     \\
     \sum \delta_i e_i + \sum \gamma_i g_i = 0
\end{gather*}

Но это базис \( W \), поэтому \( \gamma_i = 0 \), вернемся к изначальному равенству:
\[
    \sum \alpha_i e_i + \sum \beta_i f_i = 0
\]

Но это базис \( U \),
поэтому \( e_i = 0 \) и \( f_i = 0 \).
Что и требовалось доказать.

\begin{defn}{}{}
    \( V \) --- векторное пространство над \( F \).

    \( U_1, \ldots, U_k \subseteq V \) --- подпространства.

    \( U_1, \ldots, U_k \) --- ЛНЗ,
    если \( \forall u_1 \in U_1, \ldots, u_k \in U_k \) выполнено:
    \[
        u_1 + \ldots + u_k = 0 \Leftrightarrow u_1 = \ldots = u_k = 0
    \]
\end{defn}

\begin{example}{}{}
    Если в \( \RR^3 \) взять базис \( e_1, e_2, e_3 \),
    то \( <e_1, e_2> \) и \( <e_3> \) --- ЛНЗ.
\end{example}

\begin{thrm}{}{}
    \( V \) --- векторное пространство над \( F \)

    \( U_1, \ldots, U_k \subseteq V \)

    Тогда эквивалентны:
    \begin{enumerate}
        \item
            \( U_1, \ldots, U_k \) --- ЛНЗ
        \item
            \( \forall v \in U_1 + \ldots + U_k \ \exists! \: v = u_1 + \ldots + u_k \)
        \item
            \( U_i = < e_i > \) (базис) \( \Rightarrow \)
            \( e_i \cap e_j = \varnothing \)
            и \( \bigcup e_i \) --- базис \( U_1 + \ldots + U_k \)
        \item
            \( \sum\limits_{i = 1}^{k} \dim U_i = \dim (U_1 + \ldots + U_k) \)
        \item
            \( \forall i : U_i \cap \sum\limits_{j \neq i} U_j = 0 \)
        \item
            \( U_1 \times \ldots \times U_k \to U_1 + \ldots + U_k \)

            \( (u_1, \ldots, u_k) \mapsto u_1 + \ldots + u_k \) --- изоморфизм

            Иными словами \( U_1 \oplus \ldots \oplus U_k = U_1 + \ldots + U_k \)
    \end{enumerate}
\end{thrm}

\begin{itemize}
    \item
        \( (1) \Rightarrow (2) \)

        Очевидно
    \item
        \( (2) \Rightarrow (3) \)

        Если бы в каких-то базисах был общий элемент,
        то \( 0 \) можно было бы получить более чем одним способом --- противоречие.
        Значит осталось доказать то, что объединение базисов --- базис суммы.
        Понятно, что он порождающий, значит нужно показать что он ЛНЗ.
        Рассмотрим \( e_1 \alpha_1 + \ldots + e_k \alpha_k = 0 \).
        Так как \( e_i \alpha_i \) лежит в \( U_i \),
        каждое такое слагаемое должно быть нулем из \( (2) \).
        Но также \( e_i \) --- базис, поэтому \( \alpha_i = 0 \)
    \item
        \( (3) \Rightarrow (4) \)

        Очевидно
    \item
        \( (4) \Rightarrow (5) \)

        % TODO
    \item
        \( (5) \Rightarrow (1) \)

        \( u_1 + \ldots + u_k = 0 \)

        \( u_2 + \ldots + u_k = -u_1 \in U_1 \)

        \( u_1 \in U_1 \cap (U_2 + \ldots + U_k) = 0 \Rightarrow u_1 = 0 \)

        Аналогично для всех остальных получим \( u_i = 0 \)
    \item
        \( (6) \)

        Такое отображение является сюръективным и линейным по определению.
        Значит остается доказать инъективность.
        Но мы знаем, что это равносильно тому, что ядро равно нулю.

        Но это просто переформулировка первого пункта.
\end{itemize}

