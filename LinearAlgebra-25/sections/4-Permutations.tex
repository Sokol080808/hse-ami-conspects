\section{Перестановки}

\subsection{Что это вообще такое}

\begin{defn}{}{}
    \begin{gather*}
        \sigma : \{ 1, \ldots, n \} \xrightarrow{\sim} \{ 1, \ldots, n \}
        \\
        S_n = \{ \sigma - \text{перестановка на } \{ 1, \ldots, n \} \}
    \end{gather*}
\end{defn}

\subsubsection{Способы задания}

\begin{itemize}
    \item
        Явно задать отображения
    \item
        Нарисовать табличкой

        \(
            \begin{pmatrix}
                1 & 2 & 3 \\
                3 & 1 & 2
            \end{pmatrix}
        \)
    \item
        Нарисовать стрелки между точками на плоскости
\end{itemize}

\subsubsection{Операция}

Перестановки можно перемножать.

\begin{gather*}
    \\
\end{gather*}


\subsubsection{Свойства}

\begin{enumerate}
    \item
        Ассоциативность

        \(
        \begin{aligned}
            & \sigma, \tau, p \in S_n
            \\
            & \sigma \circ ( \tau \circ p ) = ( \sigma \circ \tau ) \circ p
        \end{aligned}
        \)
    \item
        \( id : \{ 1, \ldots, n \} \xrightarrow{\sim} \{ 1, \ldots, n \} \)
        \begin{gather*}
            k \mapsto k
            \\
            \sigma \circ id = id \circ \sigma = \sigma
        \end{gather*}
    \item
        \(
            \sigma \circ \sigma^{-1} = id = \sigma^{-1} \circ \sigma
        \)
\end{enumerate}

\subsubsection{Переименование}

Переименуем \( i \mapsto \tau_i \)

\( \sigma_{\text{нов}} = \tau \sigma \tau^{-1} \)

\subsubsection{Циклы}

\begin{defn}{Цикл}{}
    \begin{gather*}
        \{ 1, \ldots, n \} = \{ i_1, \ldots, i_k \} \sqcup \{ j_1, \ldots, j_{n - k} \}, \ |i| \geq 2
        \\
        \sigma
        =
        \begin{pmatrix}
            i_1 & i_2 & i_3 & \cdots & i_{k - 1} & i_k & j_1 & j_2 & \cdots j_{n - k} \\
            i_2 & i_3 & i_4 & \cdots & i_k & i_1 & j_1 & j_2 & \cdots j_{n - k}
        \end{pmatrix}
    \end{gather*}
\end{defn}

Краткая запись: \( \sigma = ( i_1, i_2, \ldots, i_k ) = ( i_2, i_3, \ldots, i_k, i_1 ) \)

\begin{defn}{Транспозиция}{}
    \( \sigma = ( i, j ) \) --- транспозиция
\end{defn}

\begin{defn}{Независимые циклы}{}
    Циклы \( \{ i_1, \ldots, i_k \}, \{ j_1, \ldots, j_l \} \) являются независимыми,
    если \( \{ i_1, \ldots, i_k \} \cap \{ j_1, \ldots, j_l \} = \varnothing \)
\end{defn}

\begin{remark}{}{}
    \( p_1, p_2 \) --- независимые циклы \( \Rightarrow p_1 p_2 = p_2 p_1 \)
\end{remark}

По сути композиция двух независимых циклов просто делает перестановку, 
в которой у нас два ``цикла'', если рисовать стрелочками.

\begin{lemma}{}{}
    \begin{enumerate}
        \item
            \( \sigma = p_1 \cdot \ldots \cdot p_k \) --- независимые циклы,
            причем разложение единственно с точностью до порядка
        \item
            \( p \) --- цикл \( p = \tau_1 \cdot \ldots \cdot \tau_{k - 1} \),
            где \( \tau_i \) --- транспозиция.
    \end{enumerate}
\end{lemma}

\subsection{Знак перестановки}

Хотим чтобы было так:

\(
    \begin{cases}
        \text{Ч \( \cdot \) Ч \( = \) Ч}
        \\
        \text{Ч \( \cdot \) Н \( = \) Н}
        \\
        \text{Н \( \cdot \) Ч \( = \) Н}
        \\
        \text{Н \( \cdot \) Н \( = \) Ч}
    \end{cases}
\)

\(
    \begin{aligned}
        & \phi : S_n \to \{ \pm 1 \}
        \\
        & \begin{cases}
            \phi ( \sigma \tau ) = \phi ( \sigma ) \phi ( \tau )
            \\
            \phi \not\equiv 1
        \end{cases}
    \end{aligned}
\)

\begin{thrm}{}{}
    \(
        \begin{aligned}
            & \exists ! \phi : S_n \to \{ \pm 1 \} :
            \\
            & \begin{cases}
                \phi ( \sigma \tau ) = \phi ( \sigma ) \phi ( \tau )
                \\
                \phi \not\equiv 1
            \end{cases}
        \end{aligned}
    \)
\end{thrm}

Доказательство:

Сначала покажем существование:

\( d_{i j} ( \sigma ) \) --- правда ли, что \( (i, j) \) --- инверсия.

Пусть \( d ( \sigma ) \) --- количество инверсий (\( i < j : \sigma(i) > \sigma(j) \)) \( \sigma \).

Возьмем \( \phi = (-1)^(d(\sigma)) \)

Нужно показать, что \( d(\sigma \tau) = d(\sigma) + d(\tau) \mod 2 \).

Нетрудно убедиться, что \( d_{i j} ( \tau ) + d_{ \tau(i) \tau(j) } = d_{i j} ( \sigma \tau )\)

\begin{thrm}{Единственность знака}{}
    \( 
        \begin{aligned}
            & \phi : S_n \to \{ \pm 1 \}
            \\
            & \begin{cases}
                \phi ( \sigma \tau ) = \phi ( \sigma ) \phi ( \tau )
                \\
                \phi \not\equiv 1
            \end{cases}
        \end{aligned}
    \)
    Тогда

    \begin{enumerate}
        \item
            \( \phi(id) = 1 \)
        \item
            \( \phi(\sigma^{-1}) = \phi(\sigma)^{-1} = \phi(\sigma) \)
        \item
            \( \phi(\tau \sigma \tau^{-1}) = \phi(\sigma) \)
        \item
            \( \phi( (i, j) ) = -1, \forall i, j \)
        \item
            \( \phi ! \)
    \end{enumerate}
\end{thrm}

\newpage

Доказательство:
\begin{enumerate}
    \item \ \\
        \(
        \begin{aligned}
            & id \circ id = id
            \\
            & \phi(id \circ id) = \phi(id)
            \\
            & \phi(id)^2 = \phi(id)
            \\
            & \phi(id) = 1
        \end{aligned}
        \)
    \item \ \\
        \(
            \begin{aligned}
                & \sigma \circ \sigma^{-1} = id
                \\
                & \phi(\sigma) \phi (\sigma^{-1}) = \phi(\sigma \sigma^{-1}) = \phi(id) = 1
                \\
                & \phi(\sigma^{-1}) = \phi(\sigma)^{-1} = \phi(\sigma)
            \end{aligned}
        \)
    \item
        \(
            \phi (\tau \sigma \tau^{-1}) 
            = 
            \phi(\tau) \phi(\sigma) \phi(\tau^{-1})
            = 
            \phi(\tau) \phi(\sigma) \phi(\tau)^{-1}
            =
            \phi (\sigma)
        \)
    \item
        \(
            (i, j) = \tau (1, 2) \tau^{-1} = ( \tau(1), \tau(2) )
        \)

        Значит все транспозиции имеют один и тот же знак.

        Перестановка раскладывается в произведение циклов, а цикл в произведение транспозиций.
        Значит перестановка --- это произведение транспозиций. 
        Если знак всех транспозиций равен \( 1 \), то знак всех перестановок равен \( 1 \) --- противоречие.
        Значит знак всех перестановок равен \( -1 \).

    \item
        Пусть есть \( \phi, \phi' \).

        \begin{gather*} 
            \phi(\sigma) 
            = 
            \phi(\tau_1 \cdot \tau_2 \cdot \ldots \cdot \tau_k)
            =
            \phi(\tau_1) \phi(\tau_2) \ldots \phi(\tau_k)
            =
            \\
            =
            (-1)^k
            =
            \\
            =
            \phi'(\tau_1) \phi'(\tau_2) \ldots \phi'(\tau_k)
            =
            \phi'(\tau_1 \cdot \tau_2 \cdot \ldots \cdot \tau_k)
            =
            \phi'(\sigma)
        \end{gather*}
\end{enumerate}
