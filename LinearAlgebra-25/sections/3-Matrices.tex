\section{Матрицы}
Матрица - это таблица с числами:
\[
A = 
\begin{pmatrix}
    a_{1 1} & a_{1 2} & \dots & a_{1 n} \\
    \vdots & \vdots & \vdots & \vdots \\
    a_{m 1} & a_{m 2} & \dots & a_{m n} \\
\end{pmatrix}
\]

В данном случае \( A \in M_{mn}(\mathbb{R}) \) 
--- множество матриц \(m \times n \) с числами из \( \mathbb{R} \)

\( M_n(\mathbb{R}) \) --- множество квадратных матриц \( n \times n \) 
с числами из \( \mathbb{R} \)

\bigskip

Операции, которые можно делать с матрицами:
\begin{itemize}
    \item Сложение / вычитание:
    \[
    \begin{pmatrix}
    a_{1 1} & \dots & a_{1 n}  \\
    \vdots & \ddots & \vdots \\
    a_{m 1} & \dots & a_{m n}  \\
    \end{pmatrix}
    \pm
    \begin{pmatrix}
    b_{1 1} & \dots & b_{1 n}  \\
    \vdots & \ddots & \vdots \\
    b_{m 1} & \dots & b_{m n}  \\
    \end{pmatrix}
    =
    \begin{pmatrix}
    a_{1 1} \pm b_{1 1} & \dots & a_{1 n} \pm b_{1 n} \\
    \vdots & \ddots & \vdots & \\
    a_{m 1} \pm b_{m 1} & \dots & a_{m n} \pm b_{m n} \\
    \end{pmatrix}
    \]
    \item Умножение на число
    \[
    \lambda \cdot
    \begin{pmatrix}
    a_{1 1} & \dots & a_{1 n}  \\
    \vdots & \ddots & \vdots \\
    a_{m 1} & \dots & a_{m n}  \\
    \end{pmatrix}
    =
    \begin{pmatrix}
    \lambda \cdot a_{1 1} & \dots & \lambda \cdot a_{1 n}  \\
    \vdots & \ddots & \vdots \\
    \lambda \cdot a_{m 1} & \dots & \lambda \cdot a_{m n}  \\
    \end{pmatrix}
    \]
    \item Транспонирование
    \[
    \begin{pmatrix}
    a_{1 1} & a_{1 2} & \dots & a_{1 n}  \\
    \vdots & \vdots & \vdots & \vdots \\
    a_{m 1} & a_{m 2} & \dots & a_{m n}  \\
    \end{pmatrix}^T
    =
    \begin{pmatrix}
    a_{1 1} & \dots & a_{m 1}  \\
    a_{1 2} & \dots & a_{m 2} \\
    \vdots & \vdots & \vdots \\
    a_{1 n} & \dots & a_{m n}  \\
    \end{pmatrix}
    \]

    Иными словами, отражаем относительно главной диагонали:
    \[
    \begin{pmatrix}
    1 & 2 & 3 \\
    4 & 5 & 6 
    \end{pmatrix}^T
    =
    \begin{pmatrix}
    1 & 4 \\
    2 & 5 \\
    3 & 6
    \end{pmatrix}
    \]
    \item Умножение на матрицу
    Пусть \(
    A \in M_{nm}(\mathbb{R}), B \in M_{mk}(\mathbb{R})
    \).

    Тогда \( AB \in M_{nk}(\mathbb{R}) \), причем 
    \( AB_{i j} = \sum\limits_{r=1}^{m} A_{i r}B_{r j} \)
\end{itemize}

Некоторые матрицы:
\begin{itemize}
    \item Нулевая матрица
    
    \( A \) -- нулевая матрица, если \( \forall i, j \: A_{i j} = 0 \)

    Если известны размеры, то можно просто написать \( A = 0 \)

    \item Единичная матрица
    
    Матрица, в которой на диагонали стоят \( 1 \), а остальные числа равны \( 0 \)

    \( E_n = 
    \begin{pmatrix}
    1 & 0 & \dots & 0 \\
    0 & 1 & \dots & 0 \\
    \vdots & \vdots & \ddots & \vdots \\
    0 & 0 & \dots & 1  \\
    \end{pmatrix}
    \) --- единичная матрица \( n \times n \)

    Называется так, потому что \( AE = EA = A \)
\end{itemize}

Тогда с точки зрения матриц, СЛУ можно записать так:
\[
A =
\begin{pmatrix}
    a_{1 1} & a_{1 2} & \dots & a_{1 n} & \vline & b_1 \\
    \vdots & \vdots & \vdots & \vdots & \vline & \vdots \\
    a_{m 1} & a_{m 2} & \dots & a_{m n} & \vline & b_m \\
\end{pmatrix} \:
x = 
\begin{pmatrix}
    x_1 \\
    x_2 \\
    \vdots \\
    x_n
\end{pmatrix} \:
b = 
\begin{pmatrix}
    y_1 \\
    y_2 \\
    \vdots \\
    y_m
\end{pmatrix}
\]
\[ Ax = b \]

Если СЛУ однородная, то \( Ax = 0 \)

\bigskip

Полезные свойства операций над матрицами:
\begin{itemize}
    \item Ассоциативность
    
    \( (AB)C = A(BC) \)
    \item Дистрибутивность относительно сложения
    
    \( (A + B)C = AC + BC \)
\end{itemize}


Связь СЛУ и ОСЛУ:

\( A \in M_{mn}(\RR), b \in \RR^n, x_0 \in \RR^n : Ax_0 = b \)

Тогда:

\( 
    E_b = \{ z \in \RR^n \ \vline \ Az = b \}, 
    E_0 = \{ y \in \RR^n \ \vline \ Ay = 0 \}
    \Rightarrow
    E_b = x_0 + E_0
\)

Если в \( A \) нулевая строка, то в \( AB \) та же строка будет нулевой.
Аналогично, если в \( B \) есть нулевой столбец.

\begin{remark}{}{}
    \[
        (AB)^T = B^T A^T
    \]
\end{remark}

\newpage

\subsection{Дефекты матричных операций}

\begin{enumerate}
    \item {
        Произведение не коммутативно.

        Если матрицы не квадратные, то произведение \( BA \) либо не существует,
        либо имеет отличные от \( AB \) размеры.

        В случае квадратных, например:

        \begin{gather*}
            \begin{pmatrix}
                0 & 1 \\ 
                0 & 0 
            \end{pmatrix}
            \cdot
            \begin{pmatrix}
                0 & 0 \\ 
                1 & 0 
            \end{pmatrix}
            =
            \begin{pmatrix}
                1 & 0 \\ 
                0 & 0 
            \end{pmatrix} 
            \\
            \begin{pmatrix}
                0 & 0 \\ 
                1 & 0 
            \end{pmatrix}
            \cdot
            \begin{pmatrix}
                0 & 1 \\ 
                0 & 0 
            \end{pmatrix} 
            =
            \begin{pmatrix}
                0 & 0 \\ 
                0 & 1 
            \end{pmatrix}
        \end{gather*}
    }

    \item {
        Есть делители нуля.

        \begin{gather*}
            \begin{pmatrix}
                1 & 0 \\ 
                0 & 0 
            \end{pmatrix}
            \cdot 
            \begin{pmatrix}
                0 & 0 \\ 
                0 & 1
            \end{pmatrix} 
            =
            \begin{pmatrix}
                0 & 0 \\ 
                0 & 0
            \end{pmatrix}
        \end{gather*}

        Диагональные матрицы ведут себя как функции на конечном множестве,
        поэтому если что-то можно получить с такими функциями,
        то и с матрицами скорее всего тоже.

        Благодаря этому у ОСЛУ могут быть ненулевые решения.
    }

    \item {
        Нильпотенты.
        
        \( A \in M_{mn}(\RR) \) --- нильпотентна, если \( \exists N : A^N = 0 \)

        Например: 
        \[ 
            \begin{pmatrix}
                0 & 1 \\ 
                0 & 0
            \end{pmatrix}^2
            =
            \begin{pmatrix}
                0 & 0 \\ 
                0 & 0
            \end{pmatrix}
        \]
    }
\end{enumerate}

\subsection{Деление}

В силу некоммутативности нам нужно деление слева и деление справа.
Но тогда придется доказывать совместимость со всеми остальными операциями.
Поэтому вместо деления принято говорить про обратимые матрицы.

\begin{defn}{}{}
    \[ A \in M_{mn}(\RR), B \in M_{nm}(\RR) \] 
    \begin{enumerate}
        \item {
            Правая обратная, если \( AB = E_m \)
        }
        \item {
            Левая обратная, если \( BA = E_n \)
        }
        \item {
            Обратная, если верны (1) и (2)
        }
    \end{enumerate}
\end{defn}

\begin{defn}{След матрицы}{trace}
    \[ A \in M_n(\RR) \]

    \( \tr A = \sum\limits_{i = 1}^n a_{i i} \)
\end{defn}

\begin{enumerate}
    \item {
        \( \tr (\lambda A) = \lambda \tr (A) \)
    }
    \item {
        \( \tr (A + B) = \tr (A) + \tr (B) \)
    }
    \item {
        \( \tr (AB) = \tr (BA) \)

        \begin{gather*}
            \tr (AB) = \sum\limits_{i = 1}^m (AB)_{i i} = 
            \sum\limits_{i = 1}^m \sum\limits_{j = 1}^n A_{i j} B_{j i}
            \\
            \tr (BA) = \sum\limits_{i = 1}^n (BA)_{i i} = 
            \sum\limits_{i = 1}^n \sum\limits_{j = 1}^m B_{i j} A_{j i}           
        \end{gather*}
    }
\end{enumerate}


\begin{lemma}{}{}
    \begin{gather*}
        A \in M_{mn}(\RR) \\
        \exists L - \text{любой левый обратный} \\
        \exists R - \text{любой правый обратный}
    \end{gather*}

    Тогда:
    \begin{enumerate}
        \item \( n = m \)
        \item \( L = R \Rightarrow \) существует единственный обратный.
    \end{enumerate}
\end{lemma}

Докажем второе:
\begin{gather*}
    LA = E \\ 
    AR = E \\
    (LA)R = L(AR) \\ 
    ER = LE \\ 
    R = L
\end{gather*}

Теперь первое:

\begin{gather*}
    \tr (AB) = \tr (E_m) = m \\ 
    \tr (BA) = \tr (E_n) = n \\ 
    \tr (AB) = \tr (BA) \Rightarrow m = n
\end{gather*}

\begin{defn}{Обратная матрица}{}
    \[ 
        A \in M_n(\RR), B - \text{обратный}
        \Leftrightarrow
        \begin{cases}
            AB = E \\
            BA = E
        \end{cases}
    \]

    Тогда \( B = A^{-1} \). 
\end{defn}

Примеры:
\begin{enumerate}
    \item {
        Матрица с нулевой строчкой или столбцом необратима.
    }
    \item {
        Диагональная матрица с ненулевыми элементами обратима:

        \[ 
            \begin{pmatrix}
                \lambda_1 & & 0 \\ 
                & \ddots & \\ 
                0 & & \lambda_n
            \end{pmatrix}
            \cdot 
            \begin{pmatrix}
                \frac{1}{\lambda_1} & & 0 \\ 
                & \ddots & \\ 
                0 & & \frac{1}{\lambda_n}
            \end{pmatrix}
            =
            E_n
        \] 
    }
\end{enumerate}

\subsection{Элементарные матрицы}

\begin{enumerate}
    \item {
        Как прибавить строчку \( j \), умноженную на \( \lambda \), к строчке \( i \)?

        Нужно умножить слева на матрицу:

        \( 
            S_{i j}(\lambda) = 
            \begin{cases}
                (S_{i j}(\lambda))_{x y} = \lambda, (x, y) = (i, j) \\
                (S_{i j}(\lambda))_{x y} = (E_n)_{x y}, \text{иначе}
            \end{cases}
        \)

        Если умножить на такую матрицу справа, 
        то к \( j \) столбцу прибавится столбец \( i \), умноженный на \( \lambda \)
    }
    \item {
        Как поменять две строчки местами?

        Умножить слева на матрицу:

        \( 
            U_{i j} = 
            \begin{cases}
                (U_{i j})_{x y} = 1, (x = y \neq i, j) \lor ((x, y) \in \{ (i, j), (j, i)\}) \\ 
                (U_{i j})_{x y} = 0, \text{иначе}
            \end{cases}
        \)

        Если умножить справа, то поменяются местами столбцы.
    }
    \item {
        Как умножить строчку \( i \) на \( \lambda \) ?

        Умножить слева на матрицу:

        \(
            D_{i j}(\lambda) = 
            \begin{cases}
                (D_{i j})_{x y} = \lambda, x = y = i \\ 
                (D_{i j})_{x y} = (E_n)_{x y}, \text{иначе}
            \end{cases}
        \)
        
        Если умножить справа, то умножится \( i \)-й столбец.
    }
\end{enumerate}

Очевидно, все элементарные матрицы обратимы.

\begin{remark}{}{}
    \[
        A, B \in M_n(\RR), \exists A^{-1}, B^{-1} \Rightarrow \exists (AB)^{-1} = B^{-1} A^{-1}
    \]
\end{remark}

\begin{thrm}{}{invert-equiv}
    \[
        A \in M_n(\RR) \\ 
    \]
    Следующие утверждения эквивалентны:
    \begin{enumerate}
        \item {
            \( Ax = 0 \Rightarrow !x \) 
        }
        \item {
            \( A^T y = 0 \Rightarrow !y \)
        }
        \item {
            \( A = U_1 \cdot U_2 \cdot \ldots \cdot U_k \), \( U_i \) --- элементарная
        }
        \item {
            \( \exists A^{-1} \)
        }
        \item {
            \( \exists L \in M_n(\RR) : LA = E \)
        }
        \item {
            \( \exists R \in M_n(\RR) : AR = E \)
        }
    \end{enumerate}
\end{thrm}

Будем доказывать через два цикла:

\(
    1 \rightarrow 3 \rightarrow 4 \rightarrow 5 \rightarrow 1, \
    2 \rightarrow 3 \rightarrow 4 \rightarrow 6 \rightarrow 2
\)

\( 3 \rightarrow 4, \ 4 \rightarrow 5, \ 5 \rightarrow 1 \) очевидны.

\( 1 \rightarrow 3 \):

Приведем Гауссом к улучшенному ступенчатому виду. 
Одно решение, значит у нас \( n \) ступенек.
Так как матрица квадратная, значит у нас в каждом столбце и каждой строчке есть лидер,
то есть улучшенный ступенчатый вид --- это \( E_n \).
Все преобразования --- умножения на элементарную матрицу.
Возьмем обратные матрицы (тоже элементарные), умножим на них обратно.
Получим \( A \).

Второй цикл аналогично.

Тогда как искать обратную матрицу? Нужно решить \( AX = E \).

\subsection{Блочные умножения}

\begin{gather*}
    \begin{pmatrix}
        A & B \\
        C & D
    \end{pmatrix}
    \begin{pmatrix}
        X & Y \\
        Z & W
    \end{pmatrix}
    =
    \begin{pmatrix}
        AX + BZ & AY + BN \\
        CX + DZ & CY + DW
    \end{pmatrix}
\end{gather*}

\newpage

Примеры:
\begin{enumerate}
    \item
        \begin{gather*}
            \begin{pmatrix}
                \ & \ & \ \\
                \ & A & \ \\
                \ & \ & \
            \end{pmatrix}
            \begin{pmatrix}
                \ & \vline & \ & \vline & \ \\
                B_1 & \vline & \cdots & \vline & B_n \\
                \  & \vline & \ & \vline & \
            \end{pmatrix}
            =
            \\
            =
            \begin{pmatrix}
                A
            \end{pmatrix}
            \begin{pmatrix}
                B_1 & \cdots & B_n
            \end{pmatrix}
            =
            \\
            =
            \begin{pmatrix}
                A B_1 & A B_2 & \cdots & A B_n
            \end{pmatrix}
        \end{gather*}
    \item
        \begin{gather*} 
            \begin{pNiceMatrix}
                \ & X_1 & \\
                \hline
                \ & \vdots & \ \\
                \hline
                \ & X_n & \
            \end{pNiceMatrix}
            \begin{pmatrix}
                \ & \vline & \ & \vline & \ \\
                Y_1 & \vline & \cdots & \vline & Y_n \\
                \  & \vline & \ & \vline & \
            \end{pmatrix}
            =
            \\
            =
            \begin{pmatrix}
                X_1 \\
                \vdots \\
                X_n
            \end{pmatrix}
            \begin{pmatrix}
                Y_1 & \cdots & Y_n
            \end{pmatrix}
            =
            \\
            = X_1 Y_1 + \ldots + X_n Y_n
        \end{gather*}
\end{enumerate}

\subsection{Единственность УСВ для ОСЛУ}

\begin{thrm}{}{uniq-RREF}
    Для \( Ax = 0 \) существует только один улучшенный ступенчатый вид.
\end{thrm}

Пусть
\[
    P_k(A) = ( \forall x : Ax = 0 \ \land \ (x_{k + 1} = x_{k + 2} = \ldots = x_n = 0) \Rightarrow x_k = 0 )
\]

\begin{lemma}{}{main-var}
    Дана ступенчатая матрица \( A \)
    \[
        P_k(A) = true \Leftrightarrow x_k - \text{главная}
    \]
\end{lemma}

Доказательство:
\begin{itemize}
    \item
        \( \Rightarrow \)
        
        Пусть \( x_k \) не является главной.
        Но тогда мы можем назначить ей абсолютно любое значение и \( P_k(A) \) не будет верно.
    \item 
        \( \Leftarrow \)

        \( x_k \) главная \( \Rightarrow \) однозначно восстанавливается по \( x_{k + 1}, \ldots, x_n \).
        Нетрудно убедиться, что если \( x_{k + 1} = \ldots = x_n \), то \( x_k = 0 \).
\end{itemize}

\begin{lemma}{}{uniq-REF}
    Пусть \( Ax = 0 \Leftrightarrow Bx = 0 \) 
    
    Приведём их к ступенчатым видам \( S_A, S_B \)

    Тогда (главные для \( S_A \)) \( \Leftrightarrow \) (главные для \( S_B \))
\end{lemma}

Доказательство
Очевидно, что у ступенчатого вида такие же решения, как и у изначальной матрицы.
Но тогда
\[
    P_k(S_A) = P_k(A) = P_k(B) = P_k(S_B)
\]

Так как верно \ref{lemma:main-var}, это доказывает лемму.

Из \ref{lemma:uniq-REF} следует, что ступенчатый вид матрицы всегда выглядит одинаково.

Осталось показать, что совпадают УСВ.

Единственная проблема может возникнуть в каком-то столбце, где нет лидера.

Сделаем решение \( A \), где у всех свободных переменных кроме одной значение \( 0 \),
а у оставшейся \( 1 \). 
Тогда главные примут значений столбца этой переменной с минусом.
Подставим в \( B \). Поскольку системы эквивалентны, а главные переменные совпадают, 
главные в \( B \) должны совпадать с столбцом той свободной переменной, умноженному
на \( -1 \). 
Но это значит, что столбцы этой переменной в \( A \) и \( B \) равны.
Проделаем такое для всех переменных \( \Rightarrow \) УСВ \( A \) и \( B \) совпадают.

\begin{thrm}{}{}
    Пусть \( A, B \in M_{mn}(\RR) \)

    Следующие утверждения равносильны:
    \begin{enumerate}
        \item
            \( Ax = 0 \Leftrightarrow Bx = 0 \)
        \item
            \( A \) можно элементарными операциями превратить в \( B \)
        \item
            \( \exists \) обратимая \( C \in M_m(\RR) : B = CA \)
        \item
            \( \text{УСВ}_A = \text{УСВ}_B \)
    \end{enumerate}
\end{thrm}

Доказательство:

\begin{itemize}
    \item
        \( 2 \rightarrow 1, 2 \rightarrow 3 \) очевидны
    \item
        \( 1 \rightarrow 4 \) доказано выше
    \item
        \( 3 \rightarrow 2 \) следует из \ref{thrm:invert-equiv}
    \item
        \( 4 \rightarrow 2 \)

        Приведем \( A \) к \( \text{УСВ}_A \), \( B \) к \( \text{УСВ}_B \).
        Они равны. Сделаем обратные элементарные преобразования.
\end{itemize}

\subsection{Полиномиальное исчисление}

Пусть есть
\[
    f = a_0 + a_1 x + \ldots + a_m x^m, A \in M_n(\RR)
\]

Подставим туда \( A \)
\[
    f(A) = a_0 + a_1 A + \ldots + a_m A^m
\]

Имеется проблема \( a_0 \) --- число, а все остальные слагаемые --- матрицы.
На самом деле при \( a_0 \) стоит \( x^0 \).
Поэтому на самом деле все будет так
\[
    f = a_0 E + a_1 A + \ldots + a_m A^m
\]

\begin{lemma}{}{}
    \[
        A \in M_n (\RR), f, g \in \RR [x]
    \]

    Тогда:
    \begin{enumerate}
        \item
            \[ 
                (f + g) (A) = f(A) + g(A)
            \]
        \item
            \[
                (f \cdot g) (A) = f(A) \cdot g(A)
            \]
        \item
            \[
                f(\lambda E) = f(\lambda) E
            \]
        \item
            \begin{gather*}
                C \in M_n (\RR) : \exists C^{-1}
                \\
                f(C^{-1} A C) = C^{-1} f(A) C^1
            \end{gather*}

    \end{enumerate}
\end{lemma}

\begin{lemma}{}{}
    \begin{gather*}
        A \in M_n (\RR)
        \\
        \Downarrow
        \\
        \exists f \in \RR [x] : \deg f \leq n^2 \ (\deg f \leq n), f(A) = 0
    \end{gather*}
\end{lemma}

Доказательство:

Рассмотрим
\[
    f = a_0 + a_1 x + \ldots + a_{n^2} x^{n^2}
\]

Подставим туда \( A \)
\[
    f(A) = a_0 E + a_1 A + \ldots + a_{n^2} A^{n^2} = 0
\]

Напишем равенство для каждого элемента.
Получится ОСЛУ на \( n^2 + 1 \) и \( n^2 \) уравнений.
А значит существует ненулевое решение \( \Rightarrow \) есть подходящий многочлен.

\begin{example}{}{}
    \[
        \begin{pmatrix}
            \lambda_1 & \ldots & 0 \\
            \vdots  & \ddots & \vdots \\
            0 & \ldots & \lambda_n
        \end{pmatrix}
    \]

    Зануляющий многочлен:
    \[
        f = (x - \lambda_1) \cdots (x - \lambda_n)
    \]

    Интересный факт:

    Числа над диагональю могут быть любыми,
    но многочлен все еще будет занулять.
\end{example}

\subsection{Спектр}

\begin{defn}{}{}
    \begin{gather*}
        A \in M_n (\RR)
        \\
        \spec_\RR A = \{ \lambda \in \RR \ \vline \ A - \lambda E - \text{необратима} \}
    \end{gather*}
\end{defn}

\begin{example}{}{}
    \begin{gather*}
        A
        =
        \begin{pmatrix}
            \lambda_1 & \ldots & 0 \\
            \vdots  & \ddots & \vdots \\
            0 & \ldots & \lambda_n
        \end{pmatrix}
        \\
        \spec_\RR A = \{ \lambda_1, \lambda_2, \ldots, \lambda_n \}
    \end{gather*}
\end{example}

\begin{example}{}{}
    \begin{gather*}
        A
        =
        \begin{pmatrix}
            \lambda_1 & \ldots & * \\
            \vdots  & \ddots & \vdots \\
            0 & \ldots & \lambda_n
        \end{pmatrix}
        \\
        \spec_\RR A = \{ \lambda_1, \lambda_2, \ldots, \lambda_n \}
    \end{gather*}
\end{example}

Доказательство:

Если после сдвига на \( \lambda E \) на диагонали нет нулевого элемента,
то у нее, очевидно, \( n \) лидеров \( \Rightarrow \) она обратима.
А если есть, то у нее \( < n \) лидеров \( \Rightarrow \) элементарными преобразованиями можно получить нулевую строку,
поэтому матрица необратима.

\begin{lemma}{}{}
    \begin{gather*}
        A \in M_n (\RR), g \in \RR [x] : g(A) = 0
        \\
        \Downarrow
        \\
        \spec_\RR A \subseteq \text{корни \( g \) в \( \RR \)}
    \end{gather*}

    Аналогично \( \spec_\CC (A) \subseteq \text{корни \( g \) в \( \CC \)} \)
\end{lemma}

Доказательство:

Будем доказывать \( \lambda \neq 0, g(\lambda) \neq 0 \Rightarrow \lambda \notin \spec_\RR A \).

Покажем, что \( A - \lambda E \) обратима.

\begin{gather*}
    x = \lambda \Rightarrow g = q \cdot (x - \lambda) + r
    \\
    g = q \cdot (x - \lambda) + g(\lambda)
    \\
    \text{Подставим сюда \( A \)}
    \\
    0 = g(A) = q(A) (A - \lambda E) + g(\lambda) E 
    \\
    -g(\lambda) E = q(A) (A - \lambda E)
    \\
    E = (A - \lambda E) \frac{q(A)}{-g(\lambda)}
\end{gather*}

\begin{defn}{Минимальный многочлен}{}
    \[
        A \in M_n (\RR), 0 \neq h \in \RR [x]
    \]

    \( h \) называется минимальным, если
    \begin{itemize}
        \item
            \( h(A) = 0 \)
        \item
            \( \deg h \to \min \)
        \item
            Старший коэффициент \( h \) равен \( 1 \)
    \end{itemize}
\end{defn}

\begin{lemma}{}{}
    \[
        A \in M_n ( \RR )
    \]

    \begin{enumerate}
        \item
            Минимальный многочлен \( h \) единственный.
        \item
            \( \forall g \in \RR [x] : g(A) = 0 \) верно \( f_{min} \vline g \)
        \item
            \( \spec_\RR A = \) корни \( f_{min} \) в \( \RR \)
    \end{enumerate}
\end{lemma}

Доказательство:
\begin{enumerate}
    \item
        Пусть есть \( h1, h2 \).
        Тогда \( h1 - h2 \) тоже зануляющий многочлен, причем степени меньше \( n \) (так как старшие члены равны \( 1 \)).
        Получается, что этот многочлен должен быть нулевым, а значит \( h1 = h2 \).
    \item
        \begin{gather*}
            g = q \cdot f_{min} + r
            \\
            \deg r < \deg f_{min}
            \\
            g(A) = q(A) \cdot f_{min} (A) + r(A)
            \\
            0 = 0 + r(A)
            \\
            r(A) = 0
        \end{gather*}
    \item
        Будем доказывать, что \( \spec_\RR A \supseteq \) корни \( f_{min} \) в \( \RR \)

        Докажем от противного.
        \begin{gather*}
            f_{min} (\lambda) = 0
            \\
            f_{min} (\lambda) = (x - \lambda) h(x)
            \\
            0 = f_{min} (A) = (A - \lambda E) h(A)
        \end{gather*}

        Если \( A - \lambda E \) обратима, то \( h(A) \) тоже является зануляющим, 
        причем степени на \( 1 \) меньше --- противоречие.
        Значит \( A - \lambda E \) необратима \( \Rightarrow \lambda \in \spec_\RR A \).
\end{enumerate}

\begin{example}{}{}
    \begin{gather*}
        A
        =
        \begin{pmatrix}
            0 & -1 \\
            1 & 0 \\
        \end{pmatrix}
        \\
        A^2
        =
        \begin{pmatrix}
            -1 & 0 \\
            0 & -1
        \end{pmatrix}
        =
        -E
    \end{gather*}

    В таком случае \( h(x) = x^2 + 1 \) является зануляющим.

    У него нет вещественных корней, а значит \( \spec_\RR A = \varnothing \).
\end{example}
