\section{Система линейных уравнений}

\subsection{Как выглядит}
\[
\begin{cases}
	a_{1 1}x_{1} + a_{1 2}x_{2} + \dots + a_{1 n}x_{n} = b_1 \\
	\cdots \\
	a_{m 1}x_{1} + a_{m 2}x_{2} + \dots + a_{m n}x_{n} = b_m
\end{cases}
\]

Система линейных уравнений называется однородной, если \( \forall i \: b_i = 0 \)

\[
\begin{pmatrix}
c_1 \\
c_2 \\
\vdots \\
c_n
\end{pmatrix} - \text{решение}
\]

Решение - какой-то столбец из \( \mathbb{R}^n \)

\bigskip

$\mathbb{R}^n = \{
    \begin{pmatrix}
    a_1 \\
    \vdots \\
    a_n
    \end{pmatrix}
    \ | \
    a_i \in \mathbb{R}
\}$

\bigskip

Пример:
\[
\begin{cases}
	1x + 2y + 3z = 0 \\
	4x + 5y + 6z = 0 \\
	7x + 8y + 9z = 0
\end{cases}
\]

Будем записывать СЛУ так:
\[
\begin{pmatrix}
    a_{1 1} & a_{1 2} & \dots & a_{1 n} & \vline & b_1 \\
    \vdots & \vdots & \vdots & \vdots & \vline & \vdots \\
    a_{m 1} & a_{m 2} & \dots & a_{m n} & \vline & b_m \\
\end{pmatrix}
\]

Может иметь:
\begin{itemize}
    \item 1 решение:
      
    $x = 1$
    \item 0 решений:
    
		$0 \cdot x = 1$
    \item $\infty$ решений:
    
		$0 \cdot x = 0$
\end{itemize}


\subsection{Как решать}
\[ \Sigma \rightarrow \Sigma_1 \rightarrow \Sigma_2 \rightarrow \dots \rightarrow \Sigma_k - \text{хорошая} \]


Какие преобразования можно делать?
\begin{itemize}
  \item Умножить строчку на ненулевое число
  \[
  \begin{pmatrix}
    \vdots & \vdots & \vdots & \vdots & \vline & \vdots \\
    a_{i 1} & a_{i 2} & \dots & a_{i n} & \vline & b_i \\
    \vdots & \vdots & \vdots & \vdots & \vline & \vdots \\
  \end{pmatrix}
  \rightarrow
  \begin{pmatrix}
    \vdots & \vdots & \vdots & \vdots & \vline & \vdots \\
    \lambda \cdot a_{i 1} & \lambda \cdot a_{i 2} & \dots & \lambda \cdot a_{i n} & \vline & \lambda \cdot b_i \\
    \vdots & \vdots & \vdots & \vdots & \vline & \vdots \\
  \end{pmatrix}, \lambda \neq 0
  \]
  \item Поменять две строки местами
  \[
  \begin{pmatrix}
    \vdots & \vdots & \vdots & \vdots & \vline & \vdots \\
    a_{i 1} & a_{i 2} & \dots & a_{i n} & \vline & b_i \\
    \vdots & \vdots & \vdots & \vdots & \vline & \vdots \\
    a_{j 1} & a_{j 2} & \dots & a_{j n} & \vline & b_j \\
    \vdots & \vdots & \vdots & \vdots & \vline & \vdots \\
  \end{pmatrix}
  \rightarrow
  \begin{pmatrix}
    \vdots & \vdots & \vdots & \vdots & \vline & \vdots \\
    a_{j 1} & a_{j 2} & \dots & a_{j n} & \vline & b_j \\
    \vdots & \vdots & \vdots & \vdots & \vline & \vdots \\
    a_{i 1} & a_{i 2} & \dots & a_{i n} & \vline & b_i \\
    \vdots & \vdots & \vdots & \vdots & \vline & \vdots \\
  \end{pmatrix}
  \]
  \item Прибавить строчку к другой
  \[
  \begin{pmatrix}
    \vdots & \vdots & \vdots & \vdots & \vline & \vdots \\
    a_{i 1} & a_{i 2} & \dots & a_{i n} & \vline & b_i \\
    \vdots & \vdots & \vdots & \vdots & \vline & \vdots \\
    a_{j 1} & a_{j 2} & \dots & a_{j n} & \vline & b_j \\
    \vdots & \vdots & \vdots & \vdots & \vline & \vdots \\
  \end{pmatrix}
  \rightarrow
  \begin{pmatrix}
    \vdots & \vdots & \vdots & \vdots & \vline & \vdots \\
    a_{i 1} + a_{j, 1} & a_{i 2} + a_{j 2} & \dots & a_{i n} + a_{j n} & \vline & b_i + b_j \\
    \vdots & \vdots & \vdots & \vdots & \vline & \vdots \\
    a_{j 1} & a_{j 2} & \dots & a_{j n} & \vline & b_j \\
    \vdots & \vdots & \vdots & \vdots & \vline & \vdots \\
  \end{pmatrix}
  \]
\end{itemize}

Очевидно, что такие преобразования не уменьшают множество решений.
Так как можем сделать обратные преобразования, СЛУ равносильны.

\subsection{Алгоритм Гаусса}

\begin{itemize}
  \item Прямой ход:
  
  Приводим матрицу к ступенчатому виду с помощью преобразований
  \item Обратный ход:
  
	Сделать все лидеры равными $1$

	Поднимаемся снизу вверх, зануляя все над лидером
\end{itemize}
