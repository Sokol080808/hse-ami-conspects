\section{Дискретная теория вероятностей}

\subsection{Вероятность}

\begin{defn}{Вероятностное пространство}{}
    Вероятностное пространство --- \( (\Omega, P) \),
    где \( \Omega = \{ \omega_1, \omega_2, \ldots, \omega_n \} \) --- множество элементарных исходов,
    а \( P \) --- функция \( \Omega \to \RR \),
    обладающая свойствами:
    \begin{itemize}
        \item
            \( 0 \leq P(\omega_i) \leq 1 \)
        \item
            \( \sum\limits_{i = 1}^{n} P(\omega_i) = 1 \)
    \end{itemize}
\end{defn}

\begin{example}{}{}
    \begin{enumerate}
        \item
            Подбрасывание монетки: \( \Omega = \{ \text{О}, \text{Р} \} \), вероятности --- \( \frac{1}{2} \)
        \item
            Подбрасывание кубика: \( \Omega = \{ 1, 2, 3, 4, 5, 6 \} \), вероятности --- \( \frac{1}{6} \)
    \end{enumerate}
\end{example}

\begin{defn}{Событие}{}
    Событие \( A \) --- подмножество \( \Omega \)

    Вероятность события \( P(A) = \sum\limits_{\omega_i \in A} P(\omega_i) \)
\end{defn}

\begin{example}{}{}
    \begin{enumerate}[start=3]
        \item
            Подбрасывание нечестной монетки \( t \) раз

            Решка падает с вероятностью \( p \), орел с \( 1 - p \)

            Тогда \( \Omega = \{ 0, 1 \}^t \)

            \( P( \{ a_1, \ldots, a_t \} ) = p^{|a|} \cdot (1 - p)^{t - |a|} \)
        \item
            Равновероятный случай

            \( P(\omega_i) = \frac{1}{n} \)

            Тогда \( P(A) = \frac{|A|}{n} \)
    \end{enumerate}
\end{example}

\subsection{Дерево событий}

Не повезло, мне лень это рисовать в tikz.

Условно, если хотим как-то описать все события вида ``случайная перестановка из символов \( (1, 2, 3) \)''
можно просто сделать бор, написав на ребрах вероятности.

\subsection{Свойства вероятности}

\begin{enumerate}
    \item
        \( P(\varnothing) = 0, P(\Omega) = 1 \)
    \item
        \( P(A \sqcup B) = P(A) + P(B) \)
    \item
        \( P(\overline{A}) = 1 - P(A) \)
    \item
        \( A \subseteq B \Rightarrow P(A) \leq P(B) \)
    \item
        Формула включений-исключений

        \begin{thrm}{}{}
            Пусть \( (\Omega, P) \) --- вероятностное пространство, \( A_1, \ldots, A_n \) --- события.
            Тогда выполнено
            \[
                P(A_1 \cup \ldots \cup A_n)
                =
                \sum\limits_{k = 1}^n (-1)^{k + 1} \sum\limits_{1 \leq i_1 < \ldots < i_k \leq n} P(A_{i_1} \cap \ldots \cap A_{i_k})
            \]
        \end{thrm}

        Введем индикаторную функцию \( I_A : \Omega \to R \), \( I_A(x) = (x \in A) \).

        Отсюда следует несколько очевидных свойств:
        \begin{itemize}
            \item
                \( I_{A \cap B} = I_A \cdot I_B \)
            \item
                \( I_{\overline{A}} = 1 - I_A \)
            \item
                \( I_{A \cup B} = I_A + I_B - I_{A \cap B}  \)
        \end{itemize}

        \begin{gather*}
            I_{\overline{A_1 \cup \ldots \cup A_n}} = I_{\overline{A_1} \cap \ldots \cap \overline{A_n}} = 
                I_{\overline{A_1}} \cdot \ldots \cdot I_{\overline{A_n}} = (1 - I_{A_1}) \ldots (1 - I_{A_n})
            \\
            \Downarrow
            \\
            I_{A_1 \cup \ldots \cup A_n} = 1 - (1 - I_{A_1}) \ldots (1 - I_{A_n})
        \end{gather*}

        Раскрыв скобочки в последнем равенстве получим искомое.
\end{enumerate}

\begin{example}{Задача о беспорядках}{}
    \( \Omega = S_n \), перестановки равновероятны.

    \( \sigma \in S_n \) --- беспорядок, если \( \forall i \: \sigma(i) \neq i \)

    Необходимо посчитать \( P(\text{\( \sigma \) --- беспорядок}) \)
\end{example}

\( Y_i = \{ \sigma \in S_n \: \vline \: \sigma(i) = i \} \)

\( P(\text{\( \sigma \) --- не беспорядок}) = P(Y_1 \cup \ldots \cup Y_n) \)

\begin{gather*}
    P(Y_{i_1} \cap \ldots \cap Y_{i_k}) = \frac{(n - k)!}{n!}
    \\
    \Downarrow
    \\
    P(Y_1 \cup \ldots \cup Y_n) = \sum\limits_{k = 1}^{n} (-1)^{k + 1} C_n^k \frac{(n - k)!}{n!}
    =
    \sum\limits_{k = 1}^{n} \frac{(-1)^{k + 1}}{k!}
    \\
    \Downarrow
    \\
    P(\text{\( \sigma \) --- беспорядок})
    =
    1 - \sum\limits_{k = 1}^{n} \frac{(-1)^{k + 1}}{k!}
    =
    \sum\limits_{k = 0}^{n} \frac{(-1)^{k}}{k!}
\end{gather*}

При большом \( n \) такая штука стремится к \( \displaystyle \frac{1}{e} \)

\subsection{Условная вероятность}

\( P(A | B) \) --- вероятность события \( A \) при условии \( B \)

\[
    P(A | B) = \frac{P(A \cap B)}{P(B)}
\]

\begin{example}{}{}
    Есть коробки, пронумерованные от \( 1 \) до \( 4 \), и один шарик.

    Сначала равновероятно выбирается коробка, а потом подбрасывается монетка,
    и с вероятностью \( \frac{1}{2} \) шарик кладется в выбранную коробку.

    Ведущий показал, что в первых трех коробках нет шара,
    какова вероятность, что он в четвертой?
\end{example}

\( P(B) = \frac{5}{8} \) и \( P(A \cap B) = \frac{1}{8} \)

Получаем, что ответ равен \( \frac{1}{5} \)

\subsection{Формулы полной вероятности и Байеса}

\begin{thrm}{Формула полной вероятности}{}
    Пусть \( (\Omega, P) \) --- вероятностное пространство и \( \Omega = \bigsqcup\limits_{i = 1}^m A_i \),
    причем \( P(A_i) > 0 \)

    Тогда \( P(B) = \sum\limits_{i = 1}^{n} P(B | A_i) \cdot P(A_i) \)
\end{thrm}

\[
    \sum\limits_{i = 1}^{n} P(B | A_i) \cdot P(A_i)
    =
    \sum\limits_{i = 1}^{n} P(A_i \cap B)
    =
    P(B \cap (A_1 \cup \ldots \cup A_n))
    =
    P(B)
\]

\begin{thrm}{Формула Байеса}{}
    Пусть \( (\Omega, P) \) --- вероятностное пространство.

    Есть события \( A, B \) ненулевой вероятности.

    Тогда \( \displaystyle P(B | A)= \frac{P(A | B) \cdot P(B)}{P(A)} \)
\end{thrm}

\begin{gather*}
    P(B | A) \cdot P(A) = P(A \cap B) = P(A | B) \cdot P(B)
    \\
    \Downarrow
    \\
    P(B | A)= \frac{P(A | B) \cdot P(B)}{P(A)}
\end{gather*}

\begin{example}{}{}
    Болеет \( 1\% \) населения.

    Тест на болезнь ошибается в \( 1\% \) случаев.
\end{example}

Посчитаем \( P(\text{человек болен (A) | тест положительный (B)}) \)

\begin{gather*}
    P(A | B)
    =
    \frac{P(B | A) \cdot P(A)}{P(B)}
    =
    \frac{99\% \cdot 1\%}{P(B | A) \cdot P(A) + P(B | \overline{A}) \cdot P(\overline{A})}
    =
    \\
    =
    \frac{99\% \cdot 1\%}{99\% \cdot 1\% + 1\% \cdot 99\%}
    =
    0.5
\end{gather*}

\subsection{Независимые события}

Пусть \( (\Omega, P) \) --- вероятностное пространство.

\begin{defn}{}{}
    \begin{itemize}
        \item
            \( A \) и \( B \) --- независимые события, если \( P(A \cap B) = P(A) \cdot P(B) \)
        \item
            \( A_1, \ldots, A_m \) --- независимые в совокупности,
            если для любого подмножества \( A_{i_1}, \ldots, A_{i_k} \)
            выполнено
            \[
                P(A_{i_1} \cap \ldots \cap A_{i_k}) = P(A_{i_1}) \cdot \ldots \cdot P(A_{i_k})
            \]
    \end{itemize}
\end{defn}

Приведем пример, показывающий недостаточность попарной независимости для независимости в совокупности:
\begin{example}{}{}
    Подбросим монетку дважды.

    \( A_1 \) --- в первом броске выпал орел

    \( A_2 \) --- во втором броске выпал орел

    \( B \) --- выпал ровно один орел

    Нетрудно убедиться, что события попарно независимы, но при этом
    \[
        P(A_1 \cap A_2 \cap B) = 0 \neq \frac{1}{2} \cdot \frac{1}{2} \cdot \frac{1}{2}
    \]
\end{example}

\begin{exercise}{}{}
    Докажите, что при подбрасывании монетке события вида \( A_i = (\text{в \( i \)-м броске выпал орел}) \) независимы в совокупности. 
\end{exercise}
