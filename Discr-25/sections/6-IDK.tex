\section{Формула включений-исключений}
\begin{thrm}{Формула включений-исключений}{}
    \[
        | A_1 \cup A_2 \cup \ldots \cup A_n | = 
        \sum\limits_{k = 1}^n \sum\limits_{1 \leq i_1 < \ldots < i_k \leq n} | A_{i_1} \cap \ldots \cap A_{i_k} |
    \]
\end{thrm}

Доказательство (через характеристические функции):
% TODO

\section{Опять множества}
\begin{defn}{Равномощные множества}{}
    Множества \( A \) и \( B \) равномощны, если существует биекция \( f : A \to B \)

    Обозначается \( A \sim B, |A| = |B| \)
\end{defn}

\begin{example}{}{}
    \begin{itemize}
        \item
            \( \NN \sim \NN \cup \{ 0 \} \) \( n \mapsto n - 1 \)
        \item
            \( [0, 1] \sim [0, 2] \) \( x \mapsto 2x \)
        \item
            \( (0, 1) \sim (1, +\infty) \) \( x \mapsto \frac{1}{x} \)
    \end{itemize}
\end{example}

\begin{exercise}{}{}
    \( [a, b] \sim [c, d], \ a, b, c, d \in \RR \)
\end{exercise}

Свойства:
\begin{enumerate}
    \item
        \( \forall A \ A \sim A \) (рефлексивность)
    \item 
        \( \forall A, B \ A \sim B \Rightarrow B \sim A \) (симметричность)
    \item
        \( \forall A, B, C \ A \sim B, B \sim C \Rightarrow A \sim C \) (транзитивность)
\end{enumerate}

\begin{defn}{Счетные множества}{}
    Множество \( A \) счетно, если \( A \sim \NN \)
\end{defn}
\begin{example}{}{}
    \begin{itemize}
        \item
            \( \ZZ \sim \NN \)
        \item
            \( \QQ \sim \NN \)
    \end{itemize}
\end{example}

\begin{lemma}{}{}
    \begin{itemize}
        \item
            \( A, B \) --- счетны \( \Rightarrow A \cup B \) --- счетно
        \item
            \( A \) --- конечно, \( B \) --- счетно \( \Rightarrow A \cup B \) --- счетно
    \end{itemize}
\end{lemma}

\begin{lemma}{}{}
    \( A \) --- счетно, \( B \subseteq A \).

    Тогда \( B \) не более чем счетно.
\end{lemma}

\begin{lemma}{}{}
    \( A \) --- бесконечно, тогда существует счетное \( B \subseteq A \)
\end{lemma}

Доказательство:

Будем строить итеративно: берем любой \( a_{n + 1} \in A \setminus \{ a_1, a_2, \ldots, a_n \} \).
\[
    B_n = \{ a_1, \ldots, a_n \}, B = \bigcup\limits_{n = 1}^{\infty} B_n
\]

\begin{lemma}{}{}
    \( A_1, A_2, \ldots, A_n, \ldots \) --- счетны

    Тогда \( \displaystyle \bigcup\limits_{n = 1}^{\infty} \Rightarrow A_n \) --- счетно
\end{lemma}

\begin{lemma}{}{}
    \( A, B \) --- счетны

    Тогда \( A \times B \) --- тоже счетно
\end{lemma}

Доказательство:
\[
    A \times B = \bigcup\limits_{n = 1}^{\infty} A \times \{ B_n \}
\]

\begin{thrm}{}{}
    Пусть \( A \) --- бесконечно, а \( B \) --- счетно.

    Тогда \( A \cup B \sim A \)
\end{thrm}

Доказательство:

\begin{gather*}
    A \cup B = A \cup ( B \setminus A )
    \\
    A \cup B = A \cup B', A \cap B' = \varnothing
\end{gather*}

Рассмотрим счетное \( C \)

\( C \sim C \cup B' \)


\( \exists f : C \to C \cup B' \) --- биекция

Построим биекцию \( g : A \to A \cup B' \)
\[
    g =
    \begin{cases}
        x, x \notin C 
        \\
        f(x), x \in C
    \end{cases}
\]

\subsection{Несчетные множества}

Счетные (\(\sim A\)): \( \NN, \ZZ, \QQ, \NN^k, \bigcup\limits_{k = 1}^n \NN^k \)

\begin{defn}{}{}
    \[
        \BB^\infty = \{ (b_1, b_2, \ldots) \ \vline \ b_i \in \{ 0, 1 \} \}
    \]
\end{defn}

\begin{thrm}{}{}
    \[
        \BB^\infty \not\sim \NN
    \]
\end{thrm}

\begin{lemma}{}{}
    \[ 
        \BB^\infty \sim [0, 1) \sim [0, 1] \sim (0, 1) \sim \RR \sim 2^\NN 
    \]
\end{lemma}

Доказательство:

\( \BB^\infty \sim [0, 1) \), так как можно сопоставить последовательности число в двоичной записи.

Все кроме последнего тривиально.

\( 2^\NN \sim \BB^\infty \), так как по сути выбираем что взяли в набор, а что нет.



\begin{defn}{}{}
    Множество \( A \) имеет мощность континуум, если \( A \sim \RR \)
\end{defn}

\begin{thrm}{}{}
    \[
        [0, 1] \sim [0, 1]^2
    \]
\end{thrm}

Будем доказывать \( \BB^\infty = ( \BB^\infty )^2 \)

\( (x_1, x_2, \ldots) \leftrightarrow (x_1, x_3, \ldots ) (x_2, x_4, \ldots ) \)

\begin{coroll}{}{}
    \[ 
        \RR^k \sim \RR 
    \]
\end{coroll}

\begin{defn}{}{}
    \( |A| \leq |B| \), если существует инъекция \( f : A \to B \)

    \( |A| < |B| \), если \( |A| \leq |B| \) и \( |A| \neq |B| \)
\end{defn}

Свойства:
\begin{itemize}
    \item
        \( |A| \leq |B| \Leftrightarrow \exists A' \subseteq B : A' \sim A \)
    \item
        \( |A| \leq |A| \)
    \item
        \( |A| \leq |B|, |B| \leq |C| \Rightarrow |A| \leq |C| \)
    \item
        \( |A| \leq |B|, |B| \leq |A| \Rightarrow |A| \sim |B| \) (теорема Кантора-Бернштейна)
    \item
        \( \forall A, B \ |A| \leq |B| \lor |B| \leq |A| \)
\end{itemize}

\begin{thrm}{}{}
    \[
        |X| < |2^X|
    \]
\end{thrm}

Доказательство:
\begin{enumerate}
    \item
        \( |X| \leq |2^X| \)

        \( \exists f : X \to 2^X \), \( f(x) = {x} \) --- инъекция
    \item
        \( |X| \neq |2^X| \)

        Докажем от противного. 
        Пусть существует \( f : X \to 2^X \).

        Определим \( Y = \{ x \in X \ \vline \ x \notin f(x) \} \subseteq X \).

        Поскольку \( Y \in 2^X, \exists y \in X : f(y) = Y \)
        \begin{itemize}
            \item
                \( y \in Y \Rightarrow y \notin f(y) = Y \) --- противоречие
            \item
                \( y \notin Y \Rightarrow y \in f(y) = Y \) --- противоречие
        \end{itemize}
\end{enumerate}

\begin{coroll}{}{}
    Существует бесконечно много мощностей.
    \[
        \NN < 2^\NN < 2^{2^\NN} < \ldots
    \]
\end{coroll}

\begin{thrm}{Теорема Кантора Бернштейна}{}
    \[ 
        |A| \leq |B|, |B| \leq |A| \Rightarrow |A| = |B|
    \]
\end{thrm}

Доказательство:

Пусть \( f : A \to B, g : B \to A \) --- инъекции.

Определим \( C_0 = A \setminus g(B) \). 
\( A \supseteq C_{n + 1} = g(f(C_n)), n \geq 0 \)

\( C = \bigcup\limits_{n \geq 0} C_n \)

Возьмем
\(
    h = 
    \begin{cases}
        f(x), x \in C
        \\
        g^{-1}(x), x \notin C
    \end{cases}
\), (\( x \notin C \Rightarrow x \notin C_0 \Rightarrow x \in g(B) \))

Покажем, что мы построили биекцию.
\begin{enumerate}
    \item
        \( h \) --- инъекция

        \( h(x_1) = h(x_2) \)
        
        \begin{itemize}
            \item
                \( x_1, x_2 \in C \Rightarrow f(x_1) = f(x_2) \Rightarrow x_1 = x_2 \)
            \item
                \( x_1, x_2 \notin C \Rightarrow g^{-1} (x_1) = g^{-1} (x_2) \Rightarrow x_1 = x_2 \)
            \item
                \( x_1 \in C, x_2 \notin C \Rightarrow f(x_1) = g^{-1} (x_2) \). 
                Но тогда \( g(f(x_1)) = x_2 \).
                \( x_1 \in C_n \Rightarrow x_2 \in C_{n + 1} \) --- противоречие.
        \end{itemize}
    \item
        \( h \) --- сюръекция

        Хотим найти такой \( x \), что \( h(x) = y \)

        \begin{itemize}
            \item
                \( y \in f(C) \Rightarrow y = f(x) = h(x) \)
            \item
                \( y \notin f(C) \).

                Рассмотрим \( g(y) \)

                Если \( g(y) \notin C \), то \( h(g(y)) = y \).

                В противном случае \( g(y) \in C_n, n > 0 \).

                В таком случае \( g(y) = g(f(x')), x \in C_{n - 1} \Rightarrow y = f(x') = h(x') \), так как \( x' \in C \).
        \end{itemize}
\end{enumerate}
