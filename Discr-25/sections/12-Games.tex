\section{Комбинаторные игры}

\subsection{Определения}

\begin{defn}{}{}
    \( G = (V, E) \) --- ориентированный граф (считаем его конечным и ациклическим, для удобства)

    MIN, MAX --- игроки

    \( V \) --- ``позиции'' игры, в каждой позиции определено, кто ходит: MIN или MAX

    \( E \) --- ходы в игре

    \( s \in V \) --- стартовая вершина

    \( T \subseteq V \) --- терминальный позиции (игра в них кончается).

    Понятно, что \( T = \{ v \in V \: | \: \deg_+ (v) = 0 \} \).

    \( f : T \to \RR \) --- функция выигрыша для MAX: если игра заканчивается в \( t \in T \),
    то он выигрывает \( f(t) \), а MIN получает \( -f(t) \).
\end{defn}

\begin{defn}{}{}
    Партия --- путь в графе игры \( G \), стартующий в \( s \) и завершающийся в \( t \in T \).

    Результат партии --- \( f(t) \)
\end{defn}

\begin{defn}{}{}
    Стратегия --- это функция \( g: V \setminus T \to V \)
\end{defn}

\begin{defn}{}{}
    MAX может гарантировать себе выигрыш \( \geq c \),
    если \( \exists g_{MAX} \forall g_{MIN} \) результат партии \( \geq c \).
\end{defn}

\begin{defn}{}{}
    Говорят, что цена игры \( = c \), если MAX может гарантировать себе выигрыш \( \geq c \),
    а MIN может гарантировать себе выигрыш \( \leq c \).
\end{defn}

\begin{thrm}{}{}
    Для любой комбинаторной игры существует цена \( c \), причем \( c \) единственная.
\end{thrm}

Сначала покажем единственность.

Пусть у игры есть две цены \( c_1 > c_2 \).

Тогда \( \exists g_{MAX} \) с результатом \( \geq c_1 \) и \( \exists g_{MIN} \) с результатом \( \leq c_2 \).

Если сыграть \( g_{MAX} \) против \( g_{MIN} \), то результат будет \( \geq c_1 > c_2 \), что противоречит тому, что \( g_{MIN} \) гарантирует результат \( \leq c_2 \).

Теперь докажем существование.

Возьмем топологическую сортировку вершин графа игры \( v_1, v_2, \dots, v_n \),
причем \( v_1 = s \), а все вершины из \( T \) идут в конце.

Будем доказывать индукцией по суффиксу вершин, что цена игры определена и реализуется на согласованных стратегиях:

\begin{itemize}
    \item
        База:

        Для всех вершин из \( T \) цена игры --- это \( f(t) \).
    \item
        Переход:

        Пусть \( v_i \) --- вершина, в которой ходит MAX.

        Тогда \( c_i = \max\limits_{(v_i, v_j) \in E} c_j \),
        продолжаем стратегию через максимум.

        Пусть \( v_i \) --- вершина, в которой ходит MIN.

        Тогда \( c_i = \min\limits_{(v_i, v_j) \in E} c_j \),
        продолжаем стратегию через минимум.
\end{itemize}

\subsection{Беспристрастная игра Ним}

\begin{defn}{}{}
    Беспристрастная игра --- игроки ходят по очереди, проигрывает тот, кто не может сделать ход.

    \( c = \pm 1 \)
\end{defn}

\begin{defn}{}{}
    Ним --- это беспристрастная игра, в которой есть \( k \) кучек, в каждой кучке \( a_i \) камней.

    Во время хода может взять \( > 0 \) камней из одной кучки.
\end{defn}

\begin{thrm}{}{}
    \( c = -1 \) тогда и только тогда, когда \( a_1 \oplus a_2 \oplus \dots \oplus a_k = 0 \).
\end{thrm}

\begin{itemize}
    \item
        \( a_1 \oplus a_2 \oplus \dots \oplus a_k = 0 \)

        Очевидно, что после любого хода XOR перестанет быть нулем.
    \item
        \( a_1 \oplus a_2 \oplus \dots \oplus a_k \neq 0 \).

        Пусть \( a_i \) --- кучка, в которой стоит старшая единица в XOR.
        Тогда можно сделать ход в \( a_i \), чтобы XOR стал нулем:
        уберем эту старшую единицу, а все более младшие биты в \( a_i \) сделаем такими же, как в XOR
        (можем так сделать, потому что после удаления ``старшей единицы'' на более младших позициях
        можно получить любую комбинацию битов)
\end{itemize}

Поскольку \( k \) пустых кучек --- терминальная позиция, то \( c = -1 \) тогда и только тогда, когда \( a_1 \oplus a_2 \oplus \dots \oplus a_k = 0 \).
