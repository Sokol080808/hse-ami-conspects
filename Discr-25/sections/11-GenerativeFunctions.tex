\section{Производящие функции}

\subsection{что-то}

\begin{gather*}
    a_n = 0 + 1 + \ldots + n
    \\
    A(x) = \frac{x}{(1 - x)^3}
    \\
    a_n = \left. \frac{A^{(n)} (x) }{n!} \right|_{x = 0}
    \\
    B(x) = (1 - x)^{-3}
    \\
    B^{(n)} (x) = 3 \cdot 4 \cdot \ldots \cdot (n + 2) \cdot (1 - x)^{-n - 3}
\end{gather*}

\( \frac{B^{(n)} (0)}{n!} = \frac{3 \cdot 4 \cdot \ldots \cdot (n + 2)}{n!} = \frac{(n + 2)(n + 1)}{2} = b_n \)

\( a_n = \frac{(n + 1)n}{2} \)

\subsection{Связь ПФ с неупорядоченными выборками}

\( S, T \) --- множества, \( S \cap T \neq \varnothing \)

\( A(x) \) --- ПФ неупорядоченных выборок из \( S \)

\( B(x) \) --- ПФ неупорядоченных выборок из \( T \)

Тогда \( A(x) B(x) \) --- ПФ неупорядоченных выборок из \( S \cup T \)

\begin{example}{}{}
    \begin{itemize}
        \item
            Бином Ньютона

            \( \{ a_1, \ldots, a_n \} : \binom{n}{k} \) способов выбрать \( k \) элементов

            \( C(x) = \sum\limits \binom{n}{k} x^k = (1 + x)^n \)

            Поскольку изначальный набор представлялся в виде объединения \( \{ a_i \} \)
            этот же результат можно было получить, сразу возведя \( (1 + x) \) в степень \( n \).
        \item
            Количество салатов

            перец: \( 0 \) или \( 1 \)

            редиска: \( 0, 2, 4, \ldots \)

            помидор: любое

            баклажан: \( \leq 3 \)

            Какое есть число салатов из \( n \) овощей при данных ограничениях?

            \( (1 + x)(1 + x^2 + x^4 + \ldots)(1 + x + x^2 + \ldots)(1 + x + x^2 + x^3) = \frac{(1 + x)(1 + x + x^2 + x^3)}{(1 - x^2)(1 - x)} \)
    \end{itemize}
\end{example}

\begin{defn}{}{}
    Пусть \( \alpha \in \CC \), \( k \in \NN_0 \)

    Тогда \( \binom{\alpha}{k} = \frac{\alpha \cdot (\alpha - 1) \cdot \ldots \cdot (\alpha - k + 1)}{k!} \)
\end{defn}

\begin{remark}{}{}
    \begin{gather*}
        \binom{-n}{k}
        =
        \frac{(-n) \cdot (-n - 1) \cdot \ldots \cdot (-n - k + 1)}{k!}
        =
        \\
        =
        (-1)^k \cdot \frac{n \cdot (n + 1) \cdot \ldots \cdot (n + k - 1)}{k!}
        =
        (-1)^k \binom{n + k - 1}{k}
    \end{gather*}
\end{remark}

\begin{thrm}{}{}
    \[
        (1 + x)^n = \sum\limits_{k = 0}^{\infty} \binom{n}{k} x^k \quad \forall n \in \ZZ
    \]
\end{thrm}

Рассмотрим \( (1 - x)^{-n} \), \( n \in \NN \).

\( (1 - x)^{-n} = (1 + x + x^2 + \ldots)^n \).

Тогда коэффициент при \( k \) степени --- число разбиений числа \( k \) на неотрицательные целые слагаемые, то есть \( \binom{n + k - 1}{k} \).

Поменяв \( x \) на \( -x \) получим:
\[
    (1 + x)^{n} = \sum\limits_{k = 0}^{\infty} (-1)^k \binom{n + k - 1}{k} x^k = \sum\limits_{k = 0}^{\infty} \binom{-n}{k} x^k
\]

Что и требовалось доказать.

\subsection{Степень}

\begin{defn}{}{}
    \[
        (1 + x)^{\alpha} = \sum\limits_{k = 0}^{\infty} \binom{\alpha}{k} x^k
    \]
\end{defn}

\begin{lemma}{}{}
    \[
        (1 + x)^{\alpha} \cdot (1 + x)^{\beta} = (1 + x)^{\alpha + \beta}
    \]
\end{lemma}

\begin{gather*}
    \left[ (1 + x)^{\alpha} \cdot (1 + x)^{\beta} \right]_n
    =
    \sum\limits_{\substack{s + t = n \\ s, t \geq 0}} \binom{\alpha}{t} \binom{\beta}{s}
    \\
    \\
    \left[ (1 + x)^{\alpha + \beta} \right]_n = \binom{\alpha + \beta}{n}
\end{gather*}

Мы знаем, что при всех натуральных \( \alpha, \beta \) равенство верно.

Зафиксируем \( \tilde{\alpha} \in \NN \).
Тогда ЛЧ(\( \tilde{\alpha}, \beta \)) и ПЧ(\( \tilde{\alpha}, \beta \)) --- многочлены от \( \beta \).
Они равны во всех натуральных точках, а значит совпадают во всех комплексных \( \beta \).
Аналогично можем зафиксировать \( \tilde{\beta} \in \CC \) и повторить идею.
Во всех натуральных точках части снова будут совпадать, а значит доказали.

\begin{defn}{}{}
    Последовательность \( (a_0, \ldots, a_n) \) --- линейное рекуррентное соотношение порядка \( k \) с постоянными коэффициентами,
    если
    \[
        \exists c_1, \ldots, c_k \in \CC : c_k \neq 0, \ \forall n \geq 0 \ a_{n + k} = c_1 a_{n + k - 1} + c_2 a_{n + k - 2} + \ldots + c_k a_n
    \]
\end{defn}

\begin{thrm}{}{}
    Пусть \( A(x) \) --- ПФ линейного рекуррентного соотношения с постоянными коэффициентами порядка \( k \).

    Тогда существуют \( P(x), Q(x) \in \CC[x] : \deg P < k, \deg Q = k, Q(0) \neq 0 \) и \( A(x) = \frac{P(x)}{Q(x)} \)
\end{thrm}

Рассмотрим \( A(x) (c_1 x + c_2 x^2 + \ldots + c_k x^k) \)

Раскрытием скобочек получим, что \( A(x) = A(x) (c_1 x + c_2 x^2 + \ldots + c_k x^k) + P(x) \).

Мы добавляем \( P(x) \), чтобы подправить первые \( k \) коэффициентов (отсюда \( \deg P < k \)), которые могли неправильно посчитаться.

Тогда \( A(x) = \frac{P(x)}{1 - c_1 x - \ldots - c_k x^k} \)

\subsection{Явная формула для линейных рекуррент}

\begin{gather*}
    A(x)
    =
    \frac{P(x)}{Q(x)}
    =
    \frac{P(x)}{(x - a_1)^{t_1} \cdot \ldots \cdot (x - a_s)^{t_s}}
    =
    \\
    =
    \sum\limits_{j = 1}^{s} \sum\limits_{l = 1}^{t_j} \frac{C_{jl}}{(x - a_j)^l}
\end{gather*}

А явные значения коэффициента у степени \( n \) у таких дробей мы умеем считать через бином Ньютона.

\subsection{Правильные скобочные последовательности}

Последовательность из левых и правых скобок (, ) называется правильной скобочной последовательностью (ПСП),
если она может быть получена за конечное число шагов по данным правилам:
\begin{itemize}
    \item
        \( \varnothing \) --- ПСП
    \item
        A --- ПСП \( \Rightarrow \) (A) --- тоже ПСП
    \item
        A, B --- ПСП \( \Rightarrow \) AB --- тоже ПСП
\end{itemize}

Число скобочных последовательностей длины \( 2n \) обозначается \( C_n \) --- числа Каталана

\begin{thrm}{}{}
    Скобочная последовательность является ПСП, тогда и только тогда,
    когда в ней равное число левых и правых скобок,
    а так же на любом префиксе число левых скобок больше или равно числа правых скобок.
\end{thrm}

\begin{itemize}
    \item
        \( \Rightarrow \) 

        Следует из определения ПСП
    \item
        \( \Leftarrow \)

        Докажем достаточность по индукции

        База понятна.

        Переход:

        Рассмотрим наименьший префикс, на котором баланс равен \( 0 \).
        Очевидно, что это закрывающая скобка.
        Тогда наша ПСП равна (A)B.
        Нетрудно убедиться, что у A и B выполнены те же условия на баланс, а значит все хорошо.
\end{itemize}

\begin{coroll}{}{}
    Каждая непустная ПСП представляется в виде (A)B, где A, B --- ПСП, и притом единственным образом.
\end{coroll}

Существование следует из доказательства выше.
Единственность.

Пусть есть \( (A)B \) и \( (A') B' \).

Тогда заметим, что если ``внутри'' \( A' \) или \( A \) есть нулевой баланс в изначальной ПСП,
то кто-то из них не ПСП.
Но это значит, что обе \( A \) и \( A' \) кончаются перед первым нулевым балансом (так как сами являются ПСП).
То есть разбиения совпадают.

\begin{coroll}{}{}
    \[
        C_n = \sum\limits_{k = 0}^{n - 1} C_k \cdot C_{n - 1 - k}
    \]
\end{coroll}

Заметим, что если мы введем производящую, то получим уравнение вида \( x \cdot C^2 (x) = C(x) - c_0 = C(x) - 1 \).

Решив такое квадратное уравнение получим \( C_{12} (x) = \frac{1 \pm \sqrt{1 - 4x}}{2x} \).

Немного (нестрогой) магии:
\begin{gather*}
    x C^2 (x) - C(x) + 1 = x \left( C(x) - \frac{1}{2x} \right)^2 - \frac{1}{4x} + 1
    \\
    x \left( C(x) - \frac{1}{2x} \right)^2 = \frac{1 - 4x}{4x}
    \\
    4x^2 \left( C(x) - \frac{1}{2x} \right)^2 = 1 - 4x
\end{gather*}

Заметим, что полученное выражение равносильно изначальному:
\begin{gather*}
    4x^2 C^2 (x) - 4x C(x) + 1 = 1 - 4x
    \\
    x^2 C(x) - x C(x) + x = 0
    \\
    x (x C^2 (x) - C(x) + 1) = 0
\end{gather*}

Заметим, что \( Y^2 (x) = 1 - 4x \) имеет ровно два решения, так как изначально \( y_0^2 = 1 \),
после чего все коэффициенты определяются однозначно.
По очевидной причине это \( \pm (1 - 4x)^{\frac{1}{2}} \).

Из штуки выше получается, что \( Y(x)  = 2x C(x) - 1 \).

Подставим \( x = 0 \) получим, что \( Y(0) = -1 \), а значит \( Y(x) = -\sqrt{1 - 4x} \)


