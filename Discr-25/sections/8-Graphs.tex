\section{Графы}

\subsection{Определения}

\begin{defn}{Ориентированный граф}{}
    \( G = ( V, E ) \), где \( V \) --- конечное множество (вершины),
    а \( E \subseteq V \times V \) --- ребра.
\end{defn}

\begin{defn}{}{}
    Пусть в ориентированном графе ---
    это последовательность \( a_0, a_1, \ldots, a_k \in V \),
    причем \( \forall \ 0 \leq i \leq k - 1 \) выполнено \( (a_i, a_{i + 1}) \in E \).

    \( k \geq 0 \) --- длина пути.

    Пусть называется простым, если \( a_0, a_1, \ldots, a_k \) различны.
\end{defn}

\begin{exercise}{}{}
    Если в ориентированном графе \( G \) существует путь из \( x \) в \( y \),
    то существует простой путь из \( x \) в \( y \).
\end{exercise}

\begin{defn}{}{}
    \( x, y \in V \) сильно связаны в \( G \), если есть путь как из \( x \) в \( y \), так и из \( y \) в \( x \)
\end{defn}

\begin{lemma}{}{}
    Сильная связность на \( G \) --- отношение эквивалентности на \( V \)
\end{lemma}

\begin{coroll}{}{}
    \( V = \bigsqcup\limits_{i = 1}^t V_i \),
    где \( V_i \) --- класс сильно связных вершин.
\end{coroll}

\begin{defn}{}{}
    \( V_i \) --- компонента сильной связности.
\end{defn}

\begin{defn}{}{}
    \( G \) --- сильно связный, если \( t = 1 \).
\end{defn}

\begin{defn}{Степень вершины}{}
    Исходящая степень: \( d_+(v) = | \{ x \in V \ \vline \ (v, x) \in E \} | \)

    Входящая степень: \( d_-(v) = | \{ x \in V \ \vline \ (x, v) \in E \} | \)
\end{defn}

\begin{thrm}{Лемма о рукопожатиях}{}
    \[
        \sum\limits_{v \in V} d_+(v) = \sum\limits_{v \in V} d_-(v) = |E|
    \]
\end{thrm}

\begin{defn}{}{}
    Цикл в ориентированном графе \( G \) --- путь длины \( k \geq 1 \), в котором \( a_0 = a_k \).
\end{defn}

\begin{defn}{Ацикличность}{}
    Граф \( G \) называют ациклическим, если в \( G \) нет циклов.
\end{defn}

\begin{thrm}{}{}
    Пусть \( G \) --- ориентированный граф без петель.

    Тогда следующие утверждения эквивалентны:
    \begin{enumerate}
        \item
            \( G \) --- ациклический.
        \item
            Все компоненты сильной связности в \( G \) одноэлементны.
        \item
            Можно пронумеровать вершины \( G \) в от \( 1 \) до \( n \) так,
            что \( (i, j) \in E \Rightarrow i < j \)

            (топологическая сортировка)
    \end{enumerate}
\end{thrm}

Доказательство:
\begin{itemize}
    \item
        \( 1 \to 2 \)

        Если размер какой-то компоненты сильной связности хотя бы \( 2 \), то в графе, очевидно, есть цикл
        (по определению компоненты сильной  связности).
    \item
        \( 3 \to 1 \)

        Перенумеруем граф.
        Пусть есть цикл \( a_0, a_1, \ldots, a_k \).

        Тогда \( a_0 < a_1 < \ldots < a_k = a_0 \) --- противоречие.
    \item
        \( 2 \to 3 \)

        Докажем по индукции по количеству вершин:
        \begin{enumerate}
            \item
                \( n = 1 : \)

                Верно
            \item
                \( n \to n + 1 : \)

                Покажем, что есть какая-то вершина, из которой не выходит ребра.

                Если это не так, то давайте сделаем граф конденсации.
                В нем нет циклов.
                Мы знаем, что из каждой компоненты сильной связности выходит
                хотя бы одно ребро в другую компоненту (так как \( |V_i| = 1 \)).
                Давайте пойдем по этим ребрам.
                Когда-нибудь мы посетим какую-нибудь вершину дважды.
                Поскольку в графе нет петель, мы нашли цикл, а такого в графе конденсации быть не может.

                Противоречие \( \Rightarrow \) есть вершина без исходящего ребра.

                Теперь давайте выкинем эту вершину, пронумеруем оставшиеся (по индукции),
                после чего присвоим выкинутой вершине номер \( n + 1 \).
        \end{enumerate}

        Научились получать топологическую сортировку.
\end{itemize}

\subsection{Эйлеровы графы}

\begin{defn}{}{}
    Граф \( G \) --- эйлеров, если в нем существует цикл, содержащий все ребра \( G \) по одному разу.
\end{defn}

\begin{thrm}{Критерий Эйлеровости}{}
    \begin{enumerate}
        \item
            Пусть \( G \) --- неориентированный граф без изолированных вершин.

            Тогда \( G \) эйлеров
            \( \Leftrightarrow \)
            \( G \) связен и \( \forall v \in V \ \deg(v) \) --- четное.
        \item
            Пусть \( G \) --- ориентированный граф без изолированных вершин (\(d_-(v) = d_+(v) = 0\)).

            Тогда \( G \) эйлеров
            \( \Leftrightarrow \)
            \( G \) сильносвязен и \( \forall v \in V \ d_-(v) = d_+(v) \).
    \end{enumerate}
\end{thrm}

Докажем второе:
\begin{itemize}
    \item
        \( \Rightarrow \)

        Очевидно
    \item
        \( \Leftarrow \)

        Рассмотрим самый длинный путь, в котором все ребра различны.

        \( v_0 \to v_1 \to \ldots \to v_k \)

        Покажем, что \( v_k = v_0 \).

        Пусть это не так, тогда у нас просто путь.

        Пусть \( v_k \) встречается в цикле \( d + 1 \) раз.
        Все ребра из \( v_k \) есть в цикле (иначе можем продлить).
        Поэтому \( d_+(v_k) = d \).
        Но \( d_-(v_k) \geq d + 1 \) --- противоречие с критерием.

        Значит это цикл.
        Покажем, что это эйлеров цикл.

        Пусть это не так, значит существует ребро \( (x, y) \notin \) цикл.

        Существует путь \( v_0 \to x \to y \).
        Тогда найдется такая вершина цикла \( v_i \), что из нее ведет ребро \( (v_i, u) \) не из цикла.
        Но тогда мы можем удлинить наш путь, сделав \( v_i, v_{i + 1}, \ldots, v_i, u \).
\end{itemize}

\begin{defn}{}{}
    \( G = (V, E) \) --- двудольный, если \( \exists X, Y : V = X \sqcup Y \), причем
    \[
        \forall e = \{ a, b \} \in E
        \Rightarrow
        \left[
            \begin{gathered}
                a \in X, b \in Y
                \\
                a \in Y, b \in X
            \end{gathered}
        \right.
    \]
\end{defn}

\begin{thrm}{Критерий двудольности}{}
    \( G \) двудолен \( \Leftrightarrow \) в \( G \) нет циклов нечетной длины.
\end{thrm}

Доказательство:
\begin{itemize}
    \item
        \( \Rightarrow \)

        Пусть есть нечетный цикл \( v_0, v_1, \ldots, v_k \)

        Не умаляя общности \( v_0 \in X \Rightarrow v_1 \in Y \Rightarrow v_2 \in X \Rightarrow \ldots \).

        В силу нечетности цикла получим, что \( v_0 = v_k \in Y \) --- противоречие.
    \item
        \( \Leftarrow \)

        Если \( G \) не связен, то проверим критерий отдельно для каждой компоненты связности.
        Теперь можно считать \( G \) связным.

        Зафиксируем какую-нибудь \( v_0 \in V \).
\end{itemize}

\subsection{Хроматическое число графа}

\subsubsection{Раскраски графа}

\begin{defn}{}{}
    \( G = (V, E) \) правильно раскрашиваем в \( k \) цветов,
    если \( \exists f : V \to \{ 1, \ldots, k \} : \forall \{ x, y \} \in E \ f(x) \neq f(y) \)
\end{defn}

\begin{defn}{Хроматическое число}{}
    Хроматическое число \( G \):

    \( \mathcal{X} (G) = \min \{ k \in \NN \ \vline \ G \ \text{правильно раскрашиваем в \( k \) цветов} \} \)
\end{defn}

\begin{defn}{Кликовое число}{}
    Клика в \( G \) --- это \( W \subseteq V : \forall a \neq b \in W \ \{ a, b \} \in E \)

    Кликовое число \( G \):

    \( \omega(G) = \max \{ m \in \NN \ \vline \ \exists \ \text{клика} \ W \subseteq V, |W| = m \} \)
\end{defn}

\begin{defn}{Число независимости}{}
    Независимое множество в \( G \) --- это \( U \subseteq V : \forall a \neq b \in U \ \{ a, b \} \notin E \)

    Число независимости \( G \):

    \( \alpha(G) = \max \{ m \in \NN \ \vline \ \exists \ \text{независимое множество} \ W \subseteq V, |W| = m \} \)
\end{defn}

Свойства \( \mathcal{X} (G) \)
\begin{enumerate}
    \item
        \( 1 \leq \mathcal{X} (G) \leq n \)
    \item
        \( \mathcal{X} (G) \geq \omega (G) \)
    \item
        \( G = \bigsqcup\limits_{i = 1}^s G_i \) --- разбиение на компоненты связности
        \( \Rightarrow \mathcal{X} (G) = \max\limits_{i = 1, \ldots, s} \mathcal{X} (G_i) \)
    \item
        \( \mathcal{X} (G) \leq n - \alpha(G) + 1 \)
    \item
        \( \mathcal{X} (G) \cdot \alpha (G) \geq n \)

        Доказательство:

        Пусть \( \mathcal{X} (G) = k \).

        Рассмотрим все вершины покрашенный в цвет \( i \): \( V_i \).

        \( V_i \) --- независимое множество \( \Rightarrow |V_i| \leq \alpha (G) \)

        \( n = |V_1| + \ldots + |V_k| \leq \alpha (G) \cdot k \)
    \item
        \( \Delta (G) = \max\limits_{v \in V} \deg v \)

        \( \mathcal{X} (G) \leq \Delta (G) + 1 \).

        Работает жадная покраска: в любой момент времени у вершины покрашено не больше \( \Delta (G) \) соседей,
        поэтому точно найдется отличный от них цвет.
\end{enumerate}

\begin{lemma}{}{}
    Пусть \( G \) связен и \( \exists v \in V : \deg v < \Delta (G) \Rightarrow \mathcal{X} (G) \leq \Delta(G) \)
\end{lemma}

Доказательство:

Пусть эта вершина \( v_0 \).
Сделаем остовное дерево и подвесим его за \( v_0 \).

Будем красить дерево сверху вниз.
Тогда для всех вершин кроме корня \( < \Delta (G) \) детей, так как \( \deg u \leq \Delta (G) \), значит сможем покрасить.
А у корня детей тоже \( < \Delta (G) \), так как \( \deg v_0 < \Delta (G) \).

\begin{thrm}{Теорема Брукса (без доказательства)}{}
    Пусть \( G \) --- связный граф и \( G \neq K_n \) (полный граф), \( C_{2n + 1} \) (простой цикл нечетной длины)

    Тогда \( \mathcal{X} (G) \leq \Delta (G) \)
\end{thrm}

\subsubsection{Хроматический многочлен}

\begin{defn}{Хроматический многочлен}{}
    \( G \) --- граф, \( \mathcal{X}_G(k) \) --- количество правильных раскрасок \( G \) в \( k \) цветов.
\end{defn}

\begin{example}{}{}
    \begin{itemize}
        \item
            Пустой граф на \( n \) вершинах: \( \mathcal{X}_G(k) = k^n \)
        \item
            Полный граф на \( n \) вершинах: \( \mathcal{X}_G(k) = \frac{k!}{(k - n)!} \)
        \item
            Дерево на \( n \) вершинах: \( \mathcal{X}_G(k) = k \cdot (k - 1)^{n - 1} \)
    \end{itemize}
\end{example}

\begin{thrm}{Теорема Уитни}{}
    Пусть \( G \) --- граф на \( n \) вершинах и \( m \) ребрах с \( s \) компонентами связности.

    Тогда \( \mathcal{X}_G(x) = x^n - a_1 x^{n - 1} + a_2 x^{n - 2} - \ldots + (-1)^{n - s} a_{n - s} x^s \),
    где \( a_1, \ldots, a_{n - s} \in \NN \), причем \( a_1 = m \).
\end{thrm}

\begin{lemma}{}{}
    Пусть \( G \) --- граф, \( \{ u, v \} \in E \).

    \( G - uv \) --- удаление ребра \( \{ u, v \} \)

    \( G \cdot uv \) --- стягивание ребра \( \{ u, v \} \)

    Тогда \( \mathcal{X}_{G - uv}(k) = \mathcal{X}_G(k) + \mathcal{X}_{G \cdot uv}(k) \)
\end{lemma}

Раскраски бывают двух разных цветов:
\begin{itemize}
    \item
        \( u \) и \( v \) разных цветов

        Такая раскраска неотличима от раскраски \( G \)
    \item
        \( u \) и \( v \) одинаковых цветов

        Такая раскраска неотличима от раскраски \( G \cdot uv \)
\end{itemize}

Теперь докажем теорему Уитни полной индукцией по \( m \).

База \( m = 0 \): проверили выше

Переход \( < m \to m \):

Рассмотрим ребро \( \{ u, v \} \in E \).

По лемме \( \mathcal{X}_G(x) = \mathcal{X}_{G - uv}(x) - \mathcal{X}_{G \cdot uv}(x) \)

В \( G \): \( n \) вершин, \( m \) ребер, \( s \) компонент связности.

В \( G - uv \): \( n \) вершин, \( m - 1 \) ребер, \( s \) или \( s + 1 \) компонента связности.

В \( G \cdot uv \): \( n - 1 \) вершин, \( < m \) ребер, \( s \) компонент связности.
\begin{gather*}
    \mathcal{X}_G(x)
    =
    (x^n - (m - 1) x^{n - 1} + a_2 x^{n - 2} - \ldots + (-1)^{n - s} a_{n - s} x^s)
    -
    \\
    -
    (x^{n - 1} - b_1 x^{n - 1} + b_2 x^{n - 2} - \ldots + (-1)^{n - 1 - s} b_{n - 1 - s} x^s)
    =
    \\
    = x^n - m x^{n - 1} + (a_2 + b_1) x^{n - 2} - (a_3 + b_2) x^{n - 3} + \ldots + (-1)^{n - s} (a_{n - s} + b_{n - 1 - s}) x^s
\end{gather*}

Поскольку \( a_1, a_2, \ldots, a_{n - 1 - s}, b_1, b_2, \ldots, b_{n - 1 - s} > 0 \) и \( a_{n - s} \geq 0 \),
теорема доказана.

\subsubsection{Графы с большим хроматическим числом}

Существует ли граф с ``большим'' хроматическим числом, не содержащий треугольников?
\[
    G : \mathcal{X}(G) > k, \omega(G) = 2
\]

Пример Зыкова (1949) - Мыцельского (1955)

\begin{defn}{}{}
    Пусть \( G = (V, E) \) --- граф.

    Мыцельскиан \( G \) --- это граф \( \mu(G) = (V', E') \):
    \begin{gather*}
        V' = \{ v_1, v_2, \ldots, v_n, u_1, u_2, \ldots, u_n, \omega \}
        \\
        E' = E \cup \{ (v_i, u_j) \ \vline \ \forall i, j : (v_i, v_j) \in E \} \cup \{ ( \omega, u_i ) \ \vline \ i = 1, \ldots, n \}
    \end{gather*}
\end{defn}

\begin{thrm}{}{}
    Пусть \( G_2 = K_2, G_3 = \mu(G_2), \ldots, G_t = \mu(G_{t - 1}), t \geq 3 \).

    Тогда \( G_t \) --- граф без треугольников, причем \( \mathcal{X}(G_t) = t \).
\end{thrm}

Докажем индукцией по \( t \).
\begin{itemize}
    \item
        База:

        \( t = 2 \) --- очевидно.

        \( t = 3 \) --- \( G_3 = C_5 \) --- очевидно.
    \item
        Переход \( < t \to t \).

        Покажем, что в \( G_t \) нет треугольников.

        Поскольку \( u_i \) не соединено с \( u_j \),
        возможен только треугольник вида \( v_i, v_j, u_k \).

        \( i, j \neq k \), так как иначе в \( G_{t - 1} \) были бы петли.

        С другой стороны, если в \( G_t \) есть такой треугольник,
        в \( G_{t - 1} \) был бы треугольник \( v_i, v_j, v_k \) --- противоречие.

        Теперь докажем, что \( \mathcal{X}(G_t) \leq t \).

        Знаем, что \( \mathcal{X}(G_{t - 1}) = t - 1 \), поэтому давайте покрасим
        \( v_i \), а после сделаем у \( u_i \) такой же цвет как и у \( v_i \).

        \( \omega \) покрасим в новый цвет.

        Теперь сделаем оценку в другую сторону: \( \mathcal{X}(G_t) \geq t \).

        От противного: \( \mathcal{X}(G_t) = t - 1 \) (так как \( \mathcal{X}(G_{t - 1}) = t - 1 \)).

        Допустим, раскрасили \( G_t \) в \( t - 1 \) цвет, причем \( \omega \) покрашена в \( t - 1 \).

        Тогда \( u_i \) покрашены в цвета \( 1, \ldots, t - 2 \).

        Давайте научимся красить \( G_{t - 1} \) в \( t - 2 \) цвета:
        \[
            c'(v_i) = \begin{cases}
                c(v_i)\ \text{если} \ c(v_i) \neq t - 1
                \\
                c(u_i), \ \text{если} \ c(v_i) = t - 1
            \end{cases}
        \]

        Покажем, что эта раскраска корректна.
        Когда раскраска могла сломаться?
        Если у каких-то вершин совпали цвета, то значит, мы перекрасили одну из них.

        \( c'(v_i) = c(u_i), c'(v_j) = c(v_j) \).

        Но в \( G_t \) есть ребро \( (u_i, v_j) \) --- противоречие.

        По предположению индукции \( G_{t - 1} \) нельзя корректно
        покрасить в \( t - 2 \) цвета --- снова противоречие.

        Значит \( \mathcal{X}(G_t) = t \)
\end{itemize}

\subsection{Паросочетания и вершинные покрытия}

\begin{defn}{Паросочетание}{}
    Пусть \( G \) --- граф,
    паросочетание в \( G \) --- это \( M \subseteq E : \forall m_1 \neq m_2 \in M \ m_1 \cap m_2 = \varnothing \)
\end{defn}

\begin{defn}{Вершинное покрытие}{}
    Пусть \( G \) --- граф,
    вершинное покрытие в \( G \) --- это \( U \subseteq V : \forall \{ a, b \} \in \ a \in U \lor b \in U \)
\end{defn}

\begin{prop}{}{}
    Пусть \( G \) --- граф, \( M \) --- паросочетание в \( G \), \( U \) --- вершинное покрытие в \( G \).

    Тогда \( |M| \leq |U| \)
\end{prop}

Среди концов каждого ребра паросочетания должна быть хотя бы одна вершина из вершинного покрытия --- доказали.

\begin{coroll}{}{}
    \( \max |M| \leq \min |U| \).
\end{coroll}

\begin{thrm}{Теорема Кенига}{}
    Если \( G \) --- двудольный граф, то \( \max |M| = \min |U| \)
\end{thrm}

\( G = (L \cup R, E) \)

\begin{defn}{Чередующийся путь}{}
    Чередующийся путь относительно \( M \) --- это простой по ребрам путь длины хотя бы один,
    стартующий в \( a \in L \), не покрытой \( M \),
    ребра в котором чередуются: \( \notin M, \in M, \notin M, \ldots \)
\end{defn}

\begin{defn}{Увеличивающий путь}{}
    Увеличивающий путь относительно \( M \) --- чередующийся путь, завершающийся в \( b \in R \),
    не покрытой \( M \)
\end{defn}

\begin{exercise}{}{}
    Если существует увеличивающий путь относительно \( M \), то \( M \) не максимально.
\end{exercise}

\begin{exercise}{}{}
    \( M \) --- максимальное паросочетание \( \Leftrightarrow \) не существует увеличивающего пути относительно \( M \)
\end{exercise}

Докажем теорему Кенига:

Рассмотрим паросочетание \( M \) максимального размера.

Строим \( U \subseteq V \):
\begin{gather*}
    \forall \{ x, y \} \in M
    \\
    \begin{cases}
        y \in U, \ \text{\( \exists \) чередующийся путь относительно \( M \), оканчивающийся в \( y \)}
        \\
        x \in U, \ \text{иначе}
    \end{cases}
\end{gather*}

\newpage

Рассмотрим ребро \( \{ a, b \} \in E \).

Для \( \{ a, b \} \in M \) очевидно, дальше считаем, что \( \{ a, b \} \notin M \)
\begin{enumerate}
    \item
        \( a \) не покрыта \( M \)
        \begin{enumerate}
            \item
                \( b \) не покрыта \( M \)

                Тогда \( \{ a, b \} \) в \( M \) --- увеличивающий путь \( \Rightarrow \) противоречие.
            \item
                \( b \) покрыта \( M \)

                Тогда \( b \in U \), так как \( \{ a, b \} \) --- чередующийся путь
                относительно \( M \), заканчивающийся в \( b \)
        \end{enumerate}
    \item
        \( a \) покрыта \( M \)

        В таком случае \( \exists \{ a, b' \} \in M \).
        Если \( b' = b \), то \( \{ a, b \} \) лежит в \( M \)
        \( \Rightarrow \) точно покрыто \( U \). Значит можно считать \( b' \neq b \).

        Если \( a \in U \), то точно верно, поэтому дальше считаем, что \( b' \in U \).
        Это значит, что существует чередующийся путь из \( a' \) в \( b' \).
        Возьмем такой кратчайший путь.
        По очевидной причине в нем нет ребра \( \{ b', a \} \).
        Если ребра \( \{ a, b \} \) нет в нашем пути, то добавим его и получим чередующийся путь из \( a' \) в \( b \),
        а если есть, то, получается, некоторый префикс пути является чередующимся из \( a' \) в \( b \).

        В любом случае найдем чередующийся путь из \( a' \) в \( b \)

        \begin{enumerate}
            \item
                \( b \) не покрыта \( M \)

                Тогда чередующийся путь из \( a \) в \( b \) --- увеличивающий \( \Rightarrow \) противоречие.
            \item
                \( b \) покрыта \( M \)

                \( \{ a'', b \} \in M \).

                Но тогда \( b \in U \), так как существует чередующийся путь из \( a' \) в \( b \).
        \end{enumerate}
\end{enumerate}

Разобрали все случаи \( \Rightarrow \) доказали теорему.

\newpage

\begin{thrm}{Лемма Холла}{}
    Пусть \( G = (L \cup R, E) \)

    \( S \subseteq L \), пусть \( N(S) = \{ y \in R \ \vline \ \exists x \in S : \{ x, y \} \in E \)

    Существует паросочетание мощности \( |L| \) тогда и только тогда,
    когда \( \forall S \subseteq L \) верно \( |S| \leq |N(S)| \).
\end{thrm}

Доказательство:
\begin{itemize}
    \item
        \( \Rightarrow \)

        Тривиально.
    \item
        \( \Leftarrow \)

        Если вершина \( R \) является изолированной, выкинем ее.
        Теперь из каждой вершины \( R \) выходит какое-нибудь ребро в \( L \).

        Используем теорему Кенига: \( \max |M| = \min |U| \).

        Рассмотрим какое-то \( U \).

        Теперь пусть \( L = L_1 \sqcup L_2 \), \( R = R_1 \sqcup R_2 \), причем \( U = L_1 \cup R_2 \).

        Тогда нетрудно понять, что между \( L_2 \) и \( R_1 \) нет ребер,
        но при этом есть между \( L_2 \) и \( R_2 \); \( L_1 \) и \( R_2 \); \( L_1 \) и \( R_1 \).

        \( |L_2| \leq |N(L_2)| \leq |R_2| \).
        Тогда \( |U| = |L_1| + |R_2| \geq |L_1| + |L_2| = |L| \).

        Но \( L \), очевидно, является вершинным покрытием, поэтому
        \( \min |U| = |L| \). Значит и \( \max |M| = |L| \), что и требовалось доказать.
\end{itemize}

\subsection{Числа Рамсея}

\begin{exercise}{}{}
    Из \( 6 \) человек можно выбрать трех попарно знакомых или попарно незнакомых людей
    (знакомство взаимно).
\end{exercise}

\begin{defn}{}{}
    Пусть \( n, k \geq 1\)

    Тогда \( R(n, k) \) --- минимальное число \( N \in \NN : \)
    \( \forall G \) на \( \geq N \) вершинах в \( G \) найдется клика размера \( n \)
    или независимое множество размера \( k \).
\end{defn}

Свойства:
\begin{enumerate}
    \item
        \( R(n, k) = R(k, n) \)
    \item
        \( R(1, k) = 1 \)
    \item
        \( R(2, k) = k \)
\end{enumerate}

\begin{thrm}{}{}
    \( R(n, k) \leq R(n - 1, k) + R(n, k - 1) \)
\end{thrm}

Докажем индукцией по сумме:

База \( n = 2, 3 \): косвенно описана выше в свойствах

Рассматриваем \( n + k \):

Пусть \( R(n - 1, k) + R(n, k - 1) = N \).

Рассмотрим граф с \( \geq N \) вершинами.

Посмотрим на какую-нибудь вершину \( v \).

Пусть \( X \) --- множество вершин, с которым \( v \) соединено, \( Y \) --- остальные.

Тогда \( |X| \geq R(n - 1, k) \) или \( |Y| \geq R(n, k - 1) \).

В обоих случаях найдем либо клику размера \( n \), либо независимое множество размера \( k \).

\begin{coroll}{}{}
    \[
        R(n, k) \leq C(n + k - 2, n - 1) = C(n + k - 2, k - 1)
    \]
\end{coroll}

Доказательство: очевидно по индукции.
