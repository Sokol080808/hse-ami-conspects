\section{Частично упорядоченные множества}

\subsection{Отношения частичного порядка}

\begin{defn}{Отношение строгого частичного порядка}{}
    Отношение \( R \) на \( A \) называется отношением строгого порядка, если:
    \begin{enumerate}
        \item
            \( \lnot \ a \ R \ a \) (иррефлексивность)
        \item
            \( a \ R \ b \ \land \ b \ R \ c \Rightarrow a \ R \ c \) (транзитивность)
    \end{enumerate}

    \( R \: = \: < \)
\end{defn}

\begin{defn}{Отношение нестрогого частичного порядка}{}
    Отношение \( R \) на \( A \) называется отношением нестрогого порядка, если:
    \begin{enumerate}
        \item
            \( a \ R \ a \) (рефлексивность)
        \item
            \( a \ R \ b \ \land b \ R \ a \Rightarrow a = b \) (антисимметричность)
        \item
            \( a \ R \ b \ \land \ b \ R \ c \Rightarrow a \ R \ c \) (транзитивность)
    \end{enumerate}

    \( R \: = \: \leq \)
\end{defn}

\begin{lemma}{О связи строгого и нестрогого порядков}{}
    \begin{itemize}
        \item
            \( \leq \) --- нестрогий порядок на \( A \)
            \( \Rightarrow \: < \: = \: \leq \: \setminus \{ (a, a) \ \vline \ a \in A \} \)
            - отношение строгого порядка на \( A \)
        \item
            \( < \) --- строгий порядок на \( A \)
            \( \Rightarrow \: \leq \: = \: < \: \cap \{ (a, a) \ \vline \ a \in A \} \)
            - отношение строгого порядка на \( A \)
    \end{itemize}
\end{lemma}

Доказательство:
\begin{itemize}
    \item
        \begin{enumerate}
            \item
                \( \lnot \ a \ R \ a \) --- очевидно.
            \item
                \( a \ R \ b \ \land b \ R \ c \Rightarrow a \neq b, b \neq c, a \leq b, b \leq c \Rightarrow a \leq c\)

                Если \( a = c \), то \( a \leq b, b \leq a \Rightarrow a = b \) --- противоречие.
        \end{enumerate}
    \item
        \begin{enumerate}
            \item
                \( \ a \ R \ a \) --- очевидно.
            \item
                Пусть \( a \leq b \land b \leq a \)

                Допустим, что \( a \neq b \).
                Тогда \( a < b \land b < a \), получили противоречие иррефлексивности.
            \item
                \( a \leq b \land b \leq c \)

                Если \( a = b \) или \( b = c \), то очевидно.

                Иначе \( a < b \land b < c \Rightarrow a < c \Rightarrow a \leq c \)
        \end{enumerate}
\end{itemize}

\subsection{Частично упорядоченные множества}

\begin{defn}{Частично упорядоченное множество}{}
    Множество \( A \neq \varnothing \) с заданным на нем отношением
    частичного порядка называется частично упорядоченным множеством.

    \( (A, <), (A, \leq) \)
\end{defn}

\begin{example}{}{}
    \begin{enumerate}
        \item
            \( (\NN, \leq) \)
        \item
            \( (\NN, |) \) (отношение делимости)
        \item
            \( (\ZZ \setminus \{ 0 \}, |) \) --- не ЧУМ (\( (-1) | 1, 1 | (-1) \), но \( -1 \neq 1 \))
        \item
            \( (2^A, \subseteq) \)
        \item
            \( (\BB^n, \leq) \)
    \end{enumerate}
\end{example}

\subsection{Операции над порядками}
\begin{enumerate}
    \item
        Покоординатный порядок \( (P \times Q, \leq) \):

        \(
            (p_1, q_1) \leq (p_2, q_)
            \Leftrightarrow
            \begin{cases}
                p_1 \leq_P p_2
                \\
                q_1 \leq_Q q_2
            \end{cases}
        \)
    \item
        Лексикографический порядок \( (P \times Q, \leq) \):

        \(
            (p_1, q_1) \leq (p_2, q_2)
            \Leftrightarrow
            p_1 < p_2 \lor (p_1 = p_2 \land q_1 \leq q_2)
        \)
    \item
        Считаем, что \( P \cap Q = \varnothing \)

        \( P + Q = (P \sqcup Q, \leq): \)

        \( x \leq y
            \Leftrightarrow
            \left[
                \begin{aligned}
                    & x \leq_P y
                    \\
                    & x \leq_Q y
                    \\
                    & x \in P, y \in Q
                \end{aligned}
            \right.
        \)
\end{enumerate}

\begin{defn}{Изоморфизм}{}
    Пусть \( (P, \leq_P), (Q, \leq_Q) \).

    Тогда говорят, что они изоморфны, если существует биекция \( \varphi : P \to Q \) такая,
    что \( \forall x, y \in P \ x \leq_P y \Leftrightarrow \varphi(x) \leq_Q \varphi(y) \).

    Обозначается \( (P, \leq_P) \cong (Q, \leq_Q) \)
\end{defn}

\begin{example}{}{}
    \begin{enumerate}
        \item
            \( (\NN, \leq) \cong (\NN \cup \{ 0 \}) \ \varphi(n) = n - 1 \)
        \item
            \( (\QQ, \leq) \not\cong (\RR, \leq) \), так как нет биекции.
        \item
            \( ([0, 1], \leq) \not\cong ((0, 1), \leq) \), так как нет наименьшего и наибольшего элементов.
    \end{enumerate}
\end{example}

\begin{defn}{}{}
    \( a \in P \) --- минимальный, если \( \not\exists b \in P : b < a \)

    \( a \in P \) --- наименьший, если \( \forall b \in P \ a \leq b \)

    \( a \in P \) --- максимальный, если \( \not\exists b \in P : b > a \)

    \( a \in P \) --- наибольший, если \( \forall b \in P \ a \geq b \)
\end{defn}

\begin{example}{}{}
    \( (Z, \leq) \not\cong (Q, \leq) \)
\end{example}

\begin{defn}{Плотный порядок}{}
    \( \forall a < b \ \exists c : a < c \land c < b \)
\end{defn}

\( (Q, \leq) \) --- плотный порядок, а \( (Z, \leq) \) --- нет.

\begin{defn}{Отрезок}{}
    Для \( a \leq b \) \( [a, b] = \{ c \in P \ \vline \ a \leq c \land c \leq b \} \)
\end{defn}

Если \( \varphi: P \to Q \) --- изоморфизм порядков, то \( \phi([a, b]) = [\phi(a), \phi(b)] \).
Доказательство тривиально.

\newpage

\subsection{Фундированные подмножества}

\begin{defn}{Фундированное множество}{}
    ЧУМ \( (P, \leq) \) называется фундированным, если \( \forall X \subseteq P, x \neq \varnothing \)
    имеет минимальный элемент.
\end{defn}

\begin{example}{}{}
    \begin{itemize}
        \item
            \( (N, \leq) \) --- фундированное
        \item
            \( (Z, \leq) \) --- не фундированное, возьмем \( X = \ZZ \)
    \end{itemize}
\end{example}

\begin{thrm}{}{}
    Для ЧУМ-а \( (P, \leq) \) эквивалентны следующие условия:
    \begin{enumerate}
        \item
            \( \forall X \subseteq P, x \neq \varnothing \Rightarrow X \) имеет минимальный элемент.
        \item
            \( \not\exists \) бесконечно убывающей цепи \( p_1 > p_2 > p_3 > \ldots \)
        \item
            Для \( P \) справедлив принцип индукции:

            \(
                \forall p \in P ( ( \forall q < p \ A(q) - \text{ист} ) \Rightarrow A(p) - \text{ист} )
                \Rightarrow
                \forall p \in P A(p) - \text{ист}
            \)
    \end{enumerate}
\end{thrm}

Доказательство:
\begin{enumerate}
    \item
        \( 1 \Rightarrow 2 \)

        Докажем от противного: есть бесконечная цепь \( \Rightarrow \) нет минимального
    \item
        \( 2 \Rightarrow 1 \)

        Вновь докажем от противного.
        Построим бесконечную цепь:  в \( X \subseteq P \) нет минимального,
        значит всегда можем найти новый элемент для цепи.
    \item
        \( 1 \Rightarrow 3 \)

        Пусть \( X = \{ p \in P \ \vline \ A(p) - \text{ложно} \} \neq \varnothing \).

        Рассмотрим \( p' \) --- минимальный элемент в \( X \).

        Но тогда \( \forall q < p' \) \( A(q) \) --- ист.
        Значит \( A(p') \) --- ист \( \Rightarrow \) противоречие \( \Rightarrow X = \varnothing \).
    \item
        \( 3 \Rightarrow 1 \)

        Пусть \( X \subseteq P, X \neq \varnothing \), и \( A(p) = p \notin X \)

        Допустим, что \( X \) не имеет минимальных элементов.

        \( \forall p \in P ( ( \forall q < p \ q \notin X) \Rightarrow p \notin X ) \) --- верно,
        так как иначе \( p \) --- минимальный элемент.

        Но тогда \( \forall p \ p \notin X \Rightarrow X = \varnothing \) --- противоречие.

        Значит у \( \forall X \subseteq P \) есть минимальный элемент.
\end{enumerate}

\subsection{Что-то про изоморфизмы}

\begin{thrm}{}{}
    Пусть \( (P, \leq_P), \ (Q, \leq_Q) \) --- счетные плотные линейные порядки без наименьшего и наибольшего элементов.

    Тогда \( (P, \leq_P) \cong (Q, \leq_Q) \)
\end{thrm}

Доказательство:
\begin{gather*}
    P = \{ p_1, p_2, p_3, \ldots \}
    \\
    Q = \{ q_1, q_2, q_3, \ldots \}
\end{gather*}

Строим изоморфизм \( \varphi : P \to Q \)

Возьмем наименьший невзятый элемент из \( P \), пусть это \( p \).

Пусть отсортированные уже взятые \( p_i \) --- \( a_1, \ldots, a_k \),
а \( q_i \) --- \( b_1, \ldots, b_k \).

Он расположен между какими-то двумя уже взятыми \( a_i \leq q \leq a_{i + 1} \) (или перед \( a_0 \) / после \( a_{k - 1} \))

Но засчет плотности и отсутствия наибольшего (и наименьшего элемента) у нас найдется элемент из \( Q \),
который находится между \( b_i \) и \( b_{i + 1} \).

Давайте продолжать выбирать так пары элементов, поочередно беря то минимальный по номеру невзятый \( p_i \), то \( q_i \).

\subsection{Цепи и антицепи}

Пусть \( (P, \leq) \) --- ЧУМ

\begin{defn}{}{}
    Цепь в \( P \) --- это \( \varnothing \neq C \subseteq P : \forall x, y \in C \ x \leq y \lor y \leq x \)

    Антицепь в \( P \) --- это \( \varnothing \neq A \subseteq P : \forall x \neq y \in A \) \( x, y \) не сравнимы
\end{defn}

\begin{exercise}{}{}
    Если \( C \) --- цепь в \( P \), а \( A \) --- антицепь в \( P \), то \( | A \cap C | \leq 1 \)
\end{exercise}

\begin{exercise}{}{}
    Пусть \( P \) --- конечный ЧУМ

    Пусть множество разбивается на \( k \) цепей.

    Тогда \( \max |A| \leq k \)

    Пусть множество разбивается на \( l \) антицепей.

    Тогда \( \max |C| \leq l \)
\end{exercise}

\begin{thrm}{Теорема Мирского}{}
    Пусть \( P \) --- конечный ЧУМ

    Пусть \( P \) можно разбить на \( l \) антицепей и нельзя разбить на меньшее число.

    Тогда \( \max |C| = l \)
\end{thrm}

Учитывая упражнения выше, достаточно показать, что можно разбить множество на \( \max |C| = l \) антицепей.

\( \min X =  \{ x \in X \ \vline \ x - \text{минимальный в} \ X \} \)

\begin{gather*}
    P_1 = \min P
    \\
    P_2 = \min ( P \setminus P_1)
    \\
    P_3 = \min ( P \setminus (P_1 \cup P_2) )
    \\
    \vdots
    \\
    P_k = \min ( P \setminus (P_1 \cup \ldots \cup P_{k - 1}) )
    \\
    \vdots
\end{gather*}

\( P_1, \ldots, P_m \) --- антицепи, причем не пересекаются.

\( p_m \in P_m \Rightarrow \exists p_{m - 1} \in P_{m - 1} : p_m > p_{m - 1} \).

Значит можем так достать цепь \( p_m > p_{m - 1} > \ldots > p_1 \).

Отсюда \( m \leq \max |C| = l \leq m \Rightarrow l = m \), что и требовалось доказать.

\begin{thrm}{Теорема Дилуорса}{}
    Пусть \( P \) --- конечный ЧУМ

    Пусть \( P \) можно разбить на \( k \) цепей и нельзя разбить на меньшее число.

    Тогда \( \max |A| = k \)
\end{thrm}

Нужно доказать только \( \max |A| \geq k \), неравенство в обратную сторону было раньше в упражнении.

Докажем по индукции по размеру множества:

База: \( s = 1 \) --- очевидно.

Переход: \( < s \to s \)

Рассмотрим минимальный элемент \( m \in P \).

Удалим его: \( P \setminus \{ m \} = P' \).
Рассмотрим в полученном множестве \( \max\limits_{A \in P'} |A| = l \).
Очевидно, что в изначальном множестве \( \max\limits_{A \in P} |A| = l \) или \( l + 1 \).

\begin{itemize}
    \item
        Случай \( \max\limits_{A \in P} |A| = l + 1 \) очевиден, так как в новой антицепи точно содержится \( m \),
        а значит его можно покрыть отдельной цепью из одного элемента.
    \item
        Теперь разберем случай \( \max\limits_{A \in P} |A| = l \).

        Рассмотрим разбиение \( P' \) на цепи \( C_1, C_2, \ldots, C_l \).
        \( p_i \) --- наименьший возможный элемент в \( C_i \), входящий в антицепь размера \( l \) в \( P' \).

        Покажем, что \( \{ p_1, \ldots, p_l \} \) --- антицепь.

        Предположим противное, путь \( p_i < p_j \).

        Рассмотрим антицепь \( A \) размера \( l \), в которой находится \( p_i \).

        \( A \cap C_j = \{ y \} \)

        \( p_j \leq y \), так как \( p_j \) --- наименьший возможный в такой антицепи.

        Но тогда \( p_i < p_j \leq y \) --- противоречие, так как \( p_i \) и \( y \) находятся в \( A \).

        Размер антицепи не меняется при добавлении \( m \) \( \Rightarrow \) \( \exists j : \) \( p_j \) сравним с \( m \).

        \( m \) --- минимальный, поэтому \( m < p_j \).

        Давайте обрежем начало цепи \( C_j \) перед \( p_j \), заменим на \( m \), получим \( C = m \to p_j \to \ldots \).

        Давайте выкинем эту цепь, получим новое множество.
        Понятно, что в этом новом множестве размер антицепи не превышает \( l \),
        так как есть разбиение на \( l \) цепей.

        Но размер не может быть \( l \),
        так как тогда бы в остатке (начале) цепи был элемент из антицепи, меньший \( p_j \).
        Значит антицепи не превышает \( l - 1 \), значит есть разбиение \( P \setminus C \) на \( l - 1 \) цепь.
        Вернем \( C \), получим разбиение \( P \) на \( l \) цепей.

        Значит \( \max |A| \geq l \), что и требовалось доказать.
\end{itemize}

\subsection{Цепи и антицепи в булевом кубе}

\( \BB^n = \{ 0, 1 \} \), \( \leq \) --- покоординатно, то есть
\( (a_1, \ldots, a_n) \leq (b_1, \ldots, b_n) \Leftrightarrow \forall i \ a_i \leq b_i \)

\begin{defn}{}{}
    Вес набора \( \tilde{a} = (a_1, \ldots, a_n) \in \BB^n \) --- количество единиц в нем.

    Обозначается \( |\tilde{a}| \)
\end{defn}

\begin{defn}{}{}
    Уровни \( \BB^n \): \( B_0, B_1, \ldots, B_n \)

    \( B_k = \{ \text{все наборы веса \( k \)} \} \)

    Не трудно заметить, что \( |B_k| = C_{n}^{k} \)
\end{defn}

\begin{exercise}{}{}
    Максимальная цепь в \( \BB^n \) имеет длину \( n + 1 \)
\end{exercise}

\begin{thrm}{Теорема Шпернера}{}
    Максимальный размер антицепи в \( \BB^n \) --- \( C_{n}^{\left\lfloor \frac{n}{2} \right\rfloor} \)
\end{thrm}

\begin{lemma}{LYM-неравенство}{}
    Lubell (1966), Yamamoto (1954), Meshalkin (1963)

    Пусть \( A \) --- антицепь в \( \BB^n \).
    Обозначим \( a_k = \left| A \cap B_k \right| \)

    Тогда \( \sum\limits_{k = 0}^{n} \frac{a_k}{C_{n}^{k}} \leq 1 \)
\end{lemma}

Зададимся вопросами:
\begin{enumerate}
    \item
        Сколько всего цепей размера \( n + 1 \) в \( \BB^n \)?

        Ответ: \( n! \)
    \item
        Сколько есть цепей размера \( n + 1 \) в \( \BB^n \) проходят через \( \tilde{a} \)

        Ответ: \( k! (n - k)! \)
\end{enumerate}

\begin{gather*}
    n! \geq \sum\limits_{a \in A} |a|! \cdot (n - |a|)! = \sum\limits_{k = 0}^n k! (n - k)! a_k
    \\
    \Downarrow
    \\
    \sum\limits_{k = 0}^{n} \frac{a_k}{C_{n}^{k}} \leq 1
\end{gather*}

Вернемся к доказательству теоремы Шпернера:
\begin{gather*}
    \sum\limits_{k = 0}^{n} \frac{a_k}{C_{n}^{\left\lfloor \frac{n}{2} \right\rfloor}}
        \leq \sum\limits_{k = 0}^{n} \frac{a_k}{C_{n}^{k}} \leq 1
    \\
    \Downarrow
    \\
    |A| = \sum\limits_{k = 0}^n a_k \leq C_{n}^{\left\lfloor \frac{n}{2} \right\rfloor}
    \\
\end{gather*}

\subsection{Графы сравнимости}

\begin{defn}{}{}
    Пусть \( (P, \leq) \) --- конечный ЧУМ.
    Его графом сравнимости называется граф \( G_P \): \( V = P, \ \{ x, y \} \in E \Leftrightarrow x < y \lor y < x \)

    Граф сравнимости \( G \)
\end{defn}

\begin{defn}{}{}
    \( G \) --- совершенный, если \( \forall \) индуцированного подграфа \( H \subseteq G \) \( \mathcal{X}(H) = \omega(H) \)
\end{defn}

\begin{defn}{}{}
    Для \( \forall \) конечного ЧУМ-а \( P \) его граф сравнимости \( G_p \) и \( \overline{G_p} \) --- совершенные графы.
\end{defn}

\begin{thrm}{Strong Perfect Graph Theorem}{}
    Граф \( G \) является совершенным, если и только если среди его индуцированных подграфов нет ни \( C_m \),
    ни \( \overline{C_m} \) для нечётного \( m > 3 \).
\end{thrm}